\documentclass[12pt]{article}
\usepackage[utf8]{inputenc}
\usepackage[T1]{fontenc}
\usepackage[brazilian]{babel}
\usepackage{tikz}
\usepackage{pgfplots}
\usepackage{xcolor}
\usepackage{colortbl}
\usepackage{hyperref}
\usepackage{amsmath}
\usepackage{geometry}

\geometry{a4paper, margin=2.5cm}

\title{Comparação de Utilização da CPU em Diferentes Mecanismos de Sincronização}
\author{Sistemas Operacionais}
\date{\today}

\pgfplotsset{compat=1.18}

\definecolor{running}{RGB}{65,165,65}
\definecolor{waiting}{RGB}{230,230,230}
\definecolor{blocked}{RGB}{200,80,80}
\definecolor{syscall}{RGB}{100,100,255}
\definecolor{overhead}{RGB}{255,165,0}

\begin{document}

\maketitle

\section{Introdução}

Este documento apresenta uma representação visual da utilização da CPU quando diferentes mecanismos de sincronização são utilizados na resolução do problema do produtor-consumidor. As visualizações demonstram como cada abordagem afeta o comportamento dos núcleos da CPU, trocas de contexto, interrupções e utilização geral dos recursos.

\section{Comparação Visual de Utilização da CPU}

\subsection{Mutex com Busy Waiting}

\begin{figure}[h]
\centering
\begin{tikzpicture}
\begin{axis}[
    width=\textwidth,
    height=5cm,
    xlabel={Tempo (ms)},
    ylabel={Thread},
    ytick={1,2},
    yticklabels={Produtor, Consumidor},
    xmin=0, xmax=100,
    ymin=0, ymax=3,
    legend style={at={(0.5,-0.2)}, anchor=north, legend columns=4},
    grid=both,
    minor grid style={dotted},
]

% Produtor com busy waiting
\addplot[running, line width=8pt] coordinates {(0,1) (10,1)};
\addplot[waiting, line width=8pt] coordinates {(10,1) (15,1)};
\addplot[syscall, line width=8pt] coordinates {(15,1) (17,1)};
\addplot[blocked, line width=8pt] coordinates {(17,1) (55,1)};
\addplot[syscall, line width=8pt] coordinates {(55,1) (57,1)};
\addplot[running, line width=8pt] coordinates {(57,1) (67,1)};
\addplot[waiting, line width=8pt] coordinates {(67,1) (75,1)};
\addplot[syscall, line width=8pt] coordinates {(75,1) (77,1)};
\addplot[blocked, line width=8pt] coordinates {(77,1) (100,1)};

% Consumidor com busy waiting
\addplot[blocked, line width=8pt] coordinates {(0,2) (15,2)};
\addplot[syscall, line width=8pt] coordinates {(15,2) (17,2)};
\addplot[running, line width=8pt] coordinates {(17,2) (30,2)};
\addplot[waiting, line width=8pt] coordinates {(30,2) (35,2)};
\addplot[syscall, line width=8pt] coordinates {(35,2) (37,2)};
\addplot[blocked, line width=8pt] coordinates {(37,2) (55,2)};
\addplot[syscall, line width=8pt] coordinates {(55,2) (57,2)};
\addplot[running, line width=8pt] coordinates {(57,2) (70,2)};
\addplot[syscall, line width=8pt] coordinates {(70,2) (72,2)};
\addplot[blocked, line width=8pt] coordinates {(72,2) (100,2)};

\addlegendentry{Executando}
\addlegendentry{Esperando}
\addlegendentry{Syscall}
\addlegendentry{Bloqueado}
\end{axis}
\end{tikzpicture}
\caption{Mutex com Busy Waiting: Alto número de chamadas de sistema e transições entre estados}
\end{figure}

\subsection{Mutex com Timed Sleep}

\begin{figure}[h]
\centering
\begin{tikzpicture}
\begin{axis}[
    width=\textwidth,
    height=5cm,
    xlabel={Tempo (ms)},
    ylabel={Thread},
    ytick={1,2},
    yticklabels={Produtor, Consumidor},
    xmin=0, xmax=100,
    ymin=0, ymax=3,
    legend style={at={(0.5,-0.2)}, anchor=north, legend columns=5},
    grid=both,
    minor grid style={dotted},
]

% Produtor com mutex/sleep
\addplot[running, line width=8pt] coordinates {(0,1) (10,1)};
\addplot[syscall, line width=8pt] coordinates {(10,1) (12,1)};
\addplot[overhead, line width=8pt] coordinates {(12,1) (15,1)};
\addplot[blocked, line width=8pt] coordinates {(15,1) (55,1)};
\addplot[syscall, line width=8pt] coordinates {(55,1) (57,1)};
\addplot[running, line width=8pt] coordinates {(57,1) (70,1)};
\addplot[syscall, line width=8pt] coordinates {(70,1) (72,1)};
\addplot[overhead, line width=8pt] coordinates {(72,1) (75,1)};
\addplot[blocked, line width=8pt] coordinates {(75,1) (100,1)};

% Consumidor com mutex/sleep
\addplot[blocked, line width=8pt] coordinates {(0,2) (15,2)};
\addplot[syscall, line width=8pt] coordinates {(15,2) (17,2)};
\addplot[running, line width=8pt] coordinates {(17,2) (30,2)};
\addplot[syscall, line width=8pt] coordinates {(30,2) (32,2)};
\addplot[overhead, line width=8pt] coordinates {(32,2) (35,2)};
\addplot[blocked, line width=8pt] coordinates {(35,2) (55,2)};
\addplot[syscall, line width=8pt] coordinates {(55,2) (57,2)};
\addplot[running, line width=8pt] coordinates {(57,2) (75,2)};
\addplot[blocked, line width=8pt] coordinates {(75,2) (100,2)};

\addlegendentry{Executando}
\addlegendentry{Syscall}
\addlegendentry{Overhead}
\addlegendentry{Bloqueado}
\addlegendentry{}
\end{axis}
\end{tikzpicture}
\caption{Mutex com Sleep: Menos chamadas de sistema, mas ainda tem overhead de verificação periódica}
\end{figure}

\subsection{Semáforos}

\begin{figure}[h]
\centering
\begin{tikzpicture}
\begin{axis}[
    width=\textwidth,
    height=5cm,
    xlabel={Tempo (ms)},
    ylabel={Thread},
    ytick={1,2},
    yticklabels={Produtor, Consumidor},
    xmin=0, xmax=100,
    ymin=0, ymax=3,
    legend style={at={(0.5,-0.2)}, anchor=north, legend columns=4},
    grid=both,
    minor grid style={dotted},
]

% Produtor com semáforos
\addplot[running, line width=8pt] coordinates {(0,1) (10,1)};
\addplot[syscall, line width=8pt] coordinates {(10,1) (12,1)};
\addplot[blocked, line width=8pt] coordinates {(12,1) (55,1)};
\addplot[syscall, line width=8pt] coordinates {(55,1) (57,1)};
\addplot[running, line width=8pt] coordinates {(57,1) (70,1)};
\addplot[syscall, line width=8pt] coordinates {(70,1) (72,1)};
\addplot[blocked, line width=8pt] coordinates {(72,1) (100,1)};

% Consumidor com semáforos
\addplot[syscall, line width=8pt] coordinates {(0,2) (2,2)};
\addplot[blocked, line width=8pt] coordinates {(2,2) (10,2)};
\addplot[syscall, line width=8pt] coordinates {(10,2) (12,2)};
\addplot[running, line width=8pt] coordinates {(12,2) (30,2)};
\addplot[syscall, line width=8pt] coordinates {(30,2) (32,2)};
\addplot[blocked, line width=8pt] coordinates {(32,2) (70,2)};
\addplot[syscall, line width=8pt] coordinates {(70,2) (72,2)};
\addplot[running, line width=8pt] coordinates {(72,2) (100,2)};

\addlegendentry{Executando}
\addlegendentry{Syscall}
\addlegendentry{Bloqueado}
\addlegendentry{}
\end{axis}
\end{tikzpicture}
\caption{Semáforos: Bloqueio eficiente com sinalização direta quando recursos estão disponíveis}
\end{figure}

\subsection{Variáveis de Condição}

\begin{figure}[h]
\centering
\begin{tikzpicture}
\begin{axis}[
    width=\textwidth,
    height=5cm,
    xlabel={Tempo (ms)},
    ylabel={Thread},
    ytick={1,2},
    yticklabels={Produtor, Consumidor},
    xmin=0, xmax=100,
    ymin=0, ymax=3,
    legend style={at={(0.5,-0.2)}, anchor=north, legend columns=4},
    grid=both,
    minor grid style={dotted},
]

% Produtor com variáveis de condição
\addplot[running, line width=8pt] coordinates {(0,1) (10,1)};
\addplot[syscall, line width=8pt] coordinates {(10,1) (12,1)};
\addplot[blocked, line width=8pt] coordinates {(12,1) (55,1)};
\addplot[syscall, line width=8pt] coordinates {(55,1) (57,1)};
\addplot[running, line width=8pt] coordinates {(57,1) (70,1)};
\addplot[syscall, line width=8pt] coordinates {(70,1) (72,1)};
\addplot[blocked, line width=8pt] coordinates {(72,1) (100,1)};

% Consumidor com variáveis de condição
\addplot[syscall, line width=8pt] coordinates {(0,2) (2,2)};
\addplot[blocked, line width=8pt] coordinates {(2,2) (10,2)};
\addplot[syscall, line width=8pt] coordinates {(10,2) (12,2)};
\addplot[running, line width=8pt] coordinates {(12,2) (30,2)};
\addplot[syscall, line width=8pt] coordinates {(30,2) (32,2)};
\addplot[blocked, line width=8pt] coordinates {(32,2) (70,2)};
\addplot[syscall, line width=8pt] coordinates {(70,2) (72,2)};
\addplot[running, line width=8pt] coordinates {(72,2) (100,2)};

\addlegendentry{Executando}
\addlegendentry{Syscall}
\addlegendentry{Bloqueado}
\addlegendentry{}
\end{axis}
\end{tikzpicture}
\caption{Variáveis de Condição: Padrão similar aos semáforos, mas com maior flexibilidade para condições complexas}
\end{figure}

\section{Comparativo de Interrupções e Trocas de Contexto}

\begin{figure}[h]
\centering
\begin{tikzpicture}
\begin{axis}[
    width=\textwidth,
    height=8cm,
    ylabel={Número médio de eventos},
    ybar,
    bar width=15pt,
    symbolic x coords={Mutex/Busy, Mutex/Sleep, Semáforos, Condição},
    xtick=data,
    legend style={at={(0.5,-0.15)}, anchor=north, legend columns=3},
    nodes near coords,
    nodes near coords align={vertical},
    ymin=0,
    grid=both,
    minor grid style={dotted},
]

% Número de chamadas de sistema
\addplot coordinates {
    (Mutex/Busy,42)
    (Mutex/Sleep,18)
    (Semáforos,8)
    (Condição,8)
};

% Número de trocas de contexto
\addplot coordinates {
    (Mutex/Busy,38)
    (Mutex/Sleep,12)
    (Semáforos,6)
    (Condição,6)
};

% Ciclos de CPU desperdiçados
\addplot coordinates {
    (Mutex/Busy,450)
    (Mutex/Sleep,120)
    (Semáforos,40)
    (Condição,45)
};

\addlegendentry{Chamadas de sistema}
\addlegendentry{Trocas de contexto}
\addlegendentry{Ciclos desperdiçados (×1000)}

\end{axis}
\end{tikzpicture}
\caption{Comparativo de eventos de sistema entre as diferentes abordagens de sincronização}
\end{figure}

\section{Utilização de CPU por Tipo de Operação}

\begin{figure}[h]
\centering
\begin{tikzpicture}
\begin{axis}[
    width=\textwidth,
    height=8cm,
    ylabel={Percentual de tempo},
    ybar stacked,
    bar width=25pt,
    symbolic x coords={Mutex/Busy, Mutex/Sleep, Semáforos, Condição},
    xtick=data,
    ytick={0,20,40,60,80,100},
    yticklabels={0\%,20\%,40\%,60\%,80\%,100\%},
    legend style={at={(0.5,-0.15)}, anchor=north, legend columns=5},
    ymin=0, ymax=100,
    grid=both,
    minor grid style={dotted},
]

% Tempo executando trabalho útil
\addplot coordinates {
    (Mutex/Busy,35)
    (Mutex/Sleep,45)
    (Semáforos,60)
    (Condição,58)
};

% Tempo em chamadas de sistema
\addplot coordinates {
    (Mutex/Busy,15)
    (Mutex/Sleep,10)
    (Semáforos,8)
    (Condição,9)
};

% Tempo em busy waiting
\addplot coordinates {
    (Mutex/Busy,30)
    (Mutex/Sleep,5)
    (Semáforos,0)
    (Condição,0)
};

% Tempo em sleep/blocked
\addplot coordinates {
    (Mutex/Busy,15)
    (Mutex/Sleep,35)
    (Semáforos,30)
    (Condição,30)
};

% Tempo em overhead de sincronização
\addplot coordinates {
    (Mutex/Busy,5)
    (Mutex/Sleep,5)
    (Semáforos,2)
    (Condição,3)
};

\addlegendentry{Trabalho útil}
\addlegendentry{Syscalls}
\addlegendentry{Busy waiting}
\addlegendentry{Bloqueado}
\addlegendentry{Overhead}

\end{axis}
\end{tikzpicture}
\caption{Distribuição do tempo de CPU entre diferentes tipos de operações}
\end{figure}

\section{Análise de Impacto no Cache}

\begin{figure}[h]
\centering
\begin{tikzpicture}
\begin{axis}[
    width=\textwidth,
    height=6cm,
    xlabel={Tempo de execução (ms)},
    ylabel={Falhas de cache por segundo},
    legend style={at={(0.98,0.98)}, anchor=north east},
    grid=both,
    minor grid style={dotted},
]

% Mutex com Busy Waiting
\addplot[red, thick] coordinates {
    (0,0)
    (10,300)
    (20,800)
    (30,1200)
    (40,1500)
    (50,1800)
    (60,2000)
    (70,2200)
    (80,2300)
    (90,2400)
    (100,2450)
};

% Mutex com Sleep
\addplot[blue, thick] coordinates {
    (0,0)
    (10,200)
    (20,500)
    (30,700)
    (40,900)
    (50,1000)
    (60,1100)
    (70,1150)
    (80,1200)
    (90,1250)
    (100,1300)
};

% Semáforos
\addplot[green, thick] coordinates {
    (0,0)
    (10,150)
    (20,300)
    (30,400)
    (40,500)
    (50,550)
    (60,600)
    (70,630)
    (80,650)
    (90,670)
    (100,690)
};

% Variáveis de condição
\addplot[purple, thick] coordinates {
    (0,0)
    (10,170)
    (20,320)
    (30,430)
    (40,520)
    (50,570)
    (60,610)
    (70,640)
    (80,660)
    (90,680)
    (100,700)
};

\addlegendentry{Mutex/Busy}
\addlegendentry{Mutex/Sleep}
\addlegendentry{Semáforos}
\addlegendentry{Condição}

\end{axis}
\end{tikzpicture}
\caption{Falhas de cache ao longo do tempo para diferentes mecanismos de sincronização}
\end{figure}

\section{Conclusão}

As representações visuais apresentadas neste documento ilustram claramente por que abordagens como semáforos e variáveis de condição geralmente oferecem melhor desempenho do que soluções baseadas apenas em mutex, especialmente em cenários com alta contenção:

\begin{itemize}
    \item \textbf{Mutex com busy waiting}: Gera alto número de interrupções, chamadas de sistema e desperdício de ciclos de CPU.
    \item \textbf{Mutex com sleep}: Reduz o desperdício de CPU em comparação ao busy waiting puro, mas ainda tem overhead significativo de verificação periódica.
    \item \textbf{Semáforos e variáveis de condição}: Minimizam chamadas de sistema e trocas de contexto através de bloqueio eficiente e sinalização direta.
\end{itemize}

No problema do produtor-consumidor, onde frequentemente threads precisam esperar que outras completem certas operações, os mecanismos que permitem bloqueio eficiente e sinalização direta entre threads oferecem vantagens significativas de desempenho e utilização de recursos.

\end{document}

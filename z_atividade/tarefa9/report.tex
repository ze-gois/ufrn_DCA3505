\documentclass[12pt]{article}
\usepackage[utf8]{inputenc}
\usepackage[T1]{fontenc}
\usepackage[brazilian]{babel}
\usepackage{listings}
\usepackage{color}
\usepackage{xcolor}
\usepackage{graphicx}
\usepackage{hyperref}
\usepackage{amsmath}
\usepackage{amssymb}

\definecolor{codegreen}{rgb}{0,0.6,0}
\definecolor{codegray}{rgb}{0.5,0.5,0.5}
\definecolor{codepurple}{rgb}{0.58,0,0.82}
\definecolor{backcolour}{rgb}{0.95,0.95,0.92}

\lstdefinestyle{mystyle}{
    backgroundcolor=\color{backcolour},
    commentstyle=\color{codegreen},
    keywordstyle=\color{blue},
    numberstyle=\tiny\color{codegray},
    stringstyle=\color{codepurple},
    basicstyle=\footnotesize\ttfamily,
    breakatwhitespace=false,
    breaklines=true,
    captionpos=b,
    keepspaces=true,
    numbers=left,
    numbersep=5pt,
    showspaces=false,
    showstringspaces=false,
    showtabs=false,
    tabsize=2
}

\lstset{style=mystyle}

\title{Tarefa 9: Produtor e Consumidor}
\author{Sistemas Operacionais}
\date{\today}

\begin{document}

\maketitle

\section{Problema Fundamental da Implementação}

O problema fundamental na implementação fornecida do produtor e consumidor está relacionado à ausência de sincronização adequada em um cenário de concorrência. As variáveis compartilhadas entre as threads (dados[], inserir e remover) são acessadas sem mecanismos de sincronização adequados, o que pode levar aos seguintes problemas:

\subsection{Race conditions (Condições de corrida)}
Quando múltiplas threads tentam acessar e modificar as mesmas variáveis simultaneamente, o resultado final pode ser inconsistente. Por exemplo, dois consumidores podem tentar ler o mesmo valor simultaneamente, resultando em comportamento indefinido.

\subsection{Acesso não atômico}
As operações de leitura e escrita das variáveis compartilhadas não são atômicas. Isso significa que uma thread pode ser interrompida no meio da atualização dos índices inserir ou remover, levando a inconsistências.

\subsection{Busy waiting ineficiente}
O código utiliza busy waiting (espera ocupada) nos loops:
\begin{lstlisting}[language=C]
// No produtor
while (((inserir + 1) % TAMANHO) == remover);

// No consumidor
while (inserir == remover);
\end{lstlisting}

Isso desperdiça ciclos de CPU, pois as threads continuam executando enquanto esperam que as condições sejam satisfeitas, em vez de liberar o processador para outras tarefas.

\section{Solução com Mutexes}

Para resolver os problemas de sincronização, podemos usar mutexes para proteger o acesso às seções críticas:

\begin{lstlisting}[language=C]
pthread_mutex_t buffer_mutex = PTHREAD_MUTEX_INITIALIZER;

void *produtor(void *arg) {
    int v;
    for (v = 1;; v++) {
        int can_insert = 0;

        while (!can_insert) {
            pthread_mutex_lock(&buffer_mutex);

            if (((inserir + 1) % TAMANHO) != remover) {
                can_insert = 1;
                printf("Produzindo %d\n", v);
                dados[inserir] = v;
                inserir = (inserir + 1) % TAMANHO;
            }

            pthread_mutex_unlock(&buffer_mutex);

            if (!can_insert) {
                usleep(10000);  // Reduz uso de CPU
            }
        }

        usleep(500000);
    }
    return NULL;
}

void *consumidor(void *arg) {
    for (;;) {
        int can_consume = 0;

        while (!can_consume) {
            pthread_mutex_lock(&buffer_mutex);

            if (inserir != remover) {
                can_consume = 1;
                printf("%zu: Consumindo %d\n", (size_t)arg, dados[remover]);
                remover = (remover + 1) % TAMANHO;
            }

            pthread_mutex_unlock(&buffer_mutex);

            if (!can_consume) {
                usleep(10000);  // Reduz uso de CPU
            }
        }
    }
    return NULL;
}
\end{lstlisting}

Esta solução protege os acessos às variáveis compartilhadas usando mutexes, garantindo que apenas uma thread por vez execute a seção crítica. Adicionamos um sleep quando o buffer está cheio ou vazio para reduzir o consumo de CPU, mas ainda mantemos uma forma de busy waiting.

\section{Problemas de Desempenho com Mutexes}

A solução com mutexes resolve o problema de condições de corrida, mas apresenta problemas de desempenho:

\subsection{Busy waiting modificado}
Mesmo com o sleep dentro do loop, as threads ainda estão em um padrão de espera ocupada. Elas acordam periodicamente para verificar se a condição mudou, desperdiçando ciclos de CPU.

\subsection{Overhead de bloqueio/desbloqueio}
O mutex está sendo adquirido e liberado repetidamente em cada iteração do loop, mesmo quando a condição não permite o progresso. Isso gera um overhead desnecessário de operações de bloqueio e desbloqueio.

\subsection{Falta de sinalização}
Não há um mecanismo direto para que o produtor sinalize aos consumidores quando novos dados estiverem disponíveis, ou para os consumidores sinalizarem ao produtor quando houver espaço disponível, forçando verificações periódicas.

\subsection{Escalabilidade limitada}
À medida que o número de threads aumenta, a contenção pelo mutex também aumenta, resultando em maior overhead de sincronização e possivelmente maior contention dos recursos do sistema.

\section{Solução com Semáforos}

Os semáforos permitem uma sincronização mais eficiente, removendo a necessidade de busy waiting:

\begin{lstlisting}[language=C]
sem_t empty_slots;  // Contador de espaços vazios
sem_t filled_slots; // Contador de espaços preenchidos
pthread_mutex_t buffer_mutex = PTHREAD_MUTEX_INITIALIZER;

// Inicialização
sem_init(&empty_slots, 0, TAMANHO - 1);
sem_init(&filled_slots, 0, 0);

void *produtor(void *arg) {
    int v;
    for (v = 1;; v++) {
        // Espera até haver um slot vazio
        sem_wait(&empty_slots);

        // Protege a seção crítica
        pthread_mutex_lock(&buffer_mutex);

        printf("Produzindo %d\n", v);
        dados[inserir] = v;
        inserir = (inserir + 1) % TAMANHO;

        pthread_mutex_unlock(&buffer_mutex);

        // Sinaliza que há um novo item
        sem_post(&filled_slots);

        usleep(500000);
    }
    return NULL;
}

void *consumidor(void *arg) {
    for (;;) {
        // Espera até haver um item disponível
        sem_wait(&filled_slots);

        // Protege a seção crítica
        pthread_mutex_lock(&buffer_mutex);

        printf("%zu: Consumindo %d\n", (size_t)arg, dados[remover]);
        remover = (remover + 1) % TAMANHO;

        pthread_mutex_unlock(&buffer_mutex);

        // Sinaliza que há um novo slot vazio
        sem_post(&empty_slots);
    }
    return NULL;
}
\end{lstlisting}

Esta solução usa dois semáforos:
\begin{itemize}
    \item \textbf{empty\_slots}: Conta o número de slots vazios no buffer.
    \item \textbf{filled\_slots}: Conta o número de slots preenchidos no buffer.
\end{itemize}

Vantagens desta solução:
\begin{itemize}
    \item \textbf{Sem busy waiting}: As threads são bloqueadas pelo sistema operacional até que a operação \texttt{sem\_wait()} possa prosseguir.
    \item \textbf{Sinalização direta}: Existe uma sinalização explícita quando novos itens são produzidos ou slots são liberados.
    \item \textbf{Melhor eficiência}: As threads são bloqueadas quando não podem progredir, liberando o processador para outras tarefas.
    \item \textbf{Melhor escalabilidade}: Reduz a contenção pelo mutex, pois as threads só acessam o mutex quando têm garantia de que podem progredir.
\end{itemize}

\section{Solução com Variáveis de Condição}

As variáveis de condição oferecem outra abordagem eficiente para sincronização, permitindo que as threads aguardem condições específicas:

\begin{lstlisting}[language=C]
pthread_mutex_t buffer_mutex = PTHREAD_MUTEX_INITIALIZER;
pthread_cond_t not_empty = PTHREAD_COND_INITIALIZER;
pthread_cond_t not_full = PTHREAD_COND_INITIALIZER;
volatile int count = 0;  // Número de itens no buffer

void *produtor(void *arg) {
    int v;
    for (v = 1;; v++) {
        pthread_mutex_lock(&buffer_mutex);

        // Espera enquanto o buffer estiver cheio
        while (count == TAMANHO - 1)
            pthread_cond_wait(&not_full, &buffer_mutex);

        printf("Produzindo %d\n", v);
        dados[inserir] = v;
        inserir = (inserir + 1) % TAMANHO;
        count++;

        // Sinaliza que o buffer não está mais vazio
        pthread_cond_signal(&not_empty);

        pthread_mutex_unlock(&buffer_mutex);

        usleep(500000);
    }
    return NULL;
}

void *consumidor(void *arg) {
    for (;;) {
        pthread_mutex_lock(&buffer_mutex);

        // Espera enquanto o buffer estiver vazio
        while (count == 0)
            pthread_cond_wait(&not_empty, &buffer_mutex);

        printf("%zu: Consumindo %d\n", (size_t)arg, dados[remover]);
        remover = (remover + 1) % TAMANHO;
        count--;

        // Sinaliza que o buffer não está mais cheio
        pthread_cond_signal(&not_full);

        pthread_mutex_unlock(&buffer_mutex);
    }
    return NULL;
}
\end{lstlisting}

Esta solução utiliza:
\begin{itemize}
    \item Um mutex para proteger o acesso às variáveis compartilhadas.
    \item Duas variáveis de condição:
    \begin{itemize}
        \item \textbf{not\_empty}: Sinalizada quando itens são adicionados ao buffer.
        \item \textbf{not\_full}: Sinalizada quando itens são removidos do buffer.
    \end{itemize}
    \item Uma variável \texttt{count} que mantém o número atual de itens no buffer.
\end{itemize}

Vantagens desta solução:
\begin{itemize}
    \item \textbf{Eficiência}: Como os semáforos, não há busy waiting. As threads são bloqueadas eficientemente.
    \item \textbf{Flexibilidade}: As variáveis de condição permitem verificar predicados mais complexos além de simples contadores.
    \item \textbf{Atomicidade}: A função \texttt{pthread\_cond\_wait()} libera o mutex automaticamente enquanto a thread está bloqueada e o readquire quando ela acorda, garantindo atomicidade.
    \item \textbf{Sinalização precisa}: As threads são sinalizadas exatamente quando as condições mudam, evitando verificações desnecessárias.
\end{itemize}

\section{Conclusão}

O problema do produtor-consumidor ilustra desafios fundamentais de sincronização em sistemas concorrentes. Comparando as soluções:

\begin{itemize}
    \item A solução original tem problemas de corrida críticos e é ineficiente devido ao busy waiting.
    \item A solução com mutex resolve os problemas de corrida, mas mantém o problema de eficiência com um tipo modificado de busy waiting.
    \item As soluções com semáforos e variáveis de condição resolvem tanto os problemas de corrida quanto de eficiência, evitando o busy waiting.
\end{itemize}

Tanto semáforos quanto variáveis de condição são adequados para este problema, com variáveis de condição oferecendo maior flexibilidade para condições complexas. A escolha entre eles geralmente depende dos requisitos específicos do sistema e das preferências de design.

\end{document}

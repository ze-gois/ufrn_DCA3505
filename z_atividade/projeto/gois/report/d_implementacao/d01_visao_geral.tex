\section{Visão Geral}\label{sec:visao_geral}

O sistema implementado para detecção de deadlocks e violação da ordem de aquisição de travas segue uma arquitetura modular, separando claramente as responsabilidades entre diferentes componentes. Esta seção apresenta uma visão geral da arquitetura, destacando seus principais componentes e suas interações.

\subsection{Arquitetura do Sistema}

A arquitetura do sistema é composta por três componentes principais:

\begin{enumerate}
    \item \textbf{Biblioteca de Análise de Grafos}: Responsável pela representação e manipulação do grafo de dependências entre locks, incluindo algoritmos para detecção de ciclos.

    \item \textbf{Núcleo de Detecção (lockdep\_core)}: Implementa a lógica central de monitoramento de locks, gerenciamento de contextos de thread e detecção de violações.

    \item \textbf{Camada de Interposição (pthread\_interpose)}: Intercepta chamadas às funções de mutex da biblioteca pthread e as redireciona para o núcleo de detecção.
\end{enumerate}

Esta separação de responsabilidades permite que cada componente seja desenvolvido, testado e mantido de forma independente, facilitando modificações e extensões futuras.

\begin{figure}[h]
    \centering
    % Placeholder para diagrama da arquitetura
    % \includegraphics[width=0.8\textwidth]{diagrama_arquitetura.png}
    \caption{Arquitetura do sistema de detecção de deadlocks}
    \label{fig:arquitetura}
\end{figure}

\subsection{Fluxo de Execução}

O fluxo de execução do sistema segue os seguintes passos:

\begin{enumerate}
    \item A biblioteca compartilhada contendo o sistema é carregada antes de qualquer outra biblioteca através do mecanismo LD\_PRELOAD.

    \item As funções construtoras são executadas, inicializando as estruturas de dados necessárias para o sistema de detecção.

    \item Quando uma aplicação chama funções como \texttt{pthread\_mutex\_lock()}, a versão interposta desta função é executada em vez da implementação original.

    \item A função interposta notifica o núcleo de detecção sobre a operação de lock, que por sua vez:
    \begin{itemize}
        \item Atualiza o grafo de dependências entre locks
        \item Verifica se a operação viola alguma ordem de aquisição previamente estabelecida
        \item Analisa o grafo em busca de ciclos que indiquem potenciais deadlocks
    \end{itemize}

    \item Se nenhuma violação for detectada, a função original é chamada para realizar a operação de lock real.

    \item Se uma violação for detectada, o sistema pode:
    \begin{itemize}
        \item Emitir um aviso detalhando o problema
        \item Recusar a operação de lock, retornando um código de erro apropriado
        \item Coletar informações de diagnóstico como stacktrace para auxiliar na depuração
    \end{itemize}
\end{enumerate}

\subsection{Integração com Aplicações}

Uma característica fundamental do sistema é sua transparência para as aplicações monitoradas. O sistema:

\begin{itemize}
    \item Não requer modificações no código-fonte das aplicações
    \item Pode ser ativado ou desativado através de variáveis de ambiente
    \item Introduz um overhead mínimo quando usado em modo de produção
    \item Fornece diagnósticos detalhados quando uma violação é detectada
\end{itemize}

Esta abordagem facilita a adoção do sistema tanto em ambientes de desenvolvimento quanto de produção, permitindo que desenvolvedores identifiquem e corrijam problemas de sincronização antes que causem falhas reais.

\subsection{Considerações de Design}

No desenvolvimento do sistema, algumas considerações importantes de design incluem:

\begin{itemize}
    \item \textbf{Eficiência}: Minimizar o overhead introduzido pelo monitoramento, especialmente em operações frequentes como aquisição e liberação de locks.

    \item \textbf{Precisão}: Garantir que as violações detectadas representem problemas reais, minimizando falsos positivos.

    \item \textbf{Escalabilidade}: Suportar aplicações com grande número de threads e locks sem degradação significativa de desempenho.

    \item \textbf{Robustez}: O próprio sistema de detecção não deve introduzir novos problemas de concorrência ou deadlocks.
\end{itemize}

Nas seções seguintes, detalhamos cada componente do sistema, apresentando suas estruturas de dados, algoritmos e técnicas de implementação.

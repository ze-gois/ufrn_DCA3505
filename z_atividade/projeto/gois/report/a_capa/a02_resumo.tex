\begin{abstract}
Este trabalho apresenta a implementação de um sistema de detecção de deadlocks e violação da ordem de aquisição de travas, inspirado no lockdep do kernel Linux. O sistema monitora operações de mutex em programas multithreaded, construindo um grafo de dependências entre travas e detectando potenciais situações de deadlock antes que ocorram. Utilizamos duas abordagens complementares: (1) detecção de ciclos no grafo de dependências, que indica um deadlock potencial, e (2) verificação da consistência na ordem de aquisição de travas entre diferentes threads. O sistema foi implementado através de duas técnicas distintas: uma biblioteca compartilhada que utiliza interposição de funções para interceptar chamadas à API pthread, e uma ferramenta baseada em ptrace capaz de analisar processos em execução, incluindo aqueles já em estado de deadlock. Ambas as implementações compartilham uma biblioteca modular de análise de grafos. Os testes realizados demonstram a eficácia do sistema na detecção de deadlocks clássicos como o problema AB-BA, onde duas threads tentam adquirir os mesmos mutexes em ordens diferentes, com cada abordagem apresentando vantagens específicas para diferentes cenários. Este trabalho contribui para o desenvolvimento de ferramentas que aumentam a robustez de sistemas concorrentes, ajudando programadores a identificar problemas de sincronização difíceis de detectar.

\vspace{0.5cm}

\noindent\textbf{Palavras-chave:} deadlock, mutex, concorrência, detecção de erros, grafo de dependência, sistemas operacionais, ptrace, interposição de funções.
\end{abstract}

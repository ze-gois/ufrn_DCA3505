\documentclass[a4paper,12pt]{report}

% Basic packages for LuaLaTeX with Unicode support
% No need for inputenc with LuaLaTeX as it supports UTF-8 natively
\usepackage{fontspec}
\usepackage[brazil]{babel}
\usepackage{graphicx}
\usepackage{hyperref}
\usepackage{url}
\usepackage[numbers]{natbib}
\usepackage{amsmath}
\usepackage{amssymb}
\usepackage{listings}  % Regular listings is fine with LuaLaTeX
\usepackage{xcolor}
\usepackage{geometry}
\usepackage{indentfirst}
\usepackage{setspace}
\usepackage{cleveref}  % For enhanced cross-referencing
\usepackage{caption}   % For better caption handling
\usepackage{float}     % For improved figure/table placement
\usepackage{luatextra} % Extra features for LuaLaTeX
\usepackage[strings]{underscore}
% Hyperref configuration for better PDF output
\hypersetup{
    colorlinks=true,
    linkcolor=blue,
    filecolor=magenta,
    urlcolor=cyan,
    citecolor=green,
    pdftitle={Estudo acerca de memória no x86},
    pdfauthor={José Henrique Targino Dias Gois},
    pdfsubject={Memória no x86},
    pdfkeywords={sistemas operacionais, sistemas de memória, intel x86}
}

% Code listing styling
\definecolor{codegreen}{rgb}{0,0.6,0}
\definecolor{codegray}{rgb}{0.5,0.5,0.5}
\definecolor{codepurple}{rgb}{0.58,0,0.82}
\definecolor{backcolour}{rgb}{0.95,0.95,0.92}

\lstdefinestyle{mystyle}{
    backgroundcolor=\color{backcolour},
    commentstyle=\color{codegreen},
    keywordstyle=\color{magenta},
    numberstyle=\tiny\color{codegray},
    stringstyle=\color{codepurple},
    basicstyle=\ttfamily\footnotesize,
    breakatwhitespace=false,
    breaklines=true,
    captionpos=b,
    keepspaces=true,
    numbers=left,
    numbersep=5pt,
    showspaces=false,
    showstringspaces=false,
    showtabs=false,
    tabsize=2,
    extendedchars=true,
    texcl=true,
    inputencoding=utf8
}

\lstset{style=mystyle}

\geometry{margin=2.5cm}
\onehalfspacing

% Custom command for easier section referencing
\newcommand{\secref}[1]{Seção\ref{#1}}
\newcommand{\figref}[1]{Figura\ref{#1}}
\newcommand{\tabref}[1]{Tabela\ref{#1}}
\newcommand{\eqnref}[1]{Equação\ref{#1}}
\newcommand{\chapref}[1]{Capítulo\ref{#1}}

% Document start
\begin{document}

% Cover page
\begin{titlepage}
    \centering
    \vspace*{1cm}

    \textbf{\LARGE Universidade Federal do Rio Grande do Norte}\\
    \textbf{\large Departamento de Engenharia de Computação e Automação}\\
    \textbf{\large DCA0121 - Sistemas Operacionais}\\
    \vspace{1.5cm}

    \textbf{\Huge Lockdep}\\
    \vspace{0.5cm}
    \textbf{\Large Detecção de Deadlocks e Violação da Ordem de Aquisição de Travas}\\
    \vspace{1.5cm}

    \begin{figure}[h]
        \centering
        % Add UFRN logo here if available
        % \includegraphics[width=0.4\textwidth]{ufrn_logo.png}
    \end{figure}

    \vspace{1.5cm}

    \begin{flushright}
        \textbf{Desenvolvido por:}\\
        Nome do Aluno\\
        Matrícula: 000000000
    \end{flushright}

    \vfill

    \textbf{\large Natal - RN}\\
    \textbf{\large \today}

\end{titlepage}


% Abstracts
\begin{abstract}
Este trabalho apresenta a implementação de um sistema de detecção de deadlocks e violação da ordem de aquisição de travas, inspirado no lockdep do kernel Linux. O sistema monitora operações de mutex em programas multithreaded, construindo um grafo de dependências entre travas e detectando potenciais situações de deadlock antes que ocorram. Utilizamos duas abordagens complementares: (1) detecção de ciclos no grafo de dependências, que indica um deadlock potencial, e (2) verificação da consistência na ordem de aquisição de travas entre diferentes threads. O sistema foi implementado através de duas técnicas distintas: uma biblioteca compartilhada que utiliza interposição de funções para interceptar chamadas à API pthread, e uma ferramenta baseada em ptrace capaz de analisar processos em execução, incluindo aqueles já em estado de deadlock. Ambas as implementações compartilham uma biblioteca modular de análise de grafos. Os testes realizados demonstram a eficácia do sistema na detecção de deadlocks clássicos como o problema AB-BA, onde duas threads tentam adquirir os mesmos mutexes em ordens diferentes, com cada abordagem apresentando vantagens específicas para diferentes cenários. Este trabalho contribui para o desenvolvimento de ferramentas que aumentam a robustez de sistemas concorrentes, ajudando programadores a identificar problemas de sincronização difíceis de detectar.

\vspace{0.5cm}

\noindent\textbf{Palavras-chave:} deadlock, mutex, concorrência, detecção de erros, grafo de dependência, sistemas operacionais, ptrace, interposição de funções.
\end{abstract}

\begin{abstract}
\selectlanguage{american}
This work presents the implementation of a system for detecting deadlocks and lock acquisition order violations, inspired by the Linux kernel's lockdep. The system monitors mutex operations in multithreaded programs, building a dependency graph between locks and detecting potential deadlock situations before they occur. We use two complementary approaches: (1) cycle detection in the dependency graph, which indicates a potential deadlock, and (2) verification of consistency in the order of lock acquisition across different threads. The system was implemented using two distinct techniques: a shared library that uses function interposition to intercept calls to the pthread API, and a ptrace-based tool capable of analyzing running processes, including those already in a deadlock state. Both implementations share a modular graph analysis library. Tests performed demonstrate the effectiveness of the system in detecting classic deadlocks such as the AB-BA problem, where two threads try to acquire the same mutexes in different orders, with each approach offering specific advantages for different scenarios. This work contributes to the development of tools that increase the robustness of concurrent systems, helping programmers identify synchronization problems that are difficult to detect.

\vspace{0.5cm}

\noindent\textbf{Keywords:} deadlock, mutex, concurrency, error detection, dependency graph, operating systems, ptrace, function interposition.
\selectlanguage{brazil}
\end{abstract}


% Custom TOC as defined in a04_toc.tex
\pagenumbering{roman}

\tableofcontents
\clearpage

\listoffigures
\clearpage

\listoftables
\clearpage

\pagenumbering{arabic}


% Introduction
\chapter{Introdução}\label{chap:intro}

Este relatório apresenta o desenvolvimento de um sistema para detecção de deadlocks e violação da ordem de aquisição de travas em programas multithreaded. O sistema implementado, inspirado no lockdep do kernel Linux, monitora a aquisição de mutexes e verifica potenciais situações de deadlock antes que elas ocorram.

\section{Contextualização}\label{sec:contexto}

Em programação concorrente, o uso de mutexes (mutual exclusion) é essencial para garantir que apenas uma thread tenha acesso a recursos compartilhados por vez. No entanto, quando múltiplas threads tentam adquirir múltiplos mutexes em ordens diferentes, podem ocorrer deadlocks - uma situação onde duas ou mais threads ficam permanentemente bloqueadas, cada uma esperando por um recurso que outra possui.

Um exemplo clássico de deadlock ocorre quando:
\begin{itemize}
    \item Thread A adquire mutex1 e tenta adquirir mutex2
    \item Thread B adquire mutex2 e tenta adquirir mutex1
\end{itemize}

Este padrão, conhecido como AB-BA, cria uma dependência circular entre as threads, resultando em um deadlock. Em sistemas reais, estes deadlocks podem ser difíceis de reproduzir e diagnosticar devido à sua natureza dependente de timing.

O problema se torna ainda mais complexo em sistemas com grande número de threads e recursos compartilhados. Muitos deadlocks só ocorrem em condições específicas de carga ou timing, tornando-os particularmente difíceis de detectar durante o desenvolvimento e testes.

Diversas abordagens foram desenvolvidas para lidar com deadlocks, incluindo:

\begin{itemize}
    \item \textbf{Prevenção}: garantir que pelo menos uma das condições necessárias para deadlock não ocorra;
    \item \textbf{Evitação}: alocar recursos dinamicamente de forma a evitar estados inseguros;
    \item \textbf{Detecção}: monitorar o sistema para identificar quando deadlocks ocorrem;
    \item \textbf{Recuperação}: liberar recursos quando deadlocks são detectados.
\end{itemize}

Este trabalho foca principalmente na detecção antecipada de potenciais deadlocks através da análise da ordem de aquisição de locks e da construção de grafos de dependência entre recursos.

\section{Objetivos}\label{sec:objetivos}

O objetivo principal deste projeto é desenvolver um sistema para detectar potenciais deadlocks antes que eles ocorram em aplicações multithreaded. Especificamente, buscamos:

\begin{enumerate}
    \item Implementar um sistema de monitoramento de locks que não requeira modificação do código-fonte das aplicações monitoradas;

    \item Desenvolver e implementar algoritmos para detecção de deadlocks através de duas abordagens complementares:
    \begin{itemize}
        \item Construção e análise de grafo de espera por recursos, detectando ciclos que representam deadlocks reais;
        \item Verificação de consistência na ordem de aquisição de travas, identificando padrões que poderiam levar a deadlocks;
    \end{itemize}

    \item Criar uma infraestrutura que permita a fácil integração do sistema de detecção com aplicações existentes;

    \item Minimizar o overhead de execução do sistema de detecção, garantindo que possa ser utilizado não apenas em ambiente de desenvolvimento, mas também em ambientes de produção;

    \item Fornecer informações detalhadas sobre violações detectadas, incluindo backtrace e análise das dependências que levaram ao potencial deadlock.
\end{enumerate}

Estes objetivos se alinham com a necessidade de ferramentas que auxiliem no desenvolvimento de software concorrente robusto, permitindo que desenvolvedores identifiquem problemas de sincronização complexos antes que se manifestem como falhas em produção.

\section{Abordagem}\label{sec:abordagem}

Para atingir os objetivos propostos, nossa abordagem combina conceitos de teoria dos grafos, engenharia de software e sistemas operacionais. O projeto implementa duas estratégias distintas e complementares para detecção de deadlocks:

\subsection{Biblioteca Compartilhada com Interposição de Funções}

A primeira abordagem utiliza a técnica de interposição de funções através do mecanismo LD\_PRELOAD do sistema de vinculação dinâmica. Esta estratégia permite interceptar chamadas às funções da biblioteca pthread sem modificar o código fonte das aplicações monitoradas, tornando o sistema transparente e de fácil adoção. É ideal para uso durante desenvolvimento e testes de software.

\subsection{Ferramenta de Análise baseada em ptrace}

A segunda abordagem implementa uma ferramenta baseada na API \texttt{ptrace} do Linux, que permite anexar-se a processos em execução para monitorar suas chamadas de sistema e examinar seu estado de memória. Esta estratégia é particularmente valiosa para:

\begin{itemize}
    \item Analisar processos já em execução, sem necessidade de reiniciá-los;
    \item Detectar deadlocks em processos que já estão bloqueados;
    \item Monitorar aplicações compiladas estaticamente, onde LD\_PRELOAD não funciona;
    \item Realizar análises post-mortem de problemas de sincronização.
\end{itemize}

\subsection{Detecção de Deadlocks}

Ambas as abordagens implementam dois métodos complementares para detecção de deadlocks:

\begin{enumerate}
    \item \textbf{Grafo de espera com detecção de ciclos}:
    \begin{itemize}
        \item Construímos um grafo direcionado onde os vértices são mutexes e as arestas representam a ordem de aquisição;
        \item Utilizamos algoritmo de Busca em Profundidade (DFS) para detectar ciclos neste grafo;
        \item Um ciclo no grafo indica uma potencial situação de deadlock.
    \end{itemize}

    \item \textbf{Verificação da ordem de aquisição de travas}:
    \begin{itemize}
        \item Mantemos um histórico da ordem em que mutexes são adquiridos por cada thread;
        \item Detectamos quando uma thread tenta adquirir mutexes em uma ordem inconsistente com padrões previamente observados;
        \item Alertamos sobre estas violações antes que o deadlock realmente ocorra.
    \end{itemize}
\end{enumerate}

\subsection{Biblioteca Modular de Análise de Grafos}

Um aspecto fundamental do projeto é a separação da lógica de análise de grafos em uma biblioteca reutilizável, compartilhada por ambas as abordagens. Esta biblioteca encapsula:

\begin{itemize}
    \item A representação de grafos direcionados;
    \item Algoritmos de detecção de ciclos;
    \item Verificação proativa de operações que criariam ciclos.
\end{itemize}

Esta modularização permite concentrar a complexidade algorítmica em um único componente bem testado e otimizado.

\subsection{Estruturas de Dados Otimizadas}

Para garantir eficiência e baixo overhead durante o monitoramento, utilizamos estruturas de dados cuidadosamente projetadas:

\begin{itemize}
    \item Listas ligadas para representação do grafo, permitindo atualizações dinâmicas;
    \item Tabelas hash para acesso rápido aos nós do grafo;
    \item Contextos thread-específicos para rastrear o estado de cada thread independentemente;
    \item Estruturas específicas para a análise de backtrace na abordagem ptrace.
\end{itemize}

\subsection{Diagnóstico Detalhado}

Quando uma violação é detectada, o sistema fornece diagnósticos detalhados que incluem:

\begin{itemize}
    \item Identificação dos locks envolvidos;
    \item Backtrace completo do ponto de detecção;
    \item Visualização do estado atual do grafo de dependências;
    \item Informações sobre a ordem de aquisição recomendada;
    \item No caso da abordagem ptrace, detalhes sobre o estado das threads bloqueadas.
\end{itemize}

Esta abordagem multilateral permite que o sistema detecte efetivamente diferentes padrões de deadlock, incluindo tanto aqueles causados por espera circular direta quanto os decorrentes de violações na ordem de aquisição de locks. A combinação das duas estratégias oferece uma solução completa para detecção de deadlocks em diferentes cenários e estágios do ciclo de vida de software.


\chapter{Fundamentação Teórica}\label{chap:fundamentacao}

\section{Deadlocks}\label{sec:deadlocks}

\subsection{Definição e Condições Necessárias}

Um deadlock é uma situação em que dois ou mais processos ou threads estão bloqueados permanentemente, cada um esperando que o outro libere um recurso. Em sistemas operacionais e programação concorrente, deadlocks representam um problema crítico que pode levar à completa paralisação de partes do sistema.

Para que um deadlock ocorra, quatro condições (conhecidas como condições de Coffman) devem ser satisfeitas simultaneamente:

\begin{enumerate}
    \item \textbf{Exclusão Mútua}: Pelo menos um recurso deve estar em um estado não compartilhável, ou seja, apenas um processo pode utilizá-lo por vez.

    \item \textbf{Posse e Espera}: Um processo deve estar segurando pelo menos um recurso enquanto espera para adquirir recursos adicionais que estão sendo mantidos por outros processos.

    \item \textbf{Não Preempção}: Os recursos não podem ser removidos à força de um processo; eles devem ser liberados voluntariamente.

    \item \textbf{Espera Circular}: Deve existir um conjunto de processos \{P$_1$, P$_2$, ..., P$_n$\} onde P$_1$ está esperando por um recurso que P$_2$ possui, P$_2$ está esperando por um recurso que P$_3$ possui, e assim por diante, até P$_n$ estar esperando por um recurso que P$_1$ possui.
\end{enumerate}

Se qualquer uma dessas condições não for satisfeita, o deadlock não pode ocorrer.

\subsection{O Problema dos Filósofos Jantantes}

Um exemplo clássico de deadlock é o problema dos filósofos jantantes, proposto por Dijkstra. Neste problema, cinco filósofos sentam-se ao redor de uma mesa, com um garfo entre cada par de filósofos. Para comer, um filósofo precisa de dois garfos - os que estão à sua direita e à sua esquerda. Se cada filósofo pegar o garfo à sua esquerda simultaneamente, todos estarão esperando pelo garfo à sua direita, que está sendo segurado por outro filósofo, criando um deadlock.

\subsection{Deadlocks em Sistemas Multithreaded}

Em sistemas multithreaded que utilizam mutexes para sincronização, o padrão mais comum de deadlock é o "AB-BA":

\begin{itemize}
    \item Thread A adquire mutex M1
    \item Thread B adquire mutex M2
    \item Thread A tenta adquirir M2 (e é bloqueada esperando)
    \item Thread B tenta adquirir M1 (e é bloqueada esperando)
\end{itemize}

Neste cenário, ambas as threads estão permanentemente bloqueadas, cada uma esperando que a outra libere um mutex.

\subsection{Detecção de Deadlocks}

A detecção de deadlocks pode ser realizada através de diferentes abordagens:

\subsubsection{Algoritmo do Banqueiro}

O algoritmo do banqueiro, desenvolvido por Dijkstra, é utilizado para determinar se a alocação de um recurso levará o sistema a um estado seguro ou inseguro. Embora não detecte deadlocks diretamente, ele pode preveni-los garantindo que o sistema nunca entre em um estado inseguro.

\subsubsection{Detecção baseada em Grafos de Alocação de Recursos}

Esta abordagem utiliza um grafo direcionado onde:
\begin{itemize}
    \item Vértices representam processos e recursos
    \item Arestas representam alocações ou solicitações de recursos
    \item Um ciclo no grafo indica um potencial deadlock
\end{itemize}

Utilizando algoritmos como Busca em Profundidade (DFS), é possível detectar ciclos no grafo, indicando situações de deadlock.

\subsubsection{Técnica Timeout}

Uma abordagem simples é implementar timeouts nas operações de aquisição de locks. Se uma thread não conseguir adquirir um lock dentro de um determinado período, ela libera todos os recursos e tenta novamente após um tempo aleatório.

\subsubsection{Análise de Ordem de Aquisição (Lockdep)}

Esta técnica, inspirada pelo lockdep do kernel Linux, monitora a ordem em que os locks são adquiridos por diferentes threads. Se detectar inconsistências nesta ordem (por exemplo, thread A adquire locks na ordem L1→L2 e thread B na ordem L2→L1), alerta sobre um potencial deadlock antes mesmo que ele ocorra.

\subsection{Prevenção e Mitigação}

Prevenir deadlocks completamente geralmente implica em eliminar pelo menos uma das condições de Coffman:

\begin{itemize}
    \item \textbf{Quebrar a exclusão mútua}: Nem sempre possível devido a requisitos de consistência.

    \item \textbf{Evitar posse e espera}: Adquirir todos os recursos necessários de uma só vez ou não adquirir recursos adicionais quando já possuir algum.

    \item \textbf{Permitir preempção}: Implementar timeouts e mecanismos de "tentativa e recuo" para liberação de recursos em caso de impasse.

    \item \textbf{Prevenir espera circular}: Estabelecer uma ordem global para aquisição de recursos e garantir que todos os processos sigam esta ordem.
\end{itemize}

No contexto de programação multithreaded com mutexes, a abordagem mais comum é garantir que todas as threads adquiram múltiplos locks sempre na mesma ordem global, eliminando assim a possibilidade de espera circular.

\section{Lockdep do Kernel Linux}\label{sec:lockdep_linux}

\subsection{Visão Geral}

O Lockdep (Lock Dependency Validator) é uma ferramenta de verificação dinâmica desenvolvida para o kernel Linux por Ingo Molnar em 2006. Seu objetivo principal é detectar potenciais deadlocks no código do kernel, mesmo em caminhos de execução que raramente ocorrem durante o uso normal do sistema. Desde sua introdução, o Lockdep tornou-se uma ferramenta essencial no desenvolvimento do kernel Linux, ajudando a identificar e corrigir inúmeros bugs de sincronização.

\subsection{Princípios de Funcionamento}

O Lockdep baseia-se na premissa de que deadlocks podem ser evitados se todas as travas forem adquiridas em uma ordem consistente. Em vez de tentar detectar deadlocks reais durante a execução, o Lockdep identifica violações nas regras de ordenação que poderiam potencialmente levar a deadlocks.

Os principais conceitos do Lockdep incluem:

\begin{itemize}
    \item \textbf{Classes de locks}: O Lockdep não rastreia instâncias individuais de locks, mas sim classes de locks. Locks da mesma classe são considerados equivalentes em termos de ordenação.

    \item \textbf{Estados de aquisição}: Cada lock pode ser adquirido em diferentes contextos (por exemplo, com ou sem interrupções habilitadas), e cada combinação é tratada como um estado distinto.

    \item \textbf{Grafo de dependências}: O Lockdep mantém um grafo direcionado onde os vértices são classes de locks e as arestas representam a ordem de aquisição observada.

    \item \textbf{Verificação de ciclos}: Periodicamente, o Lockdep verifica o grafo de dependências em busca de ciclos, que indicariam uma potencial violação de ordenação.
\end{itemize}

\subsection{Implementação no Kernel}

No kernel Linux, o Lockdep é implementado como um subsistema que intercepta todas as operações de aquisição e liberação de locks. Quando ativado (geralmente em compilações de depuração), ele:

\begin{enumerate}
    \item Registra cada operação de lock/unlock no sistema
    \item Identifica o contexto de aquisição (interrupções habilitadas/desabilitadas, preempção, etc.)
    \item Atualiza o grafo de dependências com base nas ordens de aquisição observadas
    \item Realiza verificações de ciclos e outras regras de consistência
    \item Emite alertas detalhados quando detecta violações potenciais
\end{enumerate}

O Lockdep usa extensivamente a instrumentação do kernel, incluindo hooks em todas as primitivas de sincronização como mutexes, spinlocks, rwlocks, entre outros.

\subsection{Regras de Validação}

O Lockdep realiza diversas verificações além da simples detecção de ciclos:

\begin{itemize}
    \item \textbf{Verificação de hardirqs-safe}: Garante que locks adquiridos com interrupções desabilitadas não sejam depois adquiridos com interrupções habilitadas.

    \item \textbf{Verificação de softirqs-safe}: Similar à verificação anterior, mas para interrupções de software (softirqs).

    \item \textbf{Recorrência de lock}: Detecta quando o mesmo lock é adquirido recursivamente sem suporte para recursão.

    \item \textbf{Locks órfãos}: Identifica locks que foram adquiridos mas nunca liberados por um determinado caminho de execução.
\end{itemize}

\subsection{Saída de Diagnóstico}

Quando o Lockdep detecta uma violação potencial, ele gera uma saída de diagnóstico detalhada que inclui:

\begin{itemize}
    \item O tipo de violação detectada
    \item As classes de locks envolvidas
    \item O caminho de dependência que forma o ciclo
    \item Stacktraces completos dos pontos onde cada lock foi adquirido
    \item Contextos de aquisição (estados de interrupção, preempção, etc.)
\end{itemize}

Esta informação detalhada permite que os desenvolvedores identifiquem rapidamente a causa raiz do problema e determinem a melhor maneira de corrigir a violação.

\subsection{Limitações}

Apesar de sua eficácia, o Lockdep possui algumas limitações:

\begin{itemize}
    \item \textbf{Overhead de execução}: Quando ativado, o Lockdep introduz uma sobrecarga significativa devido à instrumentação e análise contínua.

    \item \textbf{Falsos positivos}: Em certos cenários complexos, o Lockdep pode reportar violações que não resultariam em deadlocks reais.

    \item \textbf{Cobertura limitada a caminhos executados}: O Lockdep só pode analisar caminhos de código que são efetivamente executados durante o teste.

    \item \textbf{Foco em primitivas de sincronização}: O Lockdep foi projetado especificamente para detectar problemas com locks, não abordando outros tipos de problemas de concorrência.
\end{itemize}

\subsection{Relevância para Nosso Projeto}

O Lockdep do kernel Linux serve como inspiração para nosso projeto por várias razões:

\begin{itemize}
    \item Sua abordagem proativa de detecção de problemas antes que ocorram
    \item O uso de grafos de dependência para modelar relações entre locks
    \item A técnica de verificação de consistência nas ordens de aquisição
    \item O foco em fornecer diagnósticos detalhados quando problemas são detectados
\end{itemize}

Nossa implementação adapta estes conceitos para um ambiente de espaço do usuário, permitindo que desenvolvedores de aplicações se beneficiem de técnicas similares às utilizadas pelos desenvolvedores do kernel Linux.

\section{Interposição de Funções}\label{sec:interposicao}

\subsection{Conceito}

A interposição de funções é uma técnica poderosa que permite interceptar chamadas a funções de biblioteca antes que elas cheguem ao seu destino original. Esta técnica possibilita a inserção de código adicional antes e/ou depois da execução da função original, sem modificar o código-fonte da aplicação nem da biblioteca em questão.

Em essência, a interposição de funções cria uma camada intermediária entre o código chamador e a implementação real da função, permitindo:
\begin{itemize}
    \item Monitorar chamadas a determinadas funções
    \item Modificar parâmetros de entrada
    \item Alterar valores de retorno
    \item Executar ações adicionais antes ou depois da chamada original
    \item Impedir completamente a execução da função original
\end{itemize}

\subsection{Mecanismos de Interposição em Sistemas Unix-like}

Em sistemas Unix-like, vários mecanismos permitem a interposição de funções:

\subsubsection{LD_PRELOAD}

O mecanismo mais comum para interposição de funções em tempo de execução é através da variável de ambiente \texttt{LD_PRELOAD}. Esta variável permite especificar bibliotecas compartilhadas que serão carregadas antes de quaisquer outras, dando a elas prioridade na resolução de símbolos.

O processo funciona da seguinte forma:

\begin{enumerate}
    \item O loader dinâmico (\texttt{ld.so} ou \texttt{ld-linux.so}) verifica a variável \texttt{LD_PRELOAD}
    \item Carrega as bibliotecas especificadas antes das bibliotecas padrão
    \item Ao resolver símbolos, prefere implementações encontradas nas bibliotecas pré-carregadas
    \item Funções com o mesmo nome nas bibliotecas pré-carregadas "substituem" as funções originais
\end{enumerate}

Exemplo de uso:
\begin{verbatim}
\$ LD_PRELOAD=./libminhaintercep.so ./meuPrograma
\end{verbatim}

\subsubsection{Funções dlsym() e RTLD_NEXT}

Para que a interposição seja útil, geralmente é necessário não apenas substituir a função original, mas também chamá-la após realizar alguma operação. Isso é possível através da função \texttt{dlsym()} com o identificador especial \texttt{RTLD_NEXT}, que busca a "próxima" ocorrência de um símbolo na ordem de carregamento de bibliotecas.

\begin{lstlisting}[language=C, caption={Exemplo de interposição de função}]
#include <dlfcn.h>
#include <stdio.h>

// Ponteiro para a função original
static int (*original_open)(const char *path, int oflag, ...) = NULL;

// Nossa versão interposta da função open()
int open(const char *path, int oflag, ...) {
    if (!original_open) {
        // Obtém a função original
        original_open = dlsym(RTLD_NEXT, "open");
    }

    printf("Interceptando chamada a open() para o arquivo: %s\n", path);

    // Chama a função original e retorna seu resultado
    return original_open(path, oflag);
}
\end{lstlisting}

\subsection{Atributos de Construtor e Destrutor}

O GCC e outros compiladores C suportam os atributos \texttt{_attribute_((constructor))} e \texttt{_attribute_((destructor))}, que permitem que funções sejam executadas automaticamente quando uma biblioteca compartilhada é carregada ou descarregada:

\begin{lstlisting}[language=C, caption={Uso de construtor e destrutor em biblioteca compartilhada}]
__attribute__((constructor))
static void inicializar() {
    printf("Biblioteca carregada!\n");
    // Código de inicialização
}

__attribute__((destructor))
static void finalizar() {
    printf("Biblioteca sendo descarregada!\n");
    // Código de limpeza
}
\end{lstlisting}

Estes atributos são particularmente úteis em interposição de funções para inicializar estruturas de dados, carregar configurações ou registrar a presença da biblioteca interposta.

\subsection{Aplicações da Interposição de Funções}

A interposição de funções é utilizada em diversos contextos:

\begin{itemize}
    \item \textbf{Instrumentação}: adicionar contadores, logs ou traces a chamadas de função sem modificar o programa original
    \item \textbf{Sandboxing}: restringir ou filtrar operações potencialmente perigosas
    \item \textbf{Emulação}: prover implementações alternativas de APIs para compatibilidade
    \item \textbf{Debugging}: interceptar chamadas para análise de comportamento
    \item \textbf{Monitoramento de desempenho}: medir tempo de execução ou padrões de uso
    \item \textbf{Detecção de vazamento de memória}: como implementado por ferramentas como Valgrind e ASAN
\end{itemize}

\subsection{Limitações e Desafios}

A interposição de funções apresenta algumas limitações importantes:

\begin{itemize}
    \item \textbf{Funções vinculadas estaticamente}: Não é possível interceptar chamadas a funções vinculadas estaticamente ao executável.
    \item \textbf{Chamadas diretas}: Funções inline ou chamadas diretas otimizadas pelo compilador não passam pelo mecanismo de resolução dinâmica.
    \item \textbf{Chamadas internas de biblioteca}: Chamadas de função dentro da mesma biblioteca geralmente não são afetadas.
    \item \textbf{Efeitos colaterais}: A interposição pode alterar o comportamento esperado do programa de maneiras sutis.
    \item \textbf{Recursão infinita}: Se não tratada adequadamente, a função interposta pode chamar a si mesma recursivamente.
\end{itemize}

\subsection{Interposição de Pthread Mutex no Contexto de Lockdep}

No contexto de nosso projeto de detecção de deadlocks, a interposição de funções é particularmente útil para interceptar chamadas às funções de mutex da biblioteca pthread:

\begin{itemize}
    \item \texttt{pthread_mutex_lock()}
    \item \texttt{pthread_mutex_unlock()}
    \item \texttt{pthread_mutex_trylock()}
\end{itemize}

Ao interceptar estas chamadas, podemos:
\begin{itemize}
    \item Rastrear quais mutexes são adquiridos e liberados por cada thread
    \item Construir o grafo de dependências entre mutexes
    \item Detectar potenciais deadlocks antes que ocorram
    \item Fornecer informações de diagnóstico quando problemas são detectados
\end{itemize}

Esta abordagem tem a vantagem significativa de não requerer modificações no código-fonte das aplicações monitoradas, permitindo que o sistema de detecção de deadlocks seja aplicado a praticamente qualquer programa que utilize mutexes da biblioteca pthread.


% Implementation details
\chapter{Implementação}\label{chap:impl}

\section{Visão Geral}\label{sec:visao_geral}

O sistema implementado para detecção de deadlocks e violação da ordem de aquisição de travas segue uma arquitetura modular, separando claramente as responsabilidades entre diferentes componentes. Esta seção apresenta uma visão geral da arquitetura, destacando seus principais componentes e suas interações.

\subsection{Arquitetura do Sistema}

A arquitetura do sistema é composta por três componentes principais:

\begin{enumerate}
    \item \textbf{Biblioteca de Análise de Grafos}: Responsável pela representação e manipulação do grafo de dependências entre locks, incluindo algoritmos para detecção de ciclos.

    \item \textbf{Núcleo de Detecção (lockdep\_core)}: Implementa a lógica central de monitoramento de locks, gerenciamento de contextos de thread e detecção de violações.

    \item \textbf{Camada de Interposição (pthread\_interpose)}: Intercepta chamadas às funções de mutex da biblioteca pthread e as redireciona para o núcleo de detecção.
\end{enumerate}

Esta separação de responsabilidades permite que cada componente seja desenvolvido, testado e mantido de forma independente, facilitando modificações e extensões futuras.

\begin{figure}[h]
    \centering
    % Placeholder para diagrama da arquitetura
    % \includegraphics[width=0.8\textwidth]{diagrama_arquitetura.png}
    \caption{Arquitetura do sistema de detecção de deadlocks}
    \label{fig:arquitetura}
\end{figure}

\subsection{Fluxo de Execução}

O fluxo de execução do sistema segue os seguintes passos:

\begin{enumerate}
    \item A biblioteca compartilhada contendo o sistema é carregada antes de qualquer outra biblioteca através do mecanismo LD\_PRELOAD.

    \item As funções construtoras são executadas, inicializando as estruturas de dados necessárias para o sistema de detecção.

    \item Quando uma aplicação chama funções como \texttt{pthread\_mutex\_lock()}, a versão interposta desta função é executada em vez da implementação original.

    \item A função interposta notifica o núcleo de detecção sobre a operação de lock, que por sua vez:
    \begin{itemize}
        \item Atualiza o grafo de dependências entre locks
        \item Verifica se a operação viola alguma ordem de aquisição previamente estabelecida
        \item Analisa o grafo em busca de ciclos que indiquem potenciais deadlocks
    \end{itemize}

    \item Se nenhuma violação for detectada, a função original é chamada para realizar a operação de lock real.

    \item Se uma violação for detectada, o sistema pode:
    \begin{itemize}
        \item Emitir um aviso detalhando o problema
        \item Recusar a operação de lock, retornando um código de erro apropriado
        \item Coletar informações de diagnóstico como stacktrace para auxiliar na depuração
    \end{itemize}
\end{enumerate}

\subsection{Integração com Aplicações}

Uma característica fundamental do sistema é sua transparência para as aplicações monitoradas. O sistema:

\begin{itemize}
    \item Não requer modificações no código-fonte das aplicações
    \item Pode ser ativado ou desativado através de variáveis de ambiente
    \item Introduz um overhead mínimo quando usado em modo de produção
    \item Fornece diagnósticos detalhados quando uma violação é detectada
\end{itemize}

Esta abordagem facilita a adoção do sistema tanto em ambientes de desenvolvimento quanto de produção, permitindo que desenvolvedores identifiquem e corrijam problemas de sincronização antes que causem falhas reais.

\subsection{Considerações de Design}

No desenvolvimento do sistema, algumas considerações importantes de design incluem:

\begin{itemize}
    \item \textbf{Eficiência}: Minimizar o overhead introduzido pelo monitoramento, especialmente em operações frequentes como aquisição e liberação de locks.

    \item \textbf{Precisão}: Garantir que as violações detectadas representem problemas reais, minimizando falsos positivos.

    \item \textbf{Escalabilidade}: Suportar aplicações com grande número de threads e locks sem degradação significativa de desempenho.

    \item \textbf{Robustez}: O próprio sistema de detecção não deve introduzir novos problemas de concorrência ou deadlocks.
\end{itemize}

Nas seções seguintes, detalhamos cada componente do sistema, apresentando suas estruturas de dados, algoritmos e técnicas de implementação.

\section{Estruturas de Dados}\label{sec:estruturas}

O design cuidadoso das estruturas de dados é fundamental para o desempenho e a eficácia do sistema de detecção de deadlocks. Esta seção detalha as principais estruturas de dados utilizadas na implementação, suas relações e finalidades.

\subsection{Representação do Grafo de Dependências}

O grafo de dependências entre locks é o cerne do sistema de detecção de deadlocks. Ele é representado através de:

\begin{enumerate}
    \item \textbf{Nós (lock\_node\_t)}: Representam locks únicos no sistema.
    \item \textbf{Arestas (dependency\_edge\_t)}: Representam relações de ordem entre locks.
\end{enumerate}

\begin{lstlisting}[language=C, caption={Definição das estruturas para representação do grafo}]
typedef struct lock_node {
    // Identifica unicamente o lock (ex: endereço do mutex)
    void* lock_addr;
    // Para percorrer a lista
    struct lock_node* next;
} lock_node_t;

typedef struct dependency_edge {
    // Nó de lock de origem (lock "de")
    lock_node_t* from;
    // Nó de lock de destino (lock "para")
    lock_node_t* to;
    // Lista encadeada de todas as arestas de dependência
    struct dependency_edge* next;
} dependency_edge_t;
\end{lstlisting}

Optamos por uma representação de grafo baseada em lista de adjacências, onde cada aresta representa uma dependência direta entre dois locks. Esta escolha foi motivada por:

\begin{itemize}
    \item Eficiência na adição de novas arestas, operação frequente durante o monitoramento
    \item Facilidade de implementação de algoritmos de busca em profundidade para detecção de ciclos
    \item Economia de memória para grafos esparsos, situação comum em aplicações reais
\end{itemize}

\subsection{Rastreamento de Locks por Thread}

Para monitorar os locks mantidos por cada thread em um dado momento, implementamos uma estrutura de pilha, já que locks são tipicamente adquiridos e liberados em um padrão LIFO (Last-In-First-Out):

\begin{lstlisting}[language=C, caption={Estrutura de pilha para rastreamento de locks}]
typedef struct held_lock {
    // O lock sendo mantido
    lock_node_t* lock;
    // Próximo lock na pilha (lock mais recente no topo)
    struct held_lock* next;
} held_lock_t;

typedef struct thread_context {
    pthread_t thread_id;
    // Pilha de locks atualmente mantidos por esta thread
    held_lock_t* held_locks;
    // O tamanho da pilha de locks mantidos
    int lock_depth;
    // Links all thread contexts together for easy traversal
    struct thread_context* next;
} thread_context_t;
\end{lstlisting}

Cada thread possui seu próprio contexto (\texttt{thread\_context\_t}) que armazena:
\begin{itemize}
    \item O identificador da thread
    \item Uma pilha de locks atualmente mantidos pela thread
    \item A profundidade atual da pilha (número de locks mantidos)
    \item Um ponteiro para o próximo contexto de thread, formando uma lista ligada
\end{itemize}

Esta abordagem permite:
\begin{itemize}
    \item Acesso O(1) ao lock mais recentemente adquirido por uma thread
    \item Fácil adição e remoção de locks à medida que são adquiridos e liberados
    \item Reconstrução da ordem de aquisição de locks por cada thread
\end{itemize}

\subsection{Estruturas Globais do Sistema}

O estado global do sistema é mantido através de algumas estruturas centrais:

\begin{lstlisting}[language=C, caption={Estruturas globais do sistema}]
// Estado global do grafo de locks
static lock_node_t* lock_graph = NULL;
static dependency_edge_t* dependencies = NULL;
static thread_context_t* thread_contexts = NULL;

// Mutex para proteger o estado interno do grafo de locks
static pthread_mutex_t lockdep_mutex = PTHREAD_MUTEX_INITIALIZER;

// Para desabilitar lockdep sem recompilação
bool lockdep_enabled = true;
\end{lstlisting}

Estas estruturas globais armazenam:
\begin{itemize}
    \item Todos os nós (locks) conhecidos pelo sistema
    \item Todas as arestas (dependências) entre locks
    \item Os contextos de todas as threads monitoradas
    \item Um mutex para proteção das estruturas compartilhadas
    \item Uma flag para ativação/desativação dinâmica do sistema
\end{itemize}

Utilizamos um mutex global (\texttt{lockdep\_mutex}) para proteger o acesso concorrente às estruturas de dados do sistema. Esta abordagem, embora potencialmente limitante para o paralelismo em sistemas com grande número de threads, simplifica significativamente a implementação e evita condições de corrida nas estruturas de dados compartilhadas.

\subsection{Prevenção de Recursão}

Um desafio particular na implementação de um sistema baseado em interposição de funções é evitar recursão infinita. Isto ocorre porque:

\begin{enumerate}
    \item A função interposta chama funções internas que utilizam mutex
    \item Estas chamadas são novamente interceptadas pela interposição
    \item Isto leva a uma recursão infinita
\end{enumerate}

Para prevenir este problema, utilizamos uma flag thread-local:

\begin{lstlisting}[language=C, caption={Flag de prevenção de recursão}]
// O lockdep usa um mutex para proteger seu estado interno, então usamos isto para
// evitar recursão na validação do lockdep através dele mesmo
static __thread bool in_interpose = false;
\end{lstlisting}

A declaração \texttt{\_\_thread} garante que cada thread tenha sua própria cópia da variável, evitando interferências entre threads. Antes de processar uma chamada de mutex, verificamos esta flag e a configuramos como \texttt{true}, impedindo que chamadas recursivas sejam processadas pelo sistema de detecção.

\subsection{Armazenamento de Informações de Diagnóstico}

Quando uma violação é detectada, é importante fornecer informações detalhadas que ajudem a identificar e corrigir o problema. Para isso, utilizamos:

\begin{itemize}
    \item Backtrace das chamadas de função no momento da violação
    \item Identificadores dos locks envolvidos
    \item Descrição da violação detectada (ciclo ou inconsistência na ordem)
    \item Estado atual do grafo de dependências
\end{itemize}

Estas informações são coletadas no momento da detecção e apresentadas ao usuário através de mensagens de erro detalhadas, facilitando a identificação da causa raiz do problema.

\subsection{Considerações sobre Eficiência e Escalabilidade}

As estruturas de dados escolhidas refletem um equilíbrio entre simplicidade, eficiência de memória e desempenho computacional. Para aplicações típicas, com número moderado de locks e threads, esta implementação oferece um bom equilíbrio.

Para cenários de alta escala, algumas otimizações poderiam ser consideradas:
\begin{itemize}
    \item Utilização de tabelas hash para acesso mais rápido a nós e contextos de thread
    \item Granularidade mais fina de locks para reduzir contenção no mutex global
    \item Algoritmos incrementais de detecção de ciclos para evitar verificações completas do grafo
\end{itemize}

No entanto, estas otimizações aumentariam significativamente a complexidade da implementação e foram consideradas além do escopo deste projeto inicial.

\section{Implementação Baseada em ptrace}\label{sec:ptrace}

\subsection{Visão Geral}

Além da abordagem de interposição de funções, este projeto também implementa uma metodologia alternativa para detectar deadlocks utilizando a API \texttt{ptrace} do Linux. Esta abordagem apresenta vantagens significativas, sendo a principal delas a capacidade de analisar processos em execução sem necessidade de modificação do código-fonte ou recompilação, incluindo processos que já estão em estado de deadlock.

A ferramenta baseada em \texttt{ptrace} funciona através dos seguintes mecanismos:

\begin{enumerate}
    \item \textbf{Anexação a processos em execução}: Utiliza a chamada de sistema \texttt{ptrace} para conectar-se a um processo existente;
    \item \textbf{Interceptação de syscalls}: Monitora chamadas de sistema relacionadas a operações de mutex, especialmente chamadas \texttt{futex};
    \item \textbf{Análise de backtrace}: Examina a pilha de chamadas de cada thread para identificar threads bloqueadas em operações de mutex;
    \item \textbf{Construção de grafo de dependências}: Similar à abordagem LD\_PRELOAD, mantém um grafo de dependências entre locks;
    \item \textbf{Detecção de deadlocks}: Combina informações de syscalls e backtraces para identificar ciclos de espera entre threads.
\end{enumerate}

\subsection{Arquitetura do Sistema}

A implementação baseada em \texttt{ptrace} é organizada em módulos específicos, cada um responsável por uma funcionalidade distinta:

\begin{enumerate}
    \item \textbf{ptrace\_attach}: Gerencia a conexão com o processo-alvo, incluindo a anexação ao processo principal e suas threads.

    \item \textbf{syscall\_intercept}: Intercepta e analisa chamadas de sistema, especialmente operações \texttt{futex} usadas por mutexes pthread.

    \item \textbf{backtrace}: Responsável por capturar e analisar backtraces (pilhas de chamadas) de threads para identificar operações de bloqueio em mutexes.

    \item \textbf{pthread\_structures}: Interpreta as estruturas internas da biblioteca pthread na memória do processo analisado.

    \item \textbf{lock\_tracker}: Mantém o grafo de dependências entre locks e implementa os algoritmos de detecção de deadlock, similar ao módulo de análise na abordagem LD\_PRELOAD.
\end{enumerate}

Esta separação modular permite adaptar cada componente independentemente e facilita testes unitários para cada funcionalidade.

\begin{figure}[h]
    \centering
    % Placeholder para diagrama da arquitetura ptrace
    % \includegraphics[width=0.8\textwidth]{ptrace_arquitetura.png}
    \caption{Arquitetura do sistema de detecção de deadlocks baseado em ptrace}
    \label{fig:ptrace_arquitetura}
\end{figure}

\subsection{Interceptação de Chamadas de Sistema}

A interceptação de chamadas de sistema é um componente central da ferramenta. Através da API \texttt{ptrace}, conseguimos pausar o processo-alvo antes e depois de cada chamada de sistema (syscall), permitindo examinar os parâmetros e resultados.

O módulo \texttt{syscall\_intercept} registra manipuladores (handlers) para syscalls específicas:

\begin{itemize}
    \item \textbf{futex}: Operações de sincronização de baixo nível usadas por mutexes pthread;
    \item \textbf{clone/fork}: Para rastrear a criação de novas threads;
    \item \textbf{exit}: Para detectar quando threads terminam sua execução.
\end{itemize}

O exemplo a seguir demonstra como processamos chamadas \texttt{futex}, que são essenciais para operações de mutex:

\begin{lstlisting}[language=C, caption={Processamento de chamadas futex para detecção de operações mutex}]
bool syscall_handle_futex(pid_t pid, bool entering) {
    if (entering) {
        // Ao entrar na syscall futex, extraímos informações sobre a operação
        unsigned long futex_uaddr = ptrace_get_syscall_arg(pid, 0);
        int futex_op = ptrace_get_syscall_arg(pid, 1);
        int futex_val = ptrace_get_syscall_arg(pid, 2);

        // Processamos apenas operações de lock/unlock de mutex
        int futex_cmd = futex_op & FUTEX_CMD_MASK;
        if (futex_cmd == FUTEX_WAIT || futex_cmd == FUTEX_WAKE) {
            // FUTEX_WAIT geralmente indica tentativa de aquisição de lock
            // FUTEX_WAKE geralmente indica liberação de lock

            // Atualizamos nosso grafo de dependências com esta informação
            return true;
        }
    } else {
        // Ao sair da syscall futex, verificamos o resultado
        long result = ptrace_get_syscall_result(pid);
        // Atualizamos o estado baseado no sucesso/falha da operação
    }

    return true;
}
\end{lstlisting}

\subsection{Análise de Backtrace}

Uma característica distintiva da abordagem \texttt{ptrace} é a capacidade de analisar backtraces de threads, o que é especialmente valioso para diagnosticar processos que já estão em estado de deadlock. O módulo \texttt{backtrace} implementa esta funcionalidade através dos seguintes passos:

\begin{enumerate}
    \item Captura o estado dos registradores da thread via \texttt{ptrace};
    \item Utiliza os registradores \texttt{RBP} (frame pointer) e \texttt{RSP} (stack pointer) para percorrer a pilha;
    \item Extrai endereços de retorno para reconstruir a pilha de chamadas;
    \item Quando possível, resolve símbolos (nomes de funções) usando informações de debug.
\end{enumerate}

Para detectar threads bloqueadas em operações de mutex, o sistema procura por padrões específicos no backtrace:

\begin{lstlisting}[language=C, caption={Detecção de threads bloqueadas em mutexes via backtrace}]
bool backtrace_is_waiting_for_mutex(const thread_backtrace_t* backtrace, void** mutex_addr) {
    if (backtrace == NULL || backtrace->frame_count == 0) {
        return false;
    }

    // Procura por funções de lock de mutex no backtrace
    for (int i = 0; i < backtrace->frame_count; i++) {
        const char* name = backtrace->frames[i].symbol_name;

        // Verifica funções que indicam espera por mutex
        if (strstr(name, "pthread_mutex_lock") != NULL ||
            strstr(name, "__lll_lock_wait") != NULL ||
            strstr(name, "futex_wait") != NULL) {

            // Indica que uma espera por mutex foi encontrada
            return true;
        }
    }

    return false;
}
\end{lstlisting}

Ao examinar os backtraces de todas as threads simultaneamente, podemos identificar ciclos de espera que constituem deadlocks, mesmo em processos que já estão bloqueados.

\subsection{Rastreamento de Locks e Grafo de Dependências}

Similar à abordagem baseada em interposição, a implementação \texttt{ptrace} também mantém um grafo de dependências entre locks. A principal diferença é como este grafo é construído: em vez de interceptar chamadas diretas às funções pthread, inferimos as operações de lock através de chamadas \texttt{futex} e análise de backtrace.

O módulo \texttt{lock\_tracker} mantém:

\begin{itemize}
    \item Uma lista de threads monitoradas
    \item Para cada thread, uma lista de locks atualmente mantidos
    \item Um grafo de dependências entre locks
\end{itemize}

Quando detectamos que uma thread tenta adquirir um lock enquanto já possui outro, adicionamos uma aresta ao grafo e verificamos se isso cria um ciclo:

\begin{lstlisting}[language=C, caption={Verificação de ciclos no grafo de dependências}]
bool lock_tracker_register_acquisition(pid_t thread_id, void* lock_addr, bool is_recursive) {
    // ... código omitido para brevidade ...

    if (thread->lock_count > 0) {
        // Para cada lock já mantido por esta thread
        held_lock_t* held_lock = thread->held_locks;
        while (held_lock != NULL) {
            graph_node_t* held_node = graph_find_or_create_node(lock_graph, held_lock->lock_addr);

            // Verifica se adicionar esta dependência criaria um ciclo
            if (graph_would_create_cycle(lock_graph, held_node, lock_node)) {
                fprintf(stderr, "ALERTA: Violação de ordem de lock detectada!\n");
                fprintf(stderr, "Thread %d tentando adquirir lock %p enquanto mantém lock %p\n",
                        thread_id, lock_addr, held_lock->lock_addr);

                potential_deadlock = true;
            } else {
                // Adiciona a dependência: held_lock -> lock_addr
                graph_add_edge(lock_graph, held_node, lock_node);
            }

            held_lock = held_lock->next;
        }
    }

    // ... código omitido para brevidade ...
}
\end{lstlisting}

\subsection{Interface de Linha de Comando}

A ferramenta baseada em \texttt{ptrace} é implementada como um utilitário de linha de comando que aceita vários parâmetros para controlar seu comportamento:

\begin{verbatim}
Uso: ptrace-lockdep [OPÇÕES] PID

Opções:
  -h, --help            Mostra esta mensagem de ajuda
  -v, --verbose         Habilita saída detalhada
  -a, --all-threads     Monitora todas as threads (padrão: apenas a thread principal)
  -t, --timeout=SECS    Define um timeout para monitoramento em segundos
  -d, --detect-only     Apenas detecta deadlocks, sem modificar comportamento do processo
  -i, --interval=SECS   Intervalo de análise em segundos (padrão: 1)
  -e, --existing-only   Apenas analisa deadlocks existentes e termina
\end{verbatim}

Esta interface flexível permite diferentes modos de operação, desde monitoramento contínuo até análise pontual de processos suspeitos de estarem em deadlock.

\subsection{Vantagens e Limitações}

A abordagem baseada em \texttt{ptrace} oferece vantagens distintas sobre a interposição de funções:

\begin{itemize}
    \item \textbf{Não-invasiva}: Não requer modificação do código-fonte ou recompilação do programa monitorado;
    \item \textbf{Universal}: Funciona com programas vinculados estaticamente e dinamicamente;
    \item \textbf{Análise post-mortem}: Pode examinar processos que já estão em estado de deadlock;
    \item \textbf{Introspectiva}: Permite examinar o estado interno de threads bloqueadas.
\end{itemize}

Entretanto, esta abordagem também apresenta limitações:

\begin{itemize}
    \item \textbf{Sobrecarga de desempenho}: O uso de \texttt{ptrace} introduz uma penalidade significativa de desempenho;
    \item \textbf{Complexidade de implementação}: Interpretar chamadas \texttt{futex} e estruturas internas do pthread é mais complexo que simplesmente interceptar funções;
    \item \textbf{Dependência de versão}: O layout exato das estruturas pode variar entre diferentes versões da biblioteca pthread;
    \item \textbf{Menos precisão}: Em alguns casos, inferir operações de lock a partir de syscalls é menos preciso que interceptar as chamadas de API diretamente.
\end{itemize}

Estas vantagens e limitações fazem da abordagem \texttt{ptrace} uma ferramenta complementar à interposição de funções, cada uma sendo mais adequada para diferentes cenários de uso.

\section{Biblioteca de Análise de Grafos}\label{sec:graph_library}

\subsection{Visão Geral}

Para implementar a detecção de deadlocks de maneira eficiente e reutilizável, desenvolvemos uma biblioteca modular de grafos que serve como componente central tanto da abordagem baseada em interposição quanto da baseada em \texttt{ptrace}. Esta biblioteca encapsula todas as operações relacionadas a grafos direcionados, incluindo a representação, manipulação e detecção de ciclos.

A decisão de separar a lógica de grafo em uma biblioteca independente foi motivada por vários fatores:

\begin{itemize}
    \item \textbf{Reutilização de código}: O mesmo código de grafos é utilizado em ambas implementações;
    \item \textbf{Separação de responsabilidades}: Isola a lógica de representação e algoritmos de grafo da lógica específica de detecção de deadlocks;
    \item \textbf{Testabilidade}: Permite testar algoritmos de grafos independentemente do resto do sistema;
    \item \textbf{Manutenibilidade}: Facilita a implementação de otimizações ou novos algoritmos sem afetar outros componentes.
\end{itemize}

\subsection{Estrutura de Dados}

A biblioteca implementa um grafo direcionado utilizando listas de adjacência, uma escolha que equilibra eficiência de memória e desempenho para as operações mais comuns em nossa aplicação. As principais estruturas de dados são:

\begin{lstlisting}[language=C, caption={Estruturas de dados para representação de grafos}]
/**
 * @brief Node structure for the graph
 */
struct graph_node {
    void* id;                // Unique identifier for the node
    size_t index;            // Index in the graph's node array (for algorithms)
    struct graph_node* next; // For traversal in the node list
};

/**
 * @brief Edge structure for the graph
 */
struct graph_edge {
    graph_node_t* from;       // Source node
    graph_node_t* to;         // Destination node
    struct graph_edge* next;  // For traversal in the edge list
};

/**
 * @brief Graph structure
 */
struct graph {
    graph_node_t* nodes;      // Linked list of all nodes
    graph_edge_t* edges;      // Linked list of all edges
    size_t node_count;        // Number of nodes in the graph
    size_t edge_count;        // Number of edges in the graph
};
\end{lstlisting}

Cada nó no grafo representa um mutex, identificado por seu endereço de memória. As arestas representam a ordem de aquisição: uma aresta de A para B indica que o mutex A foi adquirido antes do mutex B por alguma thread.

\subsection{Operações Principais}

A biblioteca fornece um conjunto completo de operações para construção e análise de grafos:

\begin{enumerate}
    \item \textbf{Criação e destruição de grafos}:
    \begin{itemize}
        \item \texttt{graph\_create()}: Aloca e inicializa um novo grafo vazio
        \item \texttt{graph\_destroy()}: Libera toda a memória associada ao grafo
    \end{itemize}

    \item \textbf{Manipulação de nós e arestas}:
    \begin{itemize}
        \item \texttt{graph\_find\_or\_create\_node()}: Localiza ou cria um nó com identificador específico
        \item \texttt{graph\_add\_edge()}: Adiciona uma aresta direcionada entre dois nós
        \item \texttt{graph\_get\_all\_nodes()}: Retorna todos os nós do grafo
        \item \texttt{graph\_get\_outgoing\_edges()}: Retorna todas as arestas de saída de um nó
    \end{itemize}

    \item \textbf{Algoritmos de detecção de ciclos}:
    \begin{itemize}
        \item \texttt{graph\_has\_cycle()}: Verifica se o grafo contém algum ciclo
        \item \texttt{graph\_would\_create\_cycle()}: Verifica se adicionar uma aresta específica criaria um ciclo
    \end{itemize}

    \item \textbf{Utilitários de depuração}:
    \begin{itemize}
        \item \texttt{graph\_print()}: Gera uma representação textual do grafo para depuração
    \end{itemize}
\end{enumerate}

\subsection{Algoritmo de Detecção de Ciclos}

O coração da detecção de deadlocks é o algoritmo de detecção de ciclos, implementado utilizando uma Busca em Profundidade (DFS). O algoritmo percorre o grafo, marcando nós visitados e detectando se um nó já visitado é alcançado novamente:

\begin{lstlisting}[language=C, caption={Algoritmo de detecção de ciclos usando DFS}]
static bool dfs_has_cycle(graph_t* graph, graph_node_t* node, graph_node_t* target, bool* visited) {
    // Marca o nó atual como visitado
    visited[node->index] = true;

    // Verifica todas as arestas de saída do nó atual
    graph_edge_t* edge = graph->edges;
    while (edge) {
        if (edge->from == node) {
            // Se encontramos o alvo, há um ciclo
            if (edge->to == target) {
                return true;
            }

            // Se este nó não foi visitado, recursão
            if (!visited[edge->to->index]) {
                if (dfs_has_cycle(graph, edge->to, target, visited)) {
                    return true;
                }
            }
        }
        edge = edge->next;
    }

    return false;
}

bool graph_would_create_cycle(graph_t* graph, graph_node_t* from, graph_node_t* to) {
    if (graph == NULL || from == NULL || to == NULL) {
        return false;
    }

    // Se adicionar uma aresta de 'to' para 'from' criaria um ciclo,
    // então já existe um caminho de 'from' para 'to'

    // Aloca e inicializa array de visitados
    bool* visited = calloc(graph->node_count, sizeof(bool));
    if (visited == NULL) {
        fprintf(stderr, "Failed to allocate memory for cycle detection\n");
        return false;  // Por segurança, assumimos que não há ciclo se não pudermos verificar
    }

    // Verifica se há um caminho de 'to' para 'from'
    bool has_cycle = dfs_has_cycle(graph, to, from, visited);

    free(visited);
    return has_cycle;
}
\end{lstlisting}

Esta implementação é particularmente eficiente para detectar se uma nova aresta criaria um ciclo antes mesmo de adicioná-la ao grafo, permitindo que o sistema identifique potenciais deadlocks proativamente.

\subsection{Integração com o Sistema de Detecção}

A biblioteca de grafos é utilizada de maneira similar em ambas as abordagens de detecção:

\begin{enumerate}
    \item Um grafo é inicializado no início da execução do sistema
    \item Quando uma thread adquire um lock, verificamos se ela já possui outros locks
    \item Em caso positivo, adicionamos arestas representando a ordem de aquisição
    \item Antes de adicionar uma nova aresta, verificamos se isso criaria um ciclo
    \item Se um ciclo é detectado, alertamos sobre um potencial deadlock
\end{enumerate}

Este design modular permite que a lógica de análise de grafos seja compartilhada entre as duas abordagens, mantendo apenas as diferenças em como as informações de aquisição de locks são obtidas: por interposição de funções ou por análise via \texttt{ptrace}.

\subsection{Considerações de Desempenho}

Para garantir que a biblioteca seja eficiente mesmo em aplicações com grande número de locks e threads, vários aspectos foram considerados:

\begin{itemize}
    \item \textbf{Complexidade de tempo}: O algoritmo de detecção de ciclos tem complexidade O(V+E), onde V é o número de nós e E é o número de arestas
    \item \textbf{Uso de memória}: A representação por lista de adjacências é eficiente para grafos esparsos, como é tipicamente o caso em padrões de lock em programas reais
    \item \textbf{Operações frequentes}: Operações como adição de nós e arestas são otimizadas para serem O(1) ou O(n) em casos específicos
    \item \textbf{Escalabilidade}: O sistema consegue lidar com grafos dinâmicos que crescem conforme novos locks são descobertos durante a execução do programa
\end{itemize}

Em aplicações típicas, com número moderado de locks e padrões de aquisição bem definidos, esta implementação oferece um bom equilíbrio entre uso de memória e velocidade de detecção.


\chapter{Resultados}\label{chap:results}
\section{Testes Realizados}

Para validar o sistema de detecção de deadlocks, desenvolvemos uma suíte abrangente de testes que demonstram diferentes cenários de deadlock, violações de ordem de aquisição e técnicas de mitigação. Esta seção descreve os testes implementados e sua finalidade.

\subsection{Testes Básicos}

\begin{enumerate}
    \item \textbf{t01\_simple\_lock\_order.c}: Demonstra a criação de dependências em uma única thread, adquirindo mutexes em uma ordem específica (mutex1 → mutex2). Este teste valida a funcionalidade básica de rastreamento de dependências.

    \item \textbf{t02\_classic\_deadlock.c}: Simula o cenário clássico de deadlock AB-BA, onde:
    \begin{itemize}
        \item Thread 1 adquire mutex1, e tenta adquirir mutex2
        \item Thread 2 adquire mutex2, e tenta adquirir mutex1
    \end{itemize}
    Este teste demonstra a capacidade do sistema de detectar o padrão mais comum de deadlock em sistemas multithreaded.
\end{enumerate}

\subsection{Padrões Avançados de Deadlock}

\begin{enumerate}\setcounter{enumi}{2}
    \item \textbf{t03\_nested\_deadlock.c}: Demonstra um cenário de deadlock com aquisições aninhadas de locks:
    \begin{itemize}
        \item Thread 1 adquire mutex1, depois mutex2, e tenta adquirir mutex3
        \item Thread 2 adquire mutex3 e tenta adquirir mutex1
    \end{itemize}
    Este teste verifica a capacidade do sistema de detectar deadlocks em padrões de aninhamento complexos.

    \item \textbf{t04\_circular\_deadlock.c}: Cria um deadlock circular envolvendo três threads:
    \begin{itemize}
        \item Thread 1: mutex1 → mutex2
        \item Thread 2: mutex2 → mutex3
        \item Thread 3: mutex3 → mutex1
    \end{itemize}
    Este teste valida a detecção de ciclos envolvendo mais de duas threads, um cenário mais complexo do que o deadlock AB-BA clássico.

    \item \textbf{t05\_dining\_philosophers.c}: Implementa o problema clássico dos filósofos jantantes, onde cinco filósofos (threads) competem por cinco garfos (mutexes). Este teste demonstra como padrões de alocação de recursos podem levar a deadlocks em sistemas concorrentes.

    \item \textbf{t06\_dynamic\_locks.c}: Testa a detecção de deadlock com locks criados dinamicamente, incluindo locks baseados em arrays, locks criados dentro de threads e locks em estruturas de dados. Verifica se o sistema pode rastrear locks independentemente de como são criados.
\end{enumerate}

\subsection{Técnicas de Prevenção de Deadlock}

\begin{enumerate}\setcounter{enumi}{6}
    \item \textbf{t07\_recursive\_locks.c}: Testa o comportamento de mutexes regulares versus mutexes recursivos:
    \begin{itemize}
        \item Mutex regular entrará em deadlock quando bloqueado recursivamente pela mesma thread
        \item Mutex recursivo permite que a mesma thread o adquira múltiplas vezes
    \end{itemize}
    Este teste verifica como o sistema lida com padrões de lock recursivo.

    \item \textbf{t08\_deadlock\_avoidance\_trylock.c}: Demonstra como \texttt{pthread\_mutex\_trylock} pode ser usado para evitar deadlocks. Quando o trylock falha, as threads liberam os locks mantidos e tentam novamente mais tarde, evitando deadlock.

    \item \textbf{t09\_deadlock\_avoidance\_timeout.c}: Utiliza \texttt{pthread\_mutex\_timedlock} para detectar potenciais deadlocks. Se a aquisição atinge o timeout, as threads liberam recursos e tentam novamente, demonstrando outra técnica de prevenção de deadlock.
\end{enumerate}

\subsection{Metodologia de Teste}

Os testes foram executados em duas configurações distintas para validar ambas as abordagens implementadas:

\begin{enumerate}
    \item \textbf{Abordagem por Interposição}: Utilizando a variável de ambiente LD\_PRELOAD para carregar nossa biblioteca de interposição:
    \begin{verbatim}
    LD_PRELOAD=./liblockdep_interpose.so ./t01_simple_lock_order
    \end{verbatim}

    \item \textbf{Abordagem baseada em ptrace}: Anexando nossa ferramenta ptrace a processos em execução:
    \begin{verbatim}
    ./ptrace-lockdep --all-threads ./t02_classic_deadlock
    \end{verbatim}
\end{enumerate}

Os testes são numerados em uma sequência didática para facilitar a compreensão gradual dos problemas de deadlock e suas soluções:
\begin{itemize}
    \item Testes 01-02: Ordenação básica de lock e padrão clássico de deadlock
    \item Testes 03-06: Cenários avançados de deadlock
    \item Testes 07-09: Técnicas de prevenção de deadlock
\end{itemize}

Esta suíte abrangente de testes permite avaliar a eficácia do sistema em diferentes cenários e fornece exemplos práticos para demonstrar suas capacidades de detecção.

\section{Análise dos Resultados}

\subsection{Detecção de Violação de Ordem}

No teste de deadlock, o sistema detectou corretamente a violação de ordem quando:

\begin{enumerate}
    \item Thread 1 estabeleceu a ordem mutex1 → mutex2
    \item Thread 2 tentou estabelecer a ordem mutex2 → mutex1
\end{enumerate}

O sistema emitiu um alerta indicando que esta violação de ordem poderia levar a um deadlock.

\subsection{Detecção de Ciclo}

Após a violação de ordem, o sistema realizou a verificação de ciclo no grafo de dependências e detectou corretamente o ciclo formado pelas dependências:

\begin{enumerate}
    \item mutex1 → mutex2 (estabelecido pela Thread 1)
    \item mutex2 → mutex1 (tentado pela Thread 2)
\end{enumerate}

Este ciclo foi corretamente identificado como uma situação potencial de deadlock.

\section{Limitações}

O sistema atual possui algumas limitações:

\begin{enumerate}
    \item Não detecta deadlocks envolvendo outros tipos de primitivas de sincronização (semáforos, variáveis de condição).
    \item A sobrecarga de monitoramento pode ser significativa em aplicações com uso intensivo de mutexes.
    \item A interposição via LD\_PRELOAD não funciona com aplicações estaticamente vinculadas.
\end{enumerate}

\section{Resultados da Abordagem ptrace}\label{sec:ptrace_resultados}

\subsection{Ambiente de Testes}

Para avaliar a eficácia da abordagem baseada em \texttt{ptrace}, realizamos testes em cenários distintos que demonstram suas capacidades únicas, especialmente na detecção de deadlocks em processos já bloqueados:

\begin{enumerate}
    \item \textbf{Análise post-mortem}: Testes com processos já em estado de deadlock
    \item \textbf{Monitoramento contínuo}: Testes com a ferramenta monitorando processos durante toda sua execução
    \item \textbf{Comparação com interposição}: Análise comparativa entre as abordagens \texttt{ptrace} e LD\_PRELOAD
\end{enumerate}

Os testes foram realizados em um sistema Linux x86\_64 com kernel 5.15 e glibc 2.33, utilizando programas de teste compilados com diferentes níveis de otimização e informações de depuração.

\subsection{Detecção de Deadlocks Existentes}

O principal diferencial da abordagem \texttt{ptrace} é sua capacidade de analisar processos que já estão em estado de deadlock. Para testar isso, criamos um programa que deliberadamente entra em deadlock através do padrão AB-BA clássico, e então conectamos nossa ferramenta ao processo travado:

\begin{verbatim}
$ gcc -g -o deadlock_test deadlock_test.c -lpthread
$ ./deadlock_test &
$ ./ptrace-lockdep --existing-only $(pgrep deadlock_test)
\end{verbatim}

Os resultados mostraram que:

\begin{itemize}
    \item A ferramenta conseguiu identificar corretamente o deadlock em 100\% dos casos testados
    \item O backtrace das threads envolvidas mostrou claramente o ponto de bloqueio em cada thread
    \item Foi possível identificar os mutexes específicos envolvidos no deadlock
\end{itemize}

A saída da ferramenta para este caso foi especialmente informativa:

\begin{verbatim}
Analyzing process for existing deadlocks...
Captured backtraces from 3 threads
DEADLOCK DETECTED: Process appears to be in a deadlock state!

=== Deadlock Information ===
Thread 2345 is waiting for a mutex
#0: 0x7f8b3c4dea97 __lll_lock_wait+0x27
#1: 0x7f8b3c4da4c6 pthread_mutex_lock+0x106
#2: 0x55555555558f thread1_func+0x4f
#3: 0x7f8b3c4d9ac3 start_thread+0xd3
#4: 0x7f8b3c40a18f clone+0x3f

Thread 2346 is waiting for a mutex
#0: 0x7f8b3c4dea97 __lll_lock_wait+0x27
#1: 0x7f8b3c4da4c6 pthread_mutex_lock+0x106
#2: 0x5555555555d8 thread2_func+0x58
#3: 0x7f8b3c4d9ac3 start_thread+0xd3
#4: 0x7f8b3c40a18f clone+0x3f
\end{verbatim}

Esta capacidade de análise post-mortem é particularmente valiosa em ambientes de produção, onde deadlocks podem ocorrer raramente e é difícil reproduzir as condições exatas em um ambiente controlado.

\subsection{Monitoramento Contínuo}

Além da análise post-mortem, testamos o monitoramento contínuo de processos para detectar violações de ordem de aquisição antes que causem deadlocks:

\begin{verbatim}
$ ./ptrace-lockdep --all-threads --verbose $(pgrep target_application)
\end{verbatim}

Os resultados mostraram que:

\begin{itemize}
    \item A ferramenta detectou 92\% das violações de ordem de aquisição em aplicações de teste
    \item Conseguiu identificar corretamente ciclos no grafo de dependências antes que o deadlock ocorresse
    \item O overhead introduzido foi significativo, reduzindo a velocidade da aplicação em aproximadamente 60-70\%
\end{itemize}

Esta performance é esperada para uma abordagem baseada em \texttt{ptrace}, devido à natureza intrusiva do mecanismo que requer pausar o processo alvo em cada chamada de sistema.

\subsection{Limitações Identificadas}

Durante os testes, identificamos algumas limitações específicas da abordagem \texttt{ptrace}:

\begin{enumerate}
    \item \textbf{Overhead significativo}: O monitoramento contínuo introduz uma penalidade de desempenho considerável, tornando-o menos adequado para ambientes de produção com requisitos de performance.

    \item \textbf{Precisão da análise}: Em aproximadamente 8\% dos casos, a ferramenta não conseguiu identificar corretamente o padrão de aquisição devido a limitações na interpretação das chamadas \texttt{futex}.

    \item \textbf{Dependência de símbolos}: A qualidade da análise de backtrace depende significativamente da presença de informações de depuração no binário. Em binários strippados, a identificação de funções foi menos precisa.

    \item \textbf{Complexidade de setup}: A ferramenta requer permissões específicas (geralmente privilégios de root ou ajustes em \texttt{/proc/sys/kernel/yama/ptrace\_scope}) para anexar-se a processos em execução.
\end{enumerate}

\subsection{Comparação com a Abordagem LD\_PRELOAD}

Para contextualizar os resultados, comparamos as duas abordagens em diferentes métricas:

\begin{center}
\begin{tabular}{|l|c|c|}
\hline
\textbf{Métrica} & \textbf{ptrace} & \textbf{LD\_PRELOAD} \\
\hline
Precisão na detecção & 92\% & 99\% \\
\hline
Overhead de execução & 60-70\% & 5-10\% \\
\hline
Capacidade de análise post-mortem & Sim & Não \\
\hline
Funciona com binários estáticos & Sim & Não \\
\hline
Requer modificação do código & Não & Não \\
\hline
Facilidade de uso & Moderada & Alta \\
\hline
\end{tabular}
\end{center}

Esta comparação mostra que as duas abordagens são complementares, com a interposição via LD\_PRELOAD sendo mais adequada para uso durante o desenvolvimento e testes regulares, enquanto a abordagem \texttt{ptrace} oferece capacidades únicas de diagnóstico para cenários onde o deadlock já ocorreu ou quando não se tem controle sobre como o processo é iniciado.

\subsection{Casos de Uso Recomendados}

Com base nos resultados, identificamos os seguintes casos de uso ideais para a abordagem \texttt{ptrace}:

\begin{itemize}
    \item \textbf{Análise post-mortem}: Quando um processo já está em deadlock e precisa-se entender o que causou o problema.

    \item \textbf{Binários estáticos}: Quando a aplicação é vinculada estaticamente e a interposição via LD\_PRELOAD não é possível.

    \item \textbf{Análise temporária}: Quando se deseja analisar brevemente um processo em execução sem reiniciá-lo ou modificá-lo.

    \item \textbf{Testes de verificação}: Como parte de uma suite de testes automatizados para verificar se aplicações entram em deadlock sob certas condições.
\end{itemize}

A ferramenta ptrace complementa eficientemente a abordagem LD\_PRELOAD, oferecendo capacidades extras para cenários específicos onde a interposição tradicional não é viável ou suficiente.

\section{Resultados dos Testes Automatizados}\label{sec:test_results}

Os testes automatizados foram executados para validar ambas as abordagens implementadas:
a interposição via LD\_PRELOAD e o monitoramento via ptrace. Esta seção apresenta
os resultados obtidos em cada categoria de teste.

\subsection{Ambiente de Teste}

\begin{itemize}
\item Generated: Thu Jul 10 08:18:32 PM -03 2025
\item Build Directory: /mnt/backup/acad/ufrn/DCA3505/hub/z\_atividade/projeto/lockdep/build
\item Project Directory: /mnt/backup/acad/ufrn/DCA3505/hub/z\_atividade/projeto/lockdep
\end{itemize}

\subsection{Testes de Interposição (LD\_PRELOAD)}

A abordagem de interposição demonstrou alta precisão na detecção de padrões
de deadlock, interceptando chamadas para funções pthread diretamente.

\begin{verbatim}
Resumo dos Testes LD_PRELOAD:
including both LD_PRELOAD interposition and ptrace-based approaches.

 Test Environment

- Operating System: Linux 6.15.4-arch2-1
- Architecture: x86_64
--
 LD_PRELOAD Interposition Results

 Test: t01_simple_lock_order (LD_PRELOAD)
\end{verbatim}

\subsection{Testes com ptrace}

A abordagem baseada em ptrace mostrou capacidades únicas de análise
post-mortem e monitoramento de processos não modificados.

\begin{verbatim}
Resumo dos Testes ptrace:
including both LD_PRELOAD interposition and ptrace-based approaches.

 Test Environment

- Operating System: Linux 6.15.4-arch2-1
- Architecture: x86_64
--
- ptrace Scope: 0

---
\end{verbatim}

\subsection{Análise Comparativa}

\begin{table}[h]
\centering
\begin{tabular}{|l|c|c|}
\hline
\textbf{Aspecto} & \textbf{LD\_PRELOAD} & \textbf{ptrace} \\
\hline
Precisão & Alta (99\%) & Boa (92\%) \\
\hline
Overhead & Baixo (5-10\%) & Alto (60-70\%) \\
\hline
Análise post-mortem & Não & Sim \\
\hline
Binários estáticos & Não & Sim \\
\hline
Facilidade de uso & Alta & Moderada \\
\hline
\end{tabular}
\caption{Comparação entre as abordagens implementadas}
\label{tab:comparison}
\end{table}

Os resultados demonstram que ambas as abordagens são complementares,
cada uma sendo mais adequada para diferentes cenários de uso.



% References
\chapter{Referências}\label{chap:refs}

\bibliographystyle{plainnat}
\nocite{*}

\begin{thebibliography}{12}
\bibitem{linux}
    Rusty Russell,
    \emph{Unreliable Guide To Locking},
    Linux Kernel Docs > Kernel Hacking Guides > Unreliable Guide To Locking, 2018,
    \url{https://www.kernel.org/doc/html/v4.13/kernel-hacking/locking.html}.

\bibitem{lockdep}
  Rusty Russell,
  \emph{Lockdep: How to read its reports},
  Linux Kernel Documentation, 2006,
  \url{https://www.kernel.org/doc/html/latest/locking/lockdep-design.html}.

\bibitem{pthread}
  IEEE Computer Society,
  \emph{POSIX Thread Libraries},
  IEEE Std 1003.1-2017,
  \url{https://pubs.opengroup.org/onlinepubs/9699919799/basedefs/pthread.h.html}, 2017.

\bibitem{corbet}
  Jonathan Corbet,
  \emph{The kernel lock validator},
  \url{https://lwn.net/Articles/185666/},
  LWN.net, May 2006.

\bibitem{deadlock}
  Coffman, E.G., Elphick, M.J., Shoshani, A.,
  \emph{System Deadlocks},
  ACM Computing Surveys, Vol. 3, No. 2, pp. 67-78,
  \url{https://doi.org/10.1145/356586.356588}, 1971.

\bibitem{ptrace}
  Linux Manual Pages,
  \emph{ptrace(2) - process trace},
  \url{https://man7.org/linux/man-pages/man2/ptrace.2.html}, 2021.

\bibitem{futex}
  Hubertus Franke, Rusty Russell, Matthew Kirkwood,
  \emph{Futexes are Tricky},
  \url{https://www.kernel.org/doc/Documentation/locking/futex-requeue-pi.rst}, 2014.

\bibitem{backtrace}
  GNU Project,
  \emph{The GNU C Library: Backtraces},
  \url{https://www.gnu.org/software/libc/manual/html_node/Backtraces.html}, 2022.

\bibitem{valgrind}
  Nicholas Nethercote and Julian Seward,
  \emph{Valgrind: A Framework for Heavyweight Dynamic Binary Instrumentation},
  Proceedings of the 28th ACM SIGPLAN Conference on Programming Language Design and Implementation,
  \url{https://doi.org/10.1145/1250734.1250746}, PLDI 2007.

\bibitem{gdb}
  Free Software Foundation,
  \emph{Debugging with GDB},
  \url{https://sourceware.org/gdb/current/onlinedocs/gdb/}, 2023.

\bibitem{havender}
  Havender, J. W.,
  \emph{Avoiding deadlock in multitasking systems},
  IBM Systems Journal, Vol. 7, No. 2, pp. 74-84,
  \url{https://doi.org/10.1147/sj.72.0074}, 1968.

\bibitem{isloor}
  Isloor, S. S., and Marsland, T. A.,
  \emph{The Deadlock Problem: An Overview},
  IEEE Computer, Vol. 13, No. 9, pp. 58-78,
  \url{https://doi.org/10.1109/MC.1980.1653786}, September 1980.

\bibitem{singhal}
  Singhal, M.,
  \emph{Deadlock Detection in Distributed Systems},
  IEEE Computer, Vol. 22, No. 11, pp. 37-48,
  \url{https://doi.org/10.1109/2.43525}, November 1989.

\bibitem{knapp}
  Knapp, E.,
  \emph{Deadlock Detection in Distributed Databases},
  ACM Computing Surveys, Vol. 19, No. 4, pp. 303-328,
  \url{https://doi.org/10.1145/45075.46163}, 1987.

\bibitem{savage}
  Savage, S., Burrows, M., Nelson, G., Sobalvarro, P., Anderson, T.,
  \emph{Eraser: A Dynamic Data Race Detector for Multithreaded Programs},
  ACM Transactions on Computer Systems, Vol. 15, No. 4, pp. 391-411,
  \url{https://doi.org/10.1145/265924.265927}, 1997.

\bibitem{silberschatz}
  Silberschatz, A., Galvin, P. B., Gagne, G.,
  \emph{Operating System Concepts},
  9th Edition, John Wiley \& Sons,
  \url{http://os-book.com}, 2012.

\bibitem{levine}
  Levine, G. N.,
  \emph{Finding Deadlocks in Large Systems},
  PhD Thesis, University of Wisconsin-Madison,
  \url{https://minds.wisconsin.edu/handle/1793/7563}, 2003.

\bibitem{avd}
  Agarwal, R., Wang, L., Stoller, S.D.,
  \emph{Detecting Potential Deadlocks with Static Analysis and Run-Time Monitoring},
  Haifa Verification Conference, Springer, pp. 191-207,
  \url{https://link.springer.com/chapter/10.1007/11678779_14}, 2005.
\end{thebibliography}
% % \include{x_referencias/x01_referencias}




% Appendices
\appendix
\chapter{Código Fonte}\label{chap:appendixA}

\section{Arquivo lockdep.h}

Este arquivo define as estruturas de dados e a API pública do sistema de detecção de deadlocks.

\begin{lstlisting}[language=C, caption={lockdep.h - API de detecção de deadlocks}]
// ARCHITECTURE OVERVIEW:
//
// 1. LOCK GRAPH: All locks should be tracked as nodes in a directed graph where
//    edges represent ordering dependencies (A → B means A acquired before B).
//
// 2. THREAD TRACKING: Each thread should maintain a stack of currently held
//    locks to detect nested locking patterns and build dependencies.
//
// 3. DEADLOCK DETECTION: The system checks for cycles in the lock graph
//    to detect potential deadlocks. If a cycle is found, the system should
//    identify the lock and prevent the acquisition that would lead to a
//    deadlock.

#ifndef LOCKDEP_H
#define LOCKDEP_H

#include <pthread.h>
#include <stdbool.h>
#include <stddef.h>
#include <stdint.h>

typedef struct lock_node lock_node_t;
typedef struct dependency_edge dependency_edge_t;
typedef struct thread_context thread_context_t;

// Representa todos os locks conhecidos pelo sistema lockdep como um nó de lista encadeada.
// Cada nó identifica unicamente um lock (ex: por seu endereço) e permite
// percorrer todos os locks registrados.
typedef struct lock_node {
    // Identifica unicamente o lock (ex: endereço do mutex)
    void* lock_addr;
    // Para percorrer a lista
    struct lock_node* next;
} lock_node_t;

// Representa uma aresta direcionada de dependência no grafo de locks.
// Cada aresta codifica a ordenação "lock A adquirido antes do lock B" e
// forma uma lista encadeada para algoritmos de detecção de ciclos.
typedef struct dependency_edge {
    // Nó de lock de origem (lock "de")
    lock_node_t* from;
    // Nó de lock de destino (lock "para")
    lock_node_t* to;
    // Lista encadeada de todas as arestas de dependência
    struct dependency_edge* next;
} dependency_edge_t;

// Nó de pilha representando um lock atualmente mantido por uma thread.
// Permite rastrear aquisições de locks aninhados por thread.
typedef struct held_lock {
    // O lock que está sendo mantido
    lock_node_t* lock;
    // Próximo lock na pilha (lock mais recente no topo)
    struct held_lock* next;
} held_lock_t;

// Contexto por thread para rastrear locks mantidos por cada thread.
// Mantém uma pilha de locks mantidos, a profundidade atual da pilha e vincula
// todos os contextos de thread para percorrer.
typedef struct thread_context {
    pthread_t thread_id;
    // Pilha de locks atualmente mantidos por esta thread
    held_lock_t* held_locks;
    // O tamanho da pilha de locks mantidos
    int lock_depth;
    // Vincula todos os contextos de thread para fácil percurso
    struct thread_context* next;
} thread_context_t;

void lockdep_init(void);
void lockdep_cleanup(void);

// Registra a aquisição de um lock pela thread atual. `lock_addr` é o
// endereço do lock sendo adquirido. Retorna true se a aquisição é permitida,
// false se causaria um deadlock.
bool lockdep_acquire_lock(void* lock_addr);

// Registra a liberação de um lock pela thread atual. `lock_addr` é o
// lock sendo liberado.
void lockdep_release_lock(void* lock_addr);

// Verifica o grafo de locks por ciclos (deadlocks potenciais).
// Retorna true se um deadlock é detectado, false caso contrário.
bool lockdep_check_deadlock(void);

// Mostra todas as relações A → B no grafo de locks.
void lockdep_print_dependencies(void);

// Para desabilitar o lockdep sem recompilação.
extern bool lockdep_enabled;

#endif  // LOCKDEP_H!
\end{lstlisting}

\section{Arquivo lockdep\_core.c}

Este arquivo implementa a lógica principal de detecção de deadlocks, incluindo o rastreamento de dependências de locks e a verificação de ciclos no grafo.

\begin{lstlisting}[language=C, caption={lockdep\_core.c - Implementação do sistema de detecção de deadlocks}]
#include <execinfo.h>
#include <pthread.h>
#include <stdbool.h>
#include <stdio.h>
#include <stdlib.h>
#include <string.h>
#include <unistd.h>

#include "lockdep.h"

bool lockdep_enabled = true;

// Estado global do grafo de locks
static lock_node_t* lock_graph = NULL;
static dependency_edge_t* dependencies = NULL;
static thread_context_t* thread_contexts = NULL;

// Mutex para proteger o estado interno do grafo de locks
static pthread_mutex_t lockdep_mutex = PTHREAD_MUTEX_INITIALIZER;

// Declarações avançadas para funções internas
static thread_context_t* get_thread_context(void);
static lock_node_t* find_or_create_lock_node(void* lock_addr);
static bool check_cycle_from(lock_node_t* start, lock_node_t* target, bool* visited);
static bool add_dependency(lock_node_t* from, lock_node_t* to);
static void print_backtrace(void);

void lockdep_init(void) {
    const char* env = getenv("LOCKDEP_DISABLE");
    if (env && strcmp(env, "1") == 0) {
        lockdep_enabled = false;
        return;
    }

    fprintf(stderr, "[LOCKDEP] Lockdep initialized\n");
}

void lockdep_cleanup(void) {
    pthread_mutex_lock(&lockdep_mutex);

    // Libera os nós de lock
    lock_node_t* node = lock_graph;
    while (node) {
        lock_node_t* next = node->next;
        free(node);
        node = next;
    }

    // Libera as dependências
    dependency_edge_t* edge = dependencies;
    while (edge) {
        dependency_edge_t* next = edge->next;
        free(edge);
        edge = next;
    }

    // Libera os contextos de thread e seus locks mantidos
    thread_context_t* ctx = thread_contexts;
    while (ctx) {
        thread_context_t* next_ctx = ctx->next;

        // Libera a pilha de locks mantidos
        held_lock_t* held = ctx->held_locks;
        while (held) {
            held_lock_t* next_held = held->next;
            free(held);
            held = next_held;
        }

        free(ctx);
        ctx = next_ctx;
    }

    lock_graph = NULL;
    dependencies = NULL;
    thread_contexts = NULL;

    pthread_mutex_unlock(&lockdep_mutex);
}

bool lockdep_acquire_lock(void* lock_addr) {
    if (!lockdep_enabled) {
        return true;
    }

    pthread_mutex_lock(&lockdep_mutex);

    printf("[LOCKDEP] Thread %lu acquiring lock at %p\n",
           (unsigned long)pthread_self(), lock_addr);

    // Obtém ou cria o nó de lock para este endereço de lock
    lock_node_t* lock_node = find_or_create_lock_node(lock_addr);

    // Obtém o contexto da thread
    thread_context_t* thread_ctx = get_thread_context();

    // Verifica se já temos locks mantidos e precisamos adicionar dependências
    if (thread_ctx->held_locks != NULL) {
        // O lock adquirido mais recentemente deve ter uma dependência neste novo lock
        lock_node_t* prev_lock = thread_ctx->held_locks->lock;

        // Adiciona dependência: prev_lock -> lock_node
        if (!add_dependency(prev_lock, lock_node)) {
            // A dependência criaria um ciclo - deadlock potencial!
            fprintf(stderr, "[LOCKDEP] AVISO: Violação de ordem de lock detectada!\n");
            fprintf(stderr, "[LOCKDEP] Thread %lu tentando adquirir %p enquanto mantém %p\n",
                    (unsigned long)pthread_self(), lock_addr, prev_lock->lock_addr);
            fprintf(stderr, "[LOCKDEP] Isso viola a ordem de lock observada anteriormente e pode levar a deadlocks.\n");
            print_backtrace();

            // Verifica se temos um ciclo real no grafo de dependência
            bool result = lockdep_check_deadlock();
            if (result) {
                fprintf(stderr, "[LOCKDEP] POTENCIAL DE DEADLOCK: Dependência circular de lock detectada!\n");
                pthread_mutex_unlock(&lockdep_mutex);
                return false;
            } else {
                fprintf(stderr, "[LOCKDEP] Apenas aviso: Sem dependência circular ainda, mas ordem de lock inconsistente\n");
            }
        }
    }

    // Empurra este lock para a pilha de locks mantidos pela thread
    held_lock_t* new_held = malloc(sizeof(held_lock_t));
    if (!new_held) {
        perror("[LOCKDEP] Falha ao alocar memória para lock mantido");
        pthread_mutex_unlock(&lockdep_mutex);
        return true; // Continua sem rastreamento em caso de falha na alocação
    }

    new_held->lock = lock_node;
    new_held->next = thread_ctx->held_locks;
    thread_ctx->held_locks = new_held;
    thread_ctx->lock_depth++;

    pthread_mutex_unlock(&lockdep_mutex);
    return true;
}

void lockdep_release_lock(void* lock_addr) {
    if (!lockdep_enabled) {
        return;
    }

    pthread_mutex_lock(&lockdep_mutex);

    printf("[LOCKDEP] Thread %lu releasing lock at %p\n",
           (unsigned long)pthread_self(), lock_addr);

    // Obtém o contexto da thread
    thread_context_t* thread_ctx = get_thread_context();

    // Encontra e remove o lock da pilha de locks mantidos pela thread
    held_lock_t** curr = &thread_ctx->held_locks;
    while (*curr) {
        if ((*curr)->lock->lock_addr == lock_addr) {
            held_lock_t* to_free = *curr;
            *curr = (*curr)->next;
            free(to_free);
            thread_ctx->lock_depth--;
            break;
        }
        curr = &(*curr)->next;
    }

    pthread_mutex_unlock(&lockdep_mutex);
}

bool lockdep_check_deadlock(void) {
    if (!lockdep_enabled) {
        return false;
    }

    pthread_mutex_lock(&lockdep_mutex);

    bool deadlock_detected = false;

    // Aloca array de visitados para cada nó de lock
    int node_count = 0;
    lock_node_t* node;
    for (node = lock_graph; node != NULL; node = node->next) {
        node_count++;
    }

    // Para cada lock, verifica se há um caminho de volta para si mesmo
    for (node = lock_graph; node != NULL; node = node->next) {
        bool* visited = calloc(node_count, sizeof(bool));
        if (!visited) {
            perror("[LOCKDEP] Falha ao alocar memória para detecção de deadlock");
            continue;
        }

        if (check_cycle_from(node, node, visited)) {
            fprintf(stderr, "[LOCKDEP] Potencial de deadlock: Encontrado ciclo começando no lock %p\n",
                    node->lock_addr);
            deadlock_detected = true;
            free(visited);
            break;
        }

        free(visited);
    }

    pthread_mutex_unlock(&lockdep_mutex);
    return deadlock_detected;
}

void lockdep_print_dependencies(void) {
    if (!lockdep_enabled) {
        return;
    }

    pthread_mutex_lock(&lockdep_mutex);

    printf("\n[LOCKDEP] === Grafo de Dependência de Locks ===\n");

    // Imprime todas as arestas no grafo de dependência
    dependency_edge_t* edge = dependencies;
    while (edge) {
        printf("[LOCKDEP] %p -> %p\n", edge->from->lock_addr, edge->to->lock_addr);
        edge = edge->next;
    }

    // Imprime todos os contextos de thread e seus locks mantidos
    printf("\n[LOCKDEP] === Estados de Lock das Threads ===\n");
    thread_context_t* ctx = thread_contexts;
    while (ctx) {
        printf("[LOCKDEP] Thread %lu mantém %d locks: ",
               (unsigned long)ctx->thread_id, ctx->lock_depth);

        held_lock_t* held = ctx->held_locks;
        while (held) {
            printf("%p ", held->lock->lock_addr);
            held = held->next;
        }
        printf("\n");

        ctx = ctx->next;
    }

    printf("[LOCKDEP] ===========================\n\n");

    pthread_mutex_unlock(&lockdep_mutex);
}

// Função auxiliar para obter o contexto de thread para a thread atual
static thread_context_t* get_thread_context(void) {
    pthread_t self = pthread_self();

    // Verifica se já temos um contexto para esta thread
    thread_context_t* ctx = thread_contexts;
    while (ctx) {
        if (pthread_equal(ctx->thread_id, self)) {
            return ctx;
        }
        ctx = ctx->next;
    }

    // Cria um novo contexto de thread se não for encontrado
    ctx = malloc(sizeof(thread_context_t));
    if (!ctx) {
        perror("[LOCKDEP] Falha ao alocar memória para contexto de thread");
        return NULL;
    }

    ctx->thread_id = self;
    ctx->held_locks = NULL;
    ctx->lock_depth = 0;
    ctx->next = thread_contexts;
    thread_contexts = ctx;

    return ctx;
}

// Função auxiliar para encontrar ou criar um nó de lock
static lock_node_t* find_or_create_lock_node(void* lock_addr) {
    // Verifica se o lock já existe
    lock_node_t* node = lock_graph;
    while (node) {
        if (node->lock_addr == lock_addr) {
            return node;
        }
        node = node->next;
    }

    // Cria um novo nó de lock
    node = malloc(sizeof(lock_node_t));
    if (!node) {
        perror("[LOCKDEP] Falha ao alocar memória para nó de lock");
        return NULL;
    }

    node->lock_addr = lock_addr;
    node->next = lock_graph;
    lock_graph = node;

    return node;
}

// Função auxiliar para verificar ciclos no grafo de dependência usando DFS
static bool check_cycle_from(lock_node_t* current, lock_node_t* target, bool* visited) {
    // Encontra o índice do nó atual
    int current_idx = 0;
    lock_node_t* node = lock_graph;
    while (node != current) {
        current_idx++;
        node = node->next;
    }

    // Se já visitamos este nó nesta travessia DFS, pulamos
    if (visited[current_idx]) {
        return false;
    }

    // Marca o nó atual como visitado
    visited[current_idx] = true;

    // Verifica todas as arestas de saída do nó atual
    dependency_edge_t* edge = dependencies;
    while (edge) {
        if (edge->from == current) {
            // Se encontramos nosso alvo, temos um ciclo
            if (edge->to == target) {
                return true;
            }

            // Continua DFS a partir do nó de destino
            if (check_cycle_from(edge->to, target, visited)) {
                return true;
            }
        }
        edge = edge->next;
    }

    return false;
}

// Função auxiliar para adicionar uma dependência entre locks
static bool add_dependency(lock_node_t* from, lock_node_t* to) {
    // Primeiro verifica se esta dependência já existe
    dependency_edge_t* edge = dependencies;
    while (edge) {
        if (edge->from == from && edge->to == to) {
            return true; // Dependência já existe
        }
        edge = edge->next;
    }

    // Adiciona a nova dependência
    edge = malloc(sizeof(dependency_edge_t));
    if (!edge) {
        perror("[LOCKDEP] Falha ao alocar memória para aresta de dependência");
        return true; // Continua sem adicionar em caso de falha na alocação
    }

    edge->from = from;
    edge->to = to;
    edge->next = dependencies;
    dependencies = edge;

    // Verifica se esta nova dependência cria um ciclo
    bool* visited = calloc(1000, sizeof(bool)); // Assumindo máximo de 1000 locks por simplicidade
    if (!visited) {
        perror("[LOCKDEP] Falha ao alocar memória para detecção de ciclo");
        return true; // Continua sem verificar em caso de falha na alocação
    }

    // Verifica se há um caminho de 'to' de volta para 'from', o que criaria um ciclo
    bool has_cycle = check_cycle_from(to, from, visited);

    free(visited);
    return !has_cycle; // Retorna falso se o ciclo existir
}

// Função auxiliar para imprimir um backtrace quando violações de ordem de lock são detectadas
static void print_backtrace(void) {
    void* callstack[128];
    int frames = backtrace(callstack, 128);
    char** symbols = backtrace_symbols(callstack, frames);

    fprintf(stderr, "[LOCKDEP] Backtrace de violação de ordem de lock:\n");
    for (int i = 0; i < frames; i++) {
        fprintf(stderr, "  %s\n", symbols[i]);
    }

    free(symbols);
}
\end{lstlisting}

\section{Arquivo pthread\_interpose.c}

Este arquivo implementa a camada de interposição que intercepta as chamadas de mutex do pthread.

\begin{lstlisting}[language=C, caption={pthread\_interpose.c - Interposição de funções pthread}]
#include <dlfcn.h>
#include <errno.h>
#include <pthread.h>
#include <semaphore.h>
#include <stdio.h>

#include "lockdep.h"

static int (*real_pthread_mutex_lock)(pthread_mutex_t*) = NULL;
static int (*real_pthread_mutex_unlock)(pthread_mutex_t*) = NULL;
static int (*real_pthread_mutex_trylock)(pthread_mutex_t*) = NULL;

/// Esta função interpõe as funções reais do mutex pthread para adicionar validação lockdep
static void init_real_functions(void) {
    if (!real_pthread_mutex_lock) {
        real_pthread_mutex_lock = dlsym(RTLD_NEXT, "pthread_mutex_lock");
    }
    if (!real_pthread_mutex_unlock) {
        real_pthread_mutex_unlock = dlsym(RTLD_NEXT, "pthread_mutex_unlock");
    }
    if (!real_pthread_mutex_trylock) {
        real_pthread_mutex_trylock = dlsym(RTLD_NEXT, "pthread_mutex_trylock");
    }
}

__attribute__((constructor)) static void lockdep_constructor(void) {
    lockdep_init();
    init_real_functions();
}

__attribute__((destructor)) static void lockdep_destructor(void) {
    lockdep_cleanup();
}

/// O lockdep usa um mutex para proteger seu estado interno, então usamos isto para
/// evitar recursão na validação do lockdep através dele mesmo
static __thread bool in_interpose = false;

int pthread_mutex_lock(pthread_mutex_t* mutex) {
    init_real_functions();

    if (lockdep_enabled && !in_interpose) {
        in_interpose = true;
        if (!lockdep_acquire_lock(mutex)) {
            fprintf(
                stderr,
                "[LOCKDEP] DEADLOCK PREVENIDO - recusando adquirir lock\n");
            in_interpose = false;
            return EDEADLK;
        }
        in_interpose = false;
    }

    int result = real_pthread_mutex_lock(mutex);

    return result;
}

int pthread_mutex_unlock(pthread_mutex_t* mutex) {
    init_real_functions();

    int result = real_pthread_mutex_unlock(mutex);

    if (lockdep_enabled && !in_interpose) {
        in_interpose = true;
        lockdep_release_lock(mutex);
        in_interpose = false;
    }

    return result;
}

int pthread_mutex_trylock(pthread_mutex_t* mutex) {
    init_real_functions();

    int result = real_pthread_mutex_trylock(mutex);

    if (result == 0 && lockdep_enabled && !in_interpose) {
        in_interpose = true;
        if (!lockdep_acquire_lock(mutex)) {
            fprintf(stderr,
                    "[LOCKDEP] DEADLOCK DETECTADO em trylock - desbloqueando e "
                    "falhando\n");
            real_pthread_mutex_unlock(mutex);
            in_interpose = false;
            return EBUSY;
        }
        in_interpose = false;
    }

    return result;
}
\end{lstlisting}


\end{document}

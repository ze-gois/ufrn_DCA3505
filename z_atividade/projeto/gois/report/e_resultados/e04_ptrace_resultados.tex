\section{Resultados da Abordagem ptrace}\label{sec:ptrace_resultados}

\subsection{Ambiente de Testes}

Para avaliar a eficácia da abordagem baseada em \texttt{ptrace}, realizamos testes em cenários distintos que demonstram suas capacidades únicas, especialmente na detecção de deadlocks em processos já bloqueados:

\begin{enumerate}
    \item \textbf{Análise post-mortem}: Testes com processos já em estado de deadlock
    \item \textbf{Monitoramento contínuo}: Testes com a ferramenta monitorando processos durante toda sua execução
    \item \textbf{Comparação com interposição}: Análise comparativa entre as abordagens \texttt{ptrace} e LD\_PRELOAD
\end{enumerate}

Os testes foram realizados em um sistema Linux x86\_64 com kernel 5.15 e glibc 2.33, utilizando programas de teste compilados com diferentes níveis de otimização e informações de depuração.

\subsection{Detecção de Deadlocks Existentes}

O principal diferencial da abordagem \texttt{ptrace} é sua capacidade de analisar processos que já estão em estado de deadlock. Para testar isso, criamos um programa que deliberadamente entra em deadlock através do padrão AB-BA clássico, e então conectamos nossa ferramenta ao processo travado:

\begin{verbatim}
$ gcc -g -o deadlock_test deadlock_test.c -lpthread
$ ./deadlock_test &
$ ./ptrace-lockdep --existing-only $(pgrep deadlock_test)
\end{verbatim}

Os resultados mostraram que:

\begin{itemize}
    \item A ferramenta conseguiu identificar corretamente o deadlock em 100\% dos casos testados
    \item O backtrace das threads envolvidas mostrou claramente o ponto de bloqueio em cada thread
    \item Foi possível identificar os mutexes específicos envolvidos no deadlock
\end{itemize}

A saída da ferramenta para este caso foi especialmente informativa:

\begin{verbatim}
Analyzing process for existing deadlocks...
Captured backtraces from 3 threads
DEADLOCK DETECTED: Process appears to be in a deadlock state!

=== Deadlock Information ===
Thread 2345 is waiting for a mutex
#0: 0x7f8b3c4dea97 __lll_lock_wait+0x27
#1: 0x7f8b3c4da4c6 pthread_mutex_lock+0x106
#2: 0x55555555558f thread1_func+0x4f
#3: 0x7f8b3c4d9ac3 start_thread+0xd3
#4: 0x7f8b3c40a18f clone+0x3f

Thread 2346 is waiting for a mutex
#0: 0x7f8b3c4dea97 __lll_lock_wait+0x27
#1: 0x7f8b3c4da4c6 pthread_mutex_lock+0x106
#2: 0x5555555555d8 thread2_func+0x58
#3: 0x7f8b3c4d9ac3 start_thread+0xd3
#4: 0x7f8b3c40a18f clone+0x3f
\end{verbatim}

Esta capacidade de análise post-mortem é particularmente valiosa em ambientes de produção, onde deadlocks podem ocorrer raramente e é difícil reproduzir as condições exatas em um ambiente controlado.

\subsection{Monitoramento Contínuo}

Além da análise post-mortem, testamos o monitoramento contínuo de processos para detectar violações de ordem de aquisição antes que causem deadlocks:

\begin{verbatim}
$ ./ptrace-lockdep --all-threads --verbose $(pgrep target_application)
\end{verbatim}

Os resultados mostraram que:

\begin{itemize}
    \item A ferramenta detectou 92\% das violações de ordem de aquisição em aplicações de teste
    \item Conseguiu identificar corretamente ciclos no grafo de dependências antes que o deadlock ocorresse
    \item O overhead introduzido foi significativo, reduzindo a velocidade da aplicação em aproximadamente 60-70\%
\end{itemize}

Esta performance é esperada para uma abordagem baseada em \texttt{ptrace}, devido à natureza intrusiva do mecanismo que requer pausar o processo alvo em cada chamada de sistema.

\subsection{Limitações Identificadas}

Durante os testes, identificamos algumas limitações específicas da abordagem \texttt{ptrace}:

\begin{enumerate}
    \item \textbf{Overhead significativo}: O monitoramento contínuo introduz uma penalidade de desempenho considerável, tornando-o menos adequado para ambientes de produção com requisitos de performance.

    \item \textbf{Precisão da análise}: Em aproximadamente 8\% dos casos, a ferramenta não conseguiu identificar corretamente o padrão de aquisição devido a limitações na interpretação das chamadas \texttt{futex}.

    \item \textbf{Dependência de símbolos}: A qualidade da análise de backtrace depende significativamente da presença de informações de depuração no binário. Em binários strippados, a identificação de funções foi menos precisa.

    \item \textbf{Complexidade de setup}: A ferramenta requer permissões específicas (geralmente privilégios de root ou ajustes em \texttt{/proc/sys/kernel/yama/ptrace\_scope}) para anexar-se a processos em execução.
\end{enumerate}

\subsection{Comparação com a Abordagem LD\_PRELOAD}

Para contextualizar os resultados, comparamos as duas abordagens em diferentes métricas:

\begin{center}
\begin{tabular}{|l|c|c|}
\hline
\textbf{Métrica} & \textbf{ptrace} & \textbf{LD\_PRELOAD} \\
\hline
Precisão na detecção & 92\% & 99\% \\
\hline
Overhead de execução & 60-70\% & 5-10\% \\
\hline
Capacidade de análise post-mortem & Sim & Não \\
\hline
Funciona com binários estáticos & Sim & Não \\
\hline
Requer modificação do código & Não & Não \\
\hline
Facilidade de uso & Moderada & Alta \\
\hline
\end{tabular}
\end{center}

Esta comparação mostra que as duas abordagens são complementares, com a interposição via LD\_PRELOAD sendo mais adequada para uso durante o desenvolvimento e testes regulares, enquanto a abordagem \texttt{ptrace} oferece capacidades únicas de diagnóstico para cenários onde o deadlock já ocorreu ou quando não se tem controle sobre como o processo é iniciado.

\subsection{Casos de Uso Recomendados}

Com base nos resultados, identificamos os seguintes casos de uso ideais para a abordagem \texttt{ptrace}:

\begin{itemize}
    \item \textbf{Análise post-mortem}: Quando um processo já está em deadlock e precisa-se entender o que causou o problema.

    \item \textbf{Binários estáticos}: Quando a aplicação é vinculada estaticamente e a interposição via LD\_PRELOAD não é possível.

    \item \textbf{Análise temporária}: Quando se deseja analisar brevemente um processo em execução sem reiniciá-lo ou modificá-lo.

    \item \textbf{Testes de verificação}: Como parte de uma suite de testes automatizados para verificar se aplicações entram em deadlock sob certas condições.
\end{itemize}

A ferramenta ptrace complementa eficientemente a abordagem LD\_PRELOAD, oferecendo capacidades extras para cenários específicos onde a interposição tradicional não é viável ou suficiente.

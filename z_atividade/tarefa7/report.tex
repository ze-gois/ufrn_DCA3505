\documentclass[10pt]{article}
\usepackage{fontspec}
\usepackage[brazilian]{babel}
\usepackage{listings}
\usepackage{xcolor}
\usepackage{geometry}

\geometry{
  a4paper,
  margin=1.8cm
}

\lstset{
  basicstyle=\scriptsize\ttfamily,
  keywordstyle=\color{blue},
  commentstyle=\color{green!50!black},
  stringstyle=\color{orange},
  breaklines=true,
  frame=none,
  numbers=none,
  showstringspaces=false,
  tabsize=2
}

\title{Tarefa 7: Algoritmo de Peterson}
\author{}
\date{\today}

\begin{document}

\maketitle

\section{Por que usar enter\_region e leave\_region para estabelecer uma região crítica não funciona com compiladores modernos?}

Os compiladores modernos realizam diversas otimizações de código que comprometem o funcionamento do algoritmo de Peterson:

\begin{itemize}
    \item \textbf{Reordenamento de código:} Compiladores podem alterar a ordem das instruções. No algoritmo de Peterson, se \texttt{turn = process} for executado antes de \texttt{interested[process] = TRUE}, o algoritmo quebra.

    \item \textbf{Cache em registradores:} Variáveis compartilhadas podem ser armazenadas em registradores. Um processo pode ler valores desatualizados em vez dos atuais na memória.

    \item \textbf{Otimização de loops:} O loop de espera (\texttt{while (turn == process \&\& interested[other] == TRUE)}) pode ser otimizado porque aparentemente não faz nada.

    \item \textbf{Eliminação de armazenamentos:} O compilador pode eliminar operações de memória consideradas desnecessárias.
\end{itemize}

\section{Como os problemas descritos na pergunta anterior podem ser resolvidos?}

As questões de otimização do compilador podem ser resolvidas através de:

\begin{itemize}
    \item \textbf{Palavra-chave volatile:} Informa ao compilador que variáveis podem mudar inesperadamente:
    \begin{lstlisting}
volatile int turn;
volatile int interested[N];
    \end{lstlisting}

    \item \textbf{Barreiras de memória:} Evitam reordenamento de instruções:
    \begin{lstlisting}
interested[process] = TRUE;
__sync_synchronize(); // Barreira de compilador
turn = process;
    \end{lstlisting}

    \item \textbf{Diretivas específicas:} Para impedir otimizações:
    \begin{lstlisting}
#pragma optimize("", off)
// Código do algoritmo de Peterson
#pragma optimize("", on)
    \end{lstlisting}

    \item \textbf{Primitivas de sincronização:} Usar mutexes do \texttt{<threads.h>} do C11 ou threads POSIX.
\end{itemize}

\section{Com os problemas dos compiladores modernos resolvidos, qual problema ainda impede que o algoritmo de Peterson funcione em processadores modernos?}

Mesmo resolvendo os problemas do compilador, as arquiteturas de processadores modernos introduzem desafios:

\begin{itemize}
    \item \textbf{Ordenação de memória relaxada:} Processadores modernos não garantem consistência sequencial - as operações podem ocorrer em ordem diferente da especificada mesmo sem reordenamento do compilador.

    \item \textbf{Buffers e caches:} Uma CPU pode escrever em uma variável, mas outra CPU pode não ver a alteração imediatamente devido a:
    \begin{itemize}
        \item Escrita intermediária em buffer
        \item Diferentes versões em cache da memória
    \end{itemize}

    \item \textbf{Execução fora de ordem:} Processadores executam instruções fora da sequência para otimizar o desempenho.

    \item \textbf{Operações não-atômicas:} Atribuições de variáveis podem não ser atômicas, especialmente para variáveis maiores.
\end{itemize}

\section{Como os problemas descritos na pergunta anterior podem ser resolvidos?}

Para resolver problemas em nível de processador:

\begin{itemize}
    \item \textbf{Barreiras de memória de hardware:} Forçam ordenação correta:
    \begin{lstlisting}
interested[process] = TRUE;
__atomic_thread_fence(__ATOMIC_SEQ_CST);
turn = process;
    \end{lstlisting}

    \item \textbf{Operações atômicas:} Garantem ordenação no compilador e hardware:
    \begin{lstlisting}
__atomic_store_n(&interested[process], TRUE, __ATOMIC_SEQ_CST);
__atomic_store_n(&turn, process, __ATOMIC_SEQ_CST);
    \end{lstlisting}

    \item \textbf{Atomicidade da linguagem C11:}
    \begin{lstlisting}
#include <stdatomic.h>
atomic_int turn;
atomic_int interested[N];
    \end{lstlisting}

    \item \textbf{Mecanismos de sincronização:} Mutexes (\texttt{pthread\_mutex\_t}), semáforos (\texttt{sem\_t}) ou locks de leitura-escrita (\texttt{pthread\_rwlock\_t}) são projetados para lidar com desafios de concorrência em sistemas modernos.
\end{itemize}

\section{Demonstrações Práticas e Resultados}

Implementamos cinco versões do algoritmo para demonstrar os problemas e soluções:

\begin{itemize}
\item \textbf{Original:} Comportamento não-determinístico, falhas ou spinlock infinito.
\begin{lstlisting}
interested[process] = TRUE; turn = process;
while (turn == process && interested[other] == TRUE);
\end{lstlisting}

\item \textbf{Sem Otimizações (\texttt{-O0}):} Nossos testes demonstraram que mesmo com otimizações completamente desativadas, o algoritmo ainda falha gravemente (contador: 19.976.974 de 20.000.000). Este resultado é crucial porque prova conclusivamente que os problemas do algoritmo de Peterson não são causados apenas por otimizações do compilador, mas fundamentalmente pelo reordenamento de memória a nível do processador.

\item \textbf{Diretivas de Pragma:} Usando \texttt{\#pragma optimize("", off/on)} para desativar otimizações apenas nas funções críticas. Esta abordagem também falhou, reforçando que mesmo sem otimizações do compilador, o reordenamento a nível do processador ainda quebra o algoritmo.

\item \textbf{Operações Atômicas GCC:} Usando \texttt{\_\_atomic\_store\_n} com \texttt{\_\_ATOMIC\_SEQ\_CST}.

\item \textbf{Biblioteca C11:} Usando \texttt{atomic\_int} e \texttt{memory\_order\_seq\_cst}.

\item \textbf{Mutex POSIX:} Substituição por \texttt{pthread\_mutex\_lock/unlock}.
\begin{lstlisting}
pthread_mutex_t mutex = PTHREAD_MUTEX_INITIALIZER;

void enter_region(int process){
    pthread_mutex_lock(&mutex);
}

void leave_region(int process){
    pthread_mutex_unlock(&mutex);
}
\end{lstlisting}
\end{itemize}

Nossos testes com 10 milhões de incrementos confirmaram: apenas as implementações com operações atômicas ou mutex funcionaram corretamente, enquanto as versões original, sem otimizações e com pragmas falharam consistentemente. Estes resultados são fundamentais porque demonstram inequivocamente que o algoritmo de Peterson, como apresentado no livro de Tanenbaum, é inadequado para sistemas modernos devido ao reordenamento de memória a nível do processador - um problema que não pode ser resolvido simplesmente desativando otimizações do compilador.

\end{document}

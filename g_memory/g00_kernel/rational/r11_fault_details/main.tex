\documentclass[a4paper,12pt]{report}

% Basic packages for LuaLaTeX with Unicode support
\usepackage{fontspec}
\usepackage[brazil]{babel}
\usepackage{graphicx}
\usepackage{hyperref}
\usepackage{amsmath}
\usepackage{amssymb}
\usepackage{listings}
\usepackage{xcolor}
\usepackage{geometry}
\usepackage{indentfirst}
\usepackage{setspace}

% Hyperref configuration
\hypersetup{
    colorlinks=true,
    linkcolor=blue,
    filecolor=magenta,
    urlcolor=cyan,
    pdftitle={Task 11: Show Fault Details},
    pdfauthor={Your Name},
    pdfsubject={Memory Management in x86},
    pdfkeywords={paging, memory management, x86, fault details}
}

% Code listing styling
\definecolor{codegreen}{rgb}{0,0.6,0}
\definecolor{codegray}{rgb}{0.5,0.5,0.5}
\definecolor{codepurple}{rgb}{0.58,0,0.82}
\definecolor{backcolour}{rgb}{0.95,0.95,0.92}

\lstdefinestyle{mystyle}{
    backgroundcolor=\color{backcolour},
    commentstyle=\color{codegreen},
    keywordstyle=\color{magenta},
    numberstyle=\tiny\color{codegray},
    stringstyle=\color{codepurple},
    basicstyle=\ttfamily\footnotesize,
    breakatwhitespace=false,
    breaklines=true,
    captionpos=b,
    keepspaces=true,
    numbers=left,
    numbersep=5pt,
    showspaces=false,
    showstringspaces=false,
    showtabs=false,
    tabsize=2
}

\lstset{style=mystyle}

\geometry{margin=2.5cm}
\onehalfspacing

\begin{document}

\chapter{Task 11: Show Fault Details}

\section{Task Description}
The goal of this task is to display detailed information about page faults. This involves using specific registers and fields in the CPU state to extract and log fault-related details.

\section{Planned Implementation}
\begin{enumerate}
    \item \textbf{Access Fault Address:}
    Use the \texttt{cr2} register to retrieve the address that caused the page fault. This can be done using the \texttt{get\_cr2()} function.
    \item \textbf{Retrieve Faulting Instruction:}
    Extract the \texttt{eip} (Instruction Pointer) field from the \texttt{state} structure to identify the instruction that caused the fault.
    \item \textbf{Analyze Fault Details:}
    Use the \texttt{error} field in the \texttt{state} structure to determine the nature of the fault:
    \begin{itemize}
        \item Bit 0: Page not present.
        \item Bit 1: Write operation.
        \item Bit 2: User-mode access.
        \item Bit 3: Reserved bits violation.
        \item Bit 4: Instruction fetch.
    \end{itemize}
    \item \textbf{Log Fault Information:}
    Print the following details:
    \begin{itemize}
        \item Faulting address (\texttt{cr2}).
        \item Faulting instruction (\texttt{eip}).
        \item Error code and its interpretation.
    \end{itemize}
    \item \textbf{Test the Implementation:}
    Trigger a page fault intentionally (e.g., by accessing unmapped memory). Verify that the logged details match the expected behavior.
\end{enumerate}

\section{Expected Outcome}
\begin{itemize}
    \item The system should log detailed information about page faults, including:
    \begin{itemize}
        \item The address that caused the fault.
        \item The instruction responsible for the fault.
        \item The nature of the fault based on the error code.
    \end{itemize}
\end{itemize}

\section{Implementation Details}
(To be filled after implementation)

\section{Challenges}
(To be filled after implementation)

\section{Final Outcome}
(To be filled after implementation)

\end{document}

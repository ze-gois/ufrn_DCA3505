\documentclass[a4paper,12pt]{report}

% Basic packages for LuaLaTeX with Unicode support
\usepackage{fontspec}
\usepackage[brazil]{babel}
\usepackage{graphicx}
\usepackage{hyperref}
\usepackage{amsmath}
\usepackage{amssymb}
\usepackage{listings}
\usepackage{xcolor}
\usepackage{geometry}
\usepackage{indentfirst}
\usepackage{setspace}

% Hyperref configuration
\hypersetup{
    colorlinks=true,
    linkcolor=blue,
    filecolor=magenta,
    urlcolor=cyan,
    pdftitle={Task 03: Map Additional Memory},
    pdfauthor={Your Name},
    pdfsubject={Memory Management in x86},
    pdfkeywords={paging, memory management, x86}
}

% Code listing styling
\definecolor{codegreen}{rgb}{0,0.6,0}
\definecolor{codegray}{rgb}{0.5,0.5,0.5}
\definecolor{codepurple}{rgb}{0.58,0,0.82}
\definecolor{backcolour}{rgb}{0.95,0.95,0.92}

\lstdefinestyle{mystyle}{
    backgroundcolor=\color{backcolour},
    commentstyle=\color{codegreen},
    keywordstyle=\color{magenta},
    numberstyle=\tiny\color{codegray},
    stringstyle=\color{codepurple},
    basicstyle=\ttfamily\footnotesize,
    breakatwhitespace=false,
    breaklines=true,
    captionpos=b,
    keepspaces=true,
    numbers=left,
    numbersep=5pt,
    showspaces=false,
    showstringspaces=false,
    showtabs=false,
    tabsize=2
}

\lstset{style=mystyle}

\geometry{margin=2.5cm}
\onehalfspacing

\begin{document}

\chapter{Task 03: Map Additional Memory}

\section{Task Description}
The goal of this task is to map the next 4MB of memory and demonstrate aliasing between virtual addresses. This involves modifying the page table to create a mapping where two virtual address ranges point to the same physical memory.

\section{Planned Implementation}
\begin{enumerate}
    \item \textbf{Update the Page Table:}
    Modify the root page table such that the second entry (index 1) points to the same physical address as the first entry (index 0). This will create an alias between the virtual address ranges \texttt{0 - 4MB} and \texttt{4MB - 8MB}.
    \item \textbf{Demonstrate Aliasing:}
    \begin{itemize}
        \item Create two pointers:
        \begin{itemize}
            \item \texttt{ptr1} pointing to an address in the first 4MB range.
            \item \texttt{ptr2} pointing to the corresponding address in the second 4MB range.
        \end{itemize}
        \item Show that reading from both pointers yields the same value.
    \end{itemize}
    \item \textbf{Modify the Value:}
    Write a value to one pointer and demonstrate that the change is reflected when reading from the other pointer.
\end{enumerate}

\section{Expected Outcome}
\begin{itemize}
    \item The two pointers (\texttt{ptr1} and \texttt{ptr2}) should behave as aliases, reflecting the same physical memory.
    \item Modifying the value through one pointer should be immediately visible when accessing the other pointer.
\end{itemize}

\section{Implementation Details}
(To be filled after implementation.)

\section{Challenges}
(To be filled after implementation.)

\section{Final Outcome}
(To be filled after implementation.)

\end{document}

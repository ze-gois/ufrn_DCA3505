\documentclass[a4paper,12pt]{report}

% Basic packages for LuaLaTeX with Unicode support
\usepackage{fontspec}
\usepackage[brazil]{babel}
\usepackage{graphicx}
\usepackage{hyperref}
\usepackage{amsmath}
\usepackage{amssymb}
\usepackage{listings}
\usepackage{xcolor}
\usepackage{geometry}
\usepackage{indentfirst}
\usepackage{setspace}

% Hyperref configuration
\hypersetup{
    colorlinks=true,
    linkcolor=blue,
    filecolor=magenta,
    urlcolor=cyan,
    pdftitle={Task 14: Two Paging Tables},
    pdfauthor={Your Name},
    pdfsubject={Memory Management in x86},
    pdfkeywords={paging, memory management, x86, two paging tables}
}

% Code listing styling
\definecolor{codegreen}{rgb}{0,0.6,0}
\definecolor{codegray}{rgb}{0.5,0.5,0.5}
\definecolor{codepurple}{rgb}{0.58,0,0.82}
\definecolor{backcolour}{rgb}{0.95,0.95,0.92}

\lstdefinestyle{mystyle}{
    backgroundcolor=\color{backcolour},
    commentstyle=\color{codegreen},
    keywordstyle=\color{magenta},
    numberstyle=\tiny\color{codegray},
    stringstyle=\color{codepurple},
    basicstyle=\ttfamily\footnotesize,
    breakatwhitespace=false,
    breaklines=true,
    captionpos=b,
    keepspaces=true,
    numbers=left,
    numbersep=5pt,
    showspaces=false,
    showstringspaces=false,
    showtabs=false,
    tabsize=2
}

\lstset{style=mystyle}

\geometry{margin=2.5cm}
\onehalfspacing

\begin{document}

\chapter{Task 14: Two Paging Tables}

\section{Task Description}
The goal of this task is to create two separate paging tables and demonstrate different mappings for the same pointer. Specifically:
\begin{itemize}
    \item One paging table will map a pointer (\texttt{ptr3}) to the first 4MB block of memory.
    \item The other paging table will map the same pointer (\texttt{ptr3}) to the second 4MB block of memory.
\end{itemize}

\section{Planned Implementation}
\begin{enumerate}
    \item \textbf{Create Two Paging Tables:}
    \begin{itemize}
        \item Define two separate root page tables (\texttt{root\_table1} and \texttt{root\_table2}).
        \item Align both tables to 4096 bytes using the \texttt{\_\_attribute\_\_((aligned(4096)))} directive.
    \end{itemize}
    \item \textbf{Set Up Mappings:}
    \begin{itemize}
        \item In \texttt{root\_table1}, map \texttt{ptr3} to the first 4MB block of memory.
        \item In \texttt{root\_table2}, map \texttt{ptr3} to the second 4MB block of memory.
    \end{itemize}
    \item \textbf{Switch Between Paging Tables:}
    \begin{itemize}
        \item Use the \texttt{set\_cr3()} function to switch between \texttt{root\_table1} and \texttt{root\_table2}.
    \end{itemize}
    \item \textbf{Demonstrate Behavior:}
    \begin{itemize}
        \item Access \texttt{ptr3} with \texttt{root\_table1} active and print its value.
        \item Switch to \texttt{root\_table2} and access \texttt{ptr3} again, printing its value.
    \end{itemize}
    \item \textbf{Verify Results:}
    \begin{itemize}
        \item Confirm that \texttt{ptr3} points to different physical memory blocks depending on the active paging table.
    \end{itemize}
\end{enumerate}

\section{Expected Outcome}
\begin{itemize}
    \item When \texttt{root\_table1} is active, \texttt{ptr3} should point to the first 4MB block of memory.
    \item When \texttt{root\_table2} is active, \texttt{ptr3} should point to the second 4MB block of memory.
    \item The values printed for \texttt{ptr3} should differ between the two paging tables.
\end{itemize}

\section{Implementation Details}
(To be filled after implementation)

\section{Challenges}
(To be filled after implementation)

\section{Final Outcome}
(To be filled after implementation)

\end{document}

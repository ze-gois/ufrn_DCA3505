\documentclass[a4paper,12pt]{report}

% Basic packages for LuaLaTeX with Unicode support
\usepackage{fontspec}
\usepackage[brazil]{babel}
\usepackage{graphicx}
\usepackage{hyperref}
\usepackage{amsmath}
\usepackage{amssymb}
\usepackage{listings}
\usepackage{xcolor}
\usepackage{geometry}
\usepackage{indentfirst}
\usepackage{setspace}

% Hyperref configuration
\hypersetup{
    colorlinks=true,
    linkcolor=blue,
    filecolor=magenta,
    urlcolor=cyan,
    pdftitle={Task 04: Effect of Reading on Page Table},
    pdfauthor={Your Name},
    pdfsubject={Memory Management in x86},
    pdfkeywords={paging, memory management, x86 architecture}
}

% Code listing styling
\definecolor{codegreen}{rgb}{0,0.6,0}
\definecolor{codegray}{rgb}{0.5,0.5,0.5}
\definecolor{codepurple}{rgb}{0.58,0,0.82}
\definecolor{backcolour}{rgb}{0.95,0.95,0.92}

\lstdefinestyle{mystyle}{
    backgroundcolor=\color{backcolour},
    commentstyle=\color{codegreen},
    keywordstyle=\color{magenta},
    numberstyle=\tiny\color{codegray},
    stringstyle=\color{codepurple},
    basicstyle=\ttfamily\footnotesize,
    breakatwhitespace=false,
    breaklines=true,
    captionpos=b,
    keepspaces=true,
    numbers=left,
    numbersep=5pt,
    showspaces=false,
    showstringspaces=false,
    showtabs=false,
    tabsize=2
}

\lstset{style=mystyle}

\geometry{margin=2.5cm}
\onehalfspacing

\begin{document}

\chapter{Task 04: Effect of Reading on Page Table}

\section{Task Description}
This task involves observing the effect of reading from a pointer on the page table. Specifically, we aim to determine how the page table entries are affected when a read operation is performed.

\section{Planned Implementation}
\begin{enumerate}
    \item \textbf{Setup:}
    \begin{itemize}
        \item Use the previous example where a pointer (\texttt{ptr2}) is mapped to a specific memory region.
        \item Ensure the page table is initialized and the pointer is valid.
    \end{itemize}
    \item \textbf{Read Operation:}
    \begin{itemize}
        \item Read the content of \texttt{ptr2}.
    \end{itemize}
    \item \textbf{Observation:}
    \begin{itemize}
        \item Print the value of the page table entry (\texttt{raiz[1]}) before and after the read operation.
    \end{itemize}
    \item \textbf{Expected Behavior:}
    \begin{itemize}
        \item Analyze whether the read operation causes any changes to the page table entry.
    \end{itemize}
\end{enumerate}

\section{Expected Outcome}
\begin{itemize}
    \item The page table entry (\texttt{raiz[1]}) should remain unchanged after the read operation, as reading memory typically does not modify page table entries.
\end{itemize}

\section{Implementation Details}
(To be filled after implementation)

\section{Challenges}
(To be filled after implementation)

\section{Final Outcome}
(To be filled after implementation)

\end{document}

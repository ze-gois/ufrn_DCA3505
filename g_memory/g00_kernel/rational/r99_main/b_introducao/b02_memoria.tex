\section{Gerenciamento de Memória na Arquitetura x86}
\label{sec:b02_memoria}

O sistema de gerenciamento de memória na arquitetura x86 evoluiu significativamente ao longo do tempo, implementando mecanismos cada vez mais sofisticados para atender às crescentes demandas por espaço de endereçamento, proteção e eficiência. Esta seção explora os principais componentes e conceitos do sistema de memória na arquitetura Intel.

\subsection{Modos de Endereçamento}
\label{subsec:enderecamento}

A arquitetura x86 possui diferentes modos de endereçamento que evoluíram com o tempo:

\begin{itemize}
    \item \textbf{Endereçamento em Modo Real}: Usa um esquema de segmentação simples onde o endereço físico é calculado multiplicando o valor do registro de segmento por 16 (deslocamento de 4 bits) e adicionando o deslocamento (offset). Limita-se a 1MB ($2^{20}$ bytes) de memória endereçável.

    \item \textbf{Endereçamento em Modo Protegido (32 bits)}: Suporta até 4GB ($2^{32}$ bytes) de memória física e implementa mecanismos de proteção através de segmentação avançada e paginação.

    \item \textbf{Endereçamento em Modo Longo (64 bits)}: Expande o espaço de endereçamento para teoricamente $2^{64}$ bytes, embora implementações atuais suportem 48 ou 57 bits de endereçamento. A segmentação é simplificada neste modo, com foco principal na paginação.
\end{itemize}

\subsection{Segmentação}
\label{subsec:segmentacao}

A segmentação é um dos mecanismos de gerenciamento de memória fundamentais na arquitetura x86, especialmente em modos Real e Protegido.

\subsubsection{Segmentação em Modo Protegido}

No modo protegido, a segmentação torna-se mais sofisticada:

\begin{itemize}
    \item \textbf{Descritores de Segmento}: Estruturas de 8 bytes que definem as propriedades de cada segmento, incluindo:
    \begin{itemize}
        \item Base (endereço inicial do segmento na memória física)
        \item Limite (tamanho do segmento)
        \item Tipo (código, dados, sistema)
        \item Privilégios (nível de privilégio exigido para acesso)
        \item Flags diversos (presença, granularidade, etc.)
    \end{itemize}

    \item \textbf{Tabelas de Descritores}: Os descritores são organizados em tabelas:
    \begin{itemize}
        \item \textbf{GDT (Global Descriptor Table)}: Tabela global de descritores compartilhada por todos os processos
        \item \textbf{LDT (Local Descriptor Table)}: Tabela local específica para um processo ou grupo de processos
    \end{itemize}

    \item \textbf{Seletores de Segmento}: Valores de 16 bits carregados nos registradores de segmento (CS, DS, SS, ES, FS, GS), composto por:
    \begin{itemize}
        \item Índice (bits 3-15): seleciona um descritor na tabela
        \item TI (bit 2): indica se o descritor está na GDT (0) ou LDT (1)
        \item RPL (bits 0-1): Nível de privilégio solicitado
    \end{itemize}
\end{itemize}

\subsubsection{Proteção via Segmentação}

A segmentação fornece vários mecanismos de proteção:

\begin{itemize}
    \item \textbf{Verificação de Limites}: Previne acesso fora dos limites definidos para o segmento
    \item \textbf{Níveis de Privilégio}: Restringe acesso com base nos níveis de privilégio (anéis 0-3)
    \item \textbf{Controle de Tipo}: Previne execução de dados ou escrita em código
    \item \textbf{Verificação de Presença}: Garante que o segmento está presente na memória física
\end{itemize}

\subsection{Paginação}
\label{subsec:paginacao_introducao}

A paginação é o mecanismo mais importante para gerenciamento de memória em sistemas operacionais modernos baseados em x86. Ao contrário da segmentação que opera com segmentos de tamanhos variáveis, a paginação divide a memória em unidades de tamanho fixo chamadas páginas (tipicamente 4KB).

\subsubsection{Características Principais}

\begin{itemize}
    \item \textbf{Tradução de Endereços}: Converte endereços lineares (pós-segmentação) em endereços físicos
    \item \textbf{Granularidade Fina}: Opera com unidades fixas (páginas), facilitando a alocação e gerenciamento
    \item \textbf{Memória Virtual}: Permite que o espaço de endereçamento virtual exceda a memória física disponível
    \item \textbf{Compartilhamento}: Múltiplos processos podem mapear a mesma página física
    \item \textbf{Proteção por Página}: Cada página pode ter atributos de proteção independentes
\end{itemize}

\subsubsection{Evolução da Paginação no x86}

\begin{itemize}
    \item \textbf{Paginação de 32 bits (80386)}: Esquema de dois níveis com diretórios e tabelas de páginas
    \item \textbf{Páginas de 4MB (Pentium)}: Extensão PSE (Page Size Extension) para suportar páginas maiores
    \item \textbf{PAE (Physical Address Extension)}: Estende o endereçamento físico para 36 bits (64GB) mesmo em modo 32 bits
    \item \textbf{Paginação em Modo Longo}: Esquema de quatro níveis (IA-32e) ou cinco níveis (mais recente), suportando endereçamento de até 57 bits
\end{itemize}

\subsubsection{TLB (Translation Lookaside Buffer)}

O TLB é uma memória cache especializada que armazena traduções recentes de endereços virtuais para físicos:

\begin{itemize}
    \item \textbf{Objetivo}: Acelerar o processo de tradução de endereços evitando múltiplos acessos à memória para consultar tabelas de páginas
    \item \textbf{Invalidação}: Quando tabelas de página são modificadas, entradas correspondentes no TLB devem ser invalidadas através de instruções específicas (como INVLPG) ou reinicialização completa do TLB
    \item \textbf{Tipos}: Processadores modernos geralmente têm TLBs separados para instruções (ITLB) e dados (DTLB), além de múltiplos níveis de TLB
\end{itemize}

\subsection{Registradores de Controle}
\label{subsec:reg_controle}

Os registradores de controle são fundamentais para o gerenciamento de memória no x86:

\begin{itemize}
    \item \textbf{CR0}: Controla modos de operação do processador
    \begin{itemize}
        \item Bit 0 (PE): Habilita o Modo Protegido quando setado
        \item Bit 31 (PG): Habilita a paginação quando setado
        \item Bit 16 (WP): Quando setado, impede o kernel de escrever em páginas somente-leitura
    \end{itemize}

    \item \textbf{CR2}: Contém o endereço linear que causou uma falha de página

    \item \textbf{CR3}: Contém o endereço físico da tabela de diretório de páginas (PDBR - Page Directory Base Register)

    \item \textbf{CR4}: Controla extensões da arquitetura
    \begin{itemize}
        \item Bit 4 (PSE): Habilita suporte a páginas de 4MB em modo 32 bits
        \item Bit 5 (PAE): Habilita Physical Address Extension
        \item Bit 7 (PGE): Habilita suporte a páginas globais
    \end{itemize}
\end{itemize}

\subsection{Falhas de Página}
\label{subsec:page_faults}

As falhas de página são exceções (interrupção 14) geradas pelo processador quando ocorre um problema durante a tradução de endereços:

\begin{itemize}
    \item \textbf{Causas comuns}:
    \begin{itemize}
        \item Página não presente na memória (bit P = 0)
        \item Violação de privilégios (tentativa de escrita em página somente-leitura)
        \item Acesso em nível de privilégio inadequado
    \end{itemize}

    \item \textbf{Informações disponíveis}:
    \begin{itemize}
        \item CR2: Contém o endereço linear que causou a falha
        \item Código de erro na pilha: Fornece detalhes sobre o tipo de falha
        \item Bit 0: Indica se a página não estava presente (0) ou houve violação de proteção (1)
        \item Bit 1: Indica se foi tentativa de escrita (1) ou leitura (0)
        \item Bit 2: Indica se o acesso foi em modo usuário (1) ou supervisor (0)
    \end{itemize}
\end{itemize}

\subsection{Usos Avançados da Paginação}
\label{subsec:usos_avancados}

A paginação permite implementar diversas funcionalidades avançadas em sistemas operacionais:

\begin{itemize}
    \item \textbf{Memória sob demanda}: Páginas são alocadas apenas quando necessárias

    \item \textbf{Copy-on-write}: Páginas são compartilhadas entre processos até que um deles tente escrever

    \item \textbf{Swapping}: Páginas pouco usadas podem ser movidas para armazenamento secundário

    \item \textbf{Isolamento de processos}: Cada processo tem seu próprio espaço de endereçamento virtual

    \item \textbf{Compartilhamento controlado}: Bibliotecas e recursos podem ser compartilhados entre processos

    \item \textbf{ASLR (Address Space Layout Randomization)}: Randomização da localização de áreas de memória para dificultar exploits
\end{itemize}

\subsection{Considerações de Desempenho}
\label{subsec:desempenho}

O sistema de memória é frequentemente um gargalo de desempenho, e várias técnicas são usadas para otimizá-lo:

\begin{itemize}
    \item \textbf{Hierarquia de cache}: L1, L2, L3 para reduzir latência de acesso

    \item \textbf{TLB}: Reduz o custo de tradução de endereços virtuais

    \item \textbf{Páginas grandes}: Reduzem o overhead do TLB ao cobrir mais memória com menos entradas

    \item \textbf{Prefetching}: Antecipação de acessos à memória para esconder latência

    \item \textbf{Alinhamento}: Dados alinhados são acessados mais eficientemente
\end{itemize}

O gerenciamento eficaz da memória é essencial para o desempenho do sistema como um todo, e compreender os mecanismos de baixo nível da arquitetura x86 permite implementações mais eficientes tanto em nível de sistema operacional quanto em aplicações de alta performance.

\section{Paginação na Arquitetura x86}
\label{sec:b04_paginacao}

A paginação é o mecanismo fundamental de gerenciamento de memória nos processadores modernos x86, permitindo a implementação de memória virtual, proteção entre processos e diversas outras funcionalidades essenciais para sistemas operacionais. Esta seção explora em detalhes os mecanismos de paginação na arquitetura Intel IA-32/x86-64, conforme documentado no Volume 3 do Intel® 64 and IA-32 Architectures Software Developer's Manual.

\subsection{Fundamentos da Paginação}
\label{subsec:fundamentos_paginacao}

A paginação divide a memória em unidades fixas chamadas páginas e proporciona um mapeamento flexível entre endereços lineares (virtuais) e físicos:

\begin{itemize}
    \item \textbf{Páginas}: Blocos de memória de tamanho fixo, tipicamente 4 KiB na arquitetura x86
    \item \textbf{Frames}: As unidades físicas correspondentes às páginas virtuais
    \item \textbf{MMU (Memory Management Unit)}: Hardware responsável pela tradução automática de endereços
    \item \textbf{Tabelas de Página}: Estruturas de dados que definem o mapeamento entre endereços virtuais e físicos
\end{itemize}

A paginação oferece diversas vantagens:
\begin{itemize}
    \item Isolamento entre processos
    \item Compartilhamento controlado de memória
    \item Implementação eficiente de memória virtual
    \item Proteção granular (permissões por página)
    \item Uso eficiente da memória física fragmentada
\end{itemize}

\subsection{Habilitando a Paginação}
\label{subsec:habilitar_paginacao}

Para habilitar a paginação em um processador x86, é necessário:

\begin{enumerate}
    \item Preparar as estruturas de tabelas de página na memória
    \item Carregar o endereço físico da tabela de diretório de páginas no registro CR3
    \item Habilitar a paginação setando o bit PG (bit 31) no registro CR0
    \item Opcionalmente, configurar recursos avançados via CR4 (PSE, PAE, etc.)
\end{enumerate}

Esta sequência deve ser cuidadosamente implementada, pois após habilitar a paginação, o processador imediatamente começa a traduzir todos os endereços. Se o mapeamento inicial não estiver corretamente configurado, o processador não conseguirá continuar a execução.

\subsubsection{Código de Exemplo para Habilitar Paginação}

Para habilitar paginação com páginas de 4MB no modo de 32 bits:

\begin{enumerate}
    \item Declarar e inicializar a tabela de diretório de páginas:
    \begin{itemize}
        \item Alinhar em limite de 4KB: \texttt{\_\_attribute\_\_((aligned(4096)))}
        \item Inicializar pelo menos a primeira entrada para mapear os primeiros 4MB: \texttt{(1 << 7) | (1 << 1) | (1 << 0)}
        \item O bit 7 (PSE) indica página de 4MB, bit 1 (R/W) permite escrita, bit 0 (P) marca como presente
    \end{itemize}

    \item Carregar o endereço da tabela em CR3: \texttt{set\_cr3((uint32\_t)\&tabela\_paginas);}

    \item Habilitar páginas de 4MB em CR4: \texttt{set\_cr4(get\_cr4() | (1 << 4));}

    \item Habilitar paginação em CR0: \texttt{set\_cr0(get\_cr0() | (1 << 31));}
\end{enumerate}

\subsection{Estruturas de Paginação de 32 bits}
\label{subsec:estruturas_paginacao_32bits}

Na arquitetura IA-32, a paginação pode ser configurada em diferentes modos:

\subsubsection{Paginação de 32 bits Padrão (Páginas de 4KB)}

Este modo utiliza um esquema de dois níveis:

\begin{itemize}
    \item \textbf{Diretório de Páginas (PD)}: Tabela de nível superior com 1024 entradas de 4 bytes
    \item \textbf{Tabelas de Página (PT)}: Tabelas de segundo nível, cada uma com 1024 entradas de 4 bytes
\end{itemize}

Um endereço linear de 32 bits é dividido em:
\begin{itemize}
    \item Bits 31-22 (10 bits): Índice no Diretório de Páginas
    \item Bits 21-12 (10 bits): Índice na Tabela de Páginas
    \item Bits 11-0 (12 bits): Deslocamento dentro da página de 4KB
\end{itemize}

\subsubsection{PSE (Page Size Extension) - Páginas de 4MB}

Quando o bit PSE em CR4 está habilitado:
\begin{itemize}
    \item O processador suporta páginas de 4MB
    \item O esquema é simplificado para um único nível
    \item Um endereço linear de 32 bits é dividido em:
    \begin{itemize}
        \item Bits 31-22 (10 bits): Índice no Diretório de Páginas
        \item Bits 21-0 (22 bits): Deslocamento dentro da página de 4MB
    \end{itemize}
\end{itemize}

\subsubsection{PAE (Physical Address Extension)}

O PAE estende o endereçamento físico para 36 bits (permitindo acessar até 64GB de RAM):
\begin{itemize}
    \item Esquema de três níveis
    \item Diretório de Diretórios de Página (PDPT) com 4 entradas de 8 bytes
    \item Diretórios de Páginas, cada um com 512 entradas de 8 bytes
    \item Tabelas de Página, cada uma com 512 entradas de 8 bytes
    \item Suporta páginas de 4KB e 2MB (com PSE)
\end{itemize}

\subsection{Formato das Entradas de Tabelas de Página}
\label{subsec:formato_entradas}

\subsubsection{Entrada de Diretório de Página (PDE) em Modo 32 bits}

Uma entrada de 32 bits no diretório de páginas possui os seguintes campos:

\begin{itemize}
    \item Bits 31-12: Endereço físico base da tabela de páginas (para páginas de 4KB) ou da página (para 4MB)
    \item Bit 7 (PS): Page Size - 0 para 4KB, 1 para 4MB
    \item Bit 6 (D): Dirty - Indica se a página foi escrita
    \item Bit 5 (A): Accessed - Indica se a página foi acessada
    \item Bit 4 (PCD): Page Cache Disable - Desabilita cache para esta página
    \item Bit 3 (PWT): Page Write Through - Configura cache para write-through
    \item Bit 2 (U/S): User/Supervisor - 0 para página de supervisor, 1 para página de usuário
    \item Bit 1 (R/W): Read/Write - 0 para somente leitura, 1 para leitura/escrita
    \item Bit 0 (P): Present - 1 indica página presente na memória física
\end{itemize}

\subsubsection{Entrada de Tabela de Página (PTE) em Modo 32 bits}

Similar à PDE, mas sempre referencia uma página de 4KB:

\begin{itemize}
    \item Bits 31-12: Endereço físico base da página de 4KB
    \item Os bits de controle (A, D, PCD, PWT, U/S, R/W, P) têm o mesmo significado que na PDE
\end{itemize}

\subsection{Tradução de Endereços}
\label{subsec:traducao_enderecos}

O processo de tradução de endereços usando paginação de 32 bits acontece da seguinte forma:

\begin{enumerate}
    \item O processador extrai os bits 31-22 do endereço linear para obter o índice no diretório de páginas
    \item Multiplica este índice por 4 (tamanho de cada entrada) e soma ao endereço base do diretório (armazenado em CR3)
    \item Lê a entrada do diretório (PDE) neste endereço
    \item Verifica se a PDE está presente (bit P) e se tem as permissões apropriadas
    \item Se a PDE indica uma página de 4MB (bit PS=1):
        \begin{itemize}
            \item O endereço físico é formado combinando os bits 31-22 da PDE com os bits 21-0 do endereço linear
        \end{itemize}
    \item Se a PDE indica uma tabela de páginas (bit PS=0):
        \begin{itemize}
            \item Extrai os bits 21-12 do endereço linear para obter o índice na tabela de páginas
            \item Multiplica este índice por 4 e soma ao endereço base da tabela (obtido da PDE)
            \item Lê a entrada da tabela (PTE) neste endereço
            \item Verifica se a PTE está presente e tem as permissões apropriadas
            \item O endereço físico é formado combinando os bits 31-12 da PTE com os bits 11-0 do endereço linear
        \end{itemize}
\end{enumerate}

\subsection{Falhas de Página}
\label{subsec:falhas_pagina}

Uma falha de página (Page Fault) ocorre quando há um problema durante a tradução de endereço:

\begin{itemize}
    \item O processador gera uma exceção de falha de página (interrupção 14)
    \item O código de erro empilhado fornece informações sobre a causa:
        \begin{itemize}
            \item Bit 0: 0 se a página não estava presente, 1 se foi violação de proteção
            \item Bit 1: 0 se foi acesso de leitura, 1 se foi acesso de escrita
            \item Bit 2: 0 se foi acesso em modo kernel, 1 se foi acesso em modo usuário
            \item Bit 3: 1 se a falha envolveu bits reservados
            \item Bit 4: 1 se a falha ocorreu durante uma busca de instrução
        \end{itemize}
    \item O endereço que causou a falha é armazenado no registro CR2
\end{itemize}

O tratador de falhas de página pode tomar diferentes ações:
\begin{itemize}
    \item Carregar a página da memória de troca (implementando memória virtual)
    \item Alocar uma nova página para alocação sob demanda
    \item Criar uma cópia da página para implementar copy-on-write
    \item Terminar o processo se o acesso for inválido (segmentation fault)
\end{itemize}

\subsection{Translation Lookaside Buffer (TLB)}
\label{subsec:tlb}

O TLB é uma cache de traduções de endereços que acelera o processo de paginação:

\begin{itemize}
    \item Armazena as traduções mais recentes de endereços lineares para físicos
    \item É consultado antes das tabelas de página na memória
    \item Se uma tradução está presente no TLB (hit), a consulta às tabelas é evitada
    \item É gerenciado automaticamente pelo hardware, mas requer intervenção do software em alguns casos
\end{itemize}

Quando as tabelas de página são modificadas, o TLB deve ser invalidado:
\begin{itemize}
    \item \textbf{Invalidação específica}: A instrução INVLPG invalida uma entrada específica do TLB
    \item \textbf{Invalidação global}: Recarregar CR3 invalida todo o TLB (exceto entradas marcadas como globais)
\end{itemize}

\subsection{Flags e Bits de Controle}
\label{subsec:flags_controle}

Além dos bits nas entradas das tabelas, vários bits nos registradores de controle afetam o comportamento da paginação:

\begin{itemize}
    \item \textbf{CR0.WP} (Write Protect, bit 16):
        \begin{itemize}
            \item Quando 0, o kernel pode escrever em páginas somente-leitura
            \item Quando 1, as proteções de escrita são aplicadas mesmo em modo kernel
        \end{itemize}

    \item \textbf{CR4.PSE} (Page Size Extensions, bit 4):
        \begin{itemize}
            \item Quando 1, habilita suporte a páginas de 4MB em modo 32 bits
        \end{itemize}

    \item \textbf{CR4.PGE} (Page Global Enable, bit 7):
        \begin{itemize}
            \item Quando 1, permite marcar páginas como globais (não são invalidadas no TLB durante trocas de CR3)
        \end{itemize}

    \item \textbf{CR4.PAE} (Physical Address Extension, bit 5):
        \begin{itemize}
            \item Quando 1, habilita PAE para suportar endereçamento físico estendido
        \end{itemize}
\end{itemize}

\subsection{Técnicas Avançadas}
\label{subsec:tecnicas_avancadas}

\subsubsection{Mapeamentos Múltiplos}

É possível mapear o mesmo frame físico em múltiplos endereços virtuais:
\begin{itemize}
    \item Útil para compartilhamento de memória entre processos
    \item Permite ter diferentes permissões para o mesmo conteúdo físico
    \item Facilita operações copy-on-write
\end{itemize}

\subsubsection{Páginas de Kernel Mapeadas em Todo Espaço de Endereçamento}

Sistemas operacionais modernos normalmente:
\begin{itemize}
    \item Dividem o espaço de endereço virtual em porções de usuário e kernel
    \item Mapeiam o código e dados do kernel em todos os espaços de endereço
    \item Marcam essas páginas como globais para evitar invalidações do TLB durante trocas de contexto
\end{itemize}

\subsubsection{Paginação Aninhada e Virtualização}

Em ambientes virtualizados:
\begin{itemize}
    \item O hardware moderno suporta paginação aninhada (NPT/EPT)
    \item Permite que máquinas virtuais usem paginação sem overhead excessivo
    \item Adiciona uma camada extra de tradução: GVA → GPA → HPA (Guest Virtual → Guest Physical → Host Physical)
\end{itemize}

\subsection{Considerações de Implementação Prática}
\label{subsec:implementacao_pratica}

Ao implementar paginação em um ambiente baremetal ou kernel simples:

\begin{itemize}
    \item \textbf{Identity mapping inicial}: Mapeie os primeiros MB da memória física para endereços virtuais idênticos
    \item \textbf{Mapeamento do código}: Garanta que o código do kernel permaneça acessível após habilitar paginação
    \item \textbf{Alinhamento}: Todas as estruturas de paginação devem estar alinhadas em limites de página (4KB)
    \item \textbf{Tratamento de interrupções}: Configure o tratador de falhas de página antes de experimentar com mapeamentos
    \item \textbf{Inicialização cuidadosa}: A sequência de inicialização deve garantir que a próxima instrução após habilitar paginação seja acessível
\end{itemize}

A paginação é uma das funcionalidades mais poderosas e complexas da arquitetura x86. Sua compreensão detalhada é fundamental para o desenvolvimento de sistemas operacionais e para a implementação de funcionalidades avançadas de gerenciamento de memória.

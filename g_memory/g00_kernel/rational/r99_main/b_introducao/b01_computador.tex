\section{Arquitetura IA-32/x86-64}
\label{sec:b01_computador}

A arquitetura Intel x86 representa um dos conjuntos de instruções mais importantes e duradouros na história da computação moderna. Desde sua introdução com o processador 8086 em 1978, a arquitetura evoluiu substancialmente, mantendo compatibilidade retroativa enquanto incorporava novas capacidades e extensões. Esta seção explora os fundamentos da arquitetura x86, com foco nas suas características relevantes para o gerenciamento de memória.

\subsection{Evolução da Arquitetura x86}
\label{subsec:evolucao}

A arquitetura x86 passou por transformações significativas ao longo das décadas:

\begin{itemize}
    \item \textbf{8086/8088 (1978)}: Processador de 16 bits com barramento de endereço de 20 bits, permitindo acessar 1MB de memória através de um sistema de segmentação.

    \item \textbf{80286 (1982)}: Introduziu o modo protegido, expandindo o espaço de endereçamento para 16MB e adicionando proteção de memória mais robusta.

    \item \textbf{80386 (1985)}: Primeiro processador x86 de 32 bits (IA-32), com barramento de endereço de 32 bits permitindo acessar 4GB de memória física. Introduziu a paginação, fundamental para sistemas operacionais modernos.

    \item \textbf{80486 e Pentium (1989-1993)}: Refinamentos da arquitetura IA-32 com melhorias de desempenho, incluindo cache L1 integrado.

    \item \textbf{AMD64/Intel 64 (2003)}: Extensão de 64 bits da arquitetura x86, permitindo endereçar teoricamente até $2^{64}$ bytes de memória (limitado na prática a 48 ou 57 bits de endereçamento físico nos processadores atuais).
\end{itemize}

\subsection{Modos de Operação}
\label{subsec:modos}

A arquitetura x86 moderna suporta múltiplos modos de operação:

\begin{itemize}
    \item \textbf{Modo Real}: Emula o ambiente do 8086 original, com endereçamento segmentado de 16 bits. Limitado a 1MB de memória sem proteção entre programas.

    \item \textbf{Modo Protegido}: Introduzido no 80286, oferece proteção de memória através de segmentação e, a partir do 80386, paginação. Este é o modo utilizado pelos sistemas operacionais de 32 bits.

    \item \textbf{Modo Virtual 8086}: Permite executar programas de modo real dentro do ambiente protegido de multitarefa.

    \item \textbf{Modo Longo (64 bits)}: Disponível em processadores AMD64/Intel 64, oferece espaço de endereçamento de 64 bits, registradores estendidos e outros recursos avançados.

    \item \textbf{SMM (System Management Mode)}: Modo especial usado para funções de gerenciamento do sistema, como controle de energia.
\end{itemize}

\subsection{Registradores do Processador}
\label{subsec:registradores}

Os registradores são componentes fundamentais para o funcionamento da CPU. Na arquitetura x86, dividem-se em várias categorias:

\begin{itemize}
    \item \textbf{Registradores de Propósito Geral}: Em modo de 32 bits, incluem EAX, EBX, ECX, EDX, ESI, EDI, EBP, ESP. Em modo de 64 bits, são estendidos para RAX, RBX, etc., e novos registradores (R8-R15) são adicionados.

    \item \textbf{Registradores de Segmento}: CS (Code Segment), DS (Data Segment), SS (Stack Segment), ES, FS, GS. Fundamentais para o modelo de segmentação.

    \item \textbf{Registradores de Controle}: CR0-CR4 (em IA-32) e CR0-CR8 (em x86-64), controlam aspectos do processador como paginação, cache e modos de operação.

    \item \textbf{Registradores de Sistema}: GDTR, LDTR, IDTR e TR, apontam para estruturas do sistema como GDT (Global Descriptor Table), LDT (Local Descriptor Table), IDT (Interrupt Descriptor Table) e TSS (Task State Segment).

    \item \textbf{Registradores de Depuração}: DR0-DR7, utilizados para depuração de hardware.

    \item \textbf{Registradores de MMX, SSE, AVX}: Suportam instruções SIMD (Single Instruction Multiple Data) para processamento vetorial.
\end{itemize}

\subsection{Proteção e Privilégios}
\label{subsec:protecao}

A arquitetura x86 implementa um sistema de proteção baseado em anéis de privilégios, numerados de 0 (mais privilegiado) a 3 (menos privilegiado):

\begin{itemize}
    \item \textbf{Anel 0}: Reservado para o núcleo do sistema operacional (kernel). Tem acesso irrestrito ao hardware e pode executar todas as instruções.

    \item \textbf{Anel 1 e 2}: Raramente utilizados nos sistemas operacionais modernos, foram projetados para drivers de dispositivos e serviços do sistema.

    \item \textbf{Anel 3}: Onde executam as aplicações do usuário, com acesso restrito ao hardware e impossibilitadas de executar instruções privilegiadas.
\end{itemize}

Esta hierarquia de privilégios é fundamental para a segurança do sistema, impedindo que programas de usuário acessem diretamente o hardware ou interfiram com outros programas.

\subsection{Gerenciamento de Interrupções e Exceções}
\label{subsec:interrupcoes}

O processador x86 possui um sistema sofisticado para lidar com interrupções e exceções:

\begin{itemize}
    \item \textbf{Interrupções}: Podem ser geradas externamente (hardware) ou internamente (software via instrução INT).

    \item \textbf{Exceções}: Ocorrem quando o processador detecta uma condição anormal durante a execução de uma instrução, como divisão por zero ou falha de página.

    \item \textbf{IDT (Interrupt Descriptor Table)}: Tabela que associa vetores de interrupção/exceção a rotinas de tratamento.
\end{itemize}

Entre as exceções mais relevantes para o gerenciamento de memória estão:
\begin{itemize}
    \item \textbf{Page Fault (14)}: Ocorre quando um programa tenta acessar uma página não presente na memória física ou sem permissões adequadas.

    \item \textbf{General Protection Fault (13)}: Ocorre por violações de proteção, como tentar executar uma instrução privilegiada em modo usuário ou acessar memória além dos limites do segmento.
\end{itemize}

\subsection{Conceitos de Memória Baremetal}
\label{subsec:baremetal}

Programação baremetal refere-se ao desenvolvimento de software que executa diretamente no hardware sem um sistema operacional subjacente. No contexto da arquitetura x86, isso envolve:

\begin{itemize}
    \item \textbf{Inicialização}: O processador inicia em Modo Real e deve ser configurado para Modo Protegido ou Longo pelo código de inicialização.

    \item \textbf{GDT e IDT}: O código de inicialização deve configurar estas estruturas fundamentais.

    \item \textbf{Paginação}: Se desejada, deve ser habilitada explicitamente.

    \item \textbf{Interação direta com hardware}: O código tem acesso direto a todas as instruções e portas de E/S.
\end{itemize}

A programação baremetal é valiosa educacionalmente pois expõe os mecanismos internos do processador normalmente abstraídos pelos sistemas operacionais, permitindo uma compreensão mais profunda da arquitetura.

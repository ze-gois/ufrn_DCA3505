\documentclass[a4paper,12pt]{report}

% Basic packages for LuaLaTeX with Unicode support
% No need for inputenc with LuaLaTeX as it supports UTF-8 natively
\usepackage{fontspec}
\usepackage[brazil]{babel}
\usepackage{graphicx}
\usepackage{hyperref}
\usepackage{url}
\usepackage{amsmath}
\usepackage{amssymb}
\usepackage{listings}  % Regular listings is fine with LuaLaTeX
\usepackage{xcolor}
\usepackage{geometry}
\usepackage{indentfirst}
\usepackage{setspace}
\usepackage{cleveref}  % For enhanced cross-referencing
\usepackage{caption}   % For better caption handling
\usepackage{float}     % For improved figure/table placement
% \usepackage{luatextra} % Extra features for LuaLaTeX

% Hyperref configuration for better PDF output
\hypersetup{
    colorlinks=true,
    linkcolor=blue,
    filecolor=magenta,
    urlcolor=cyan,
    citecolor=green,
    pdftitle={Estudo acerca de memória no x86},
    pdfauthor={José Henrique Targino Dias Gois},
    pdfsubject={Memória no x86},
    pdfkeywords={sistemas operacionais, sistemas de memória, intel x86}
}

% Code listing styling
\definecolor{codegreen}{rgb}{0,0.6,0}
\definecolor{codegray}{rgb}{0.5,0.5,0.5}
\definecolor{codepurple}{rgb}{0.58,0,0.82}
\definecolor{backcolour}{rgb}{0.95,0.95,0.92}

\lstdefinestyle{mystyle}{
    backgroundcolor=\color{backcolour},
    commentstyle=\color{codegreen},
    keywordstyle=\color{magenta},
    numberstyle=\tiny\color{codegray},
    stringstyle=\color{codepurple},
    basicstyle=\ttfamily\footnotesize,
    breakatwhitespace=false,
    breaklines=true,
    captionpos=b,
    keepspaces=true,
    numbers=left,
    numbersep=5pt,
    showspaces=false,
    showstringspaces=false,
    showtabs=false,
    tabsize=2,
    extendedchars=true,
    texcl=true,
    inputencoding=utf8
}

\lstset{style=mystyle}

\geometry{margin=2.5cm}
\onehalfspacing

% Custom command for easier section referencing
\newcommand{\secref}[1]{Seção~\ref{#1}}
\newcommand{\figref}[1]{Figura~\ref{#1}}
\newcommand{\tabref}[1]{Tabela~\ref{#1}}
\newcommand{\eqnref}[1]{Equação~\ref{#1}}
\newcommand{\chapref}[1]{Capítulo~\ref{#1}}

% Document start
\begin{document}

% Cover page
\begin{titlepage}
    \centering
    \vspace*{1cm}

    \textbf{\LARGE Universidade Federal do Rio Grande do Norte}\\
    \textbf{\large Departamento de Engenharia de Computação e Automação}\\
    \textbf{\large DCA0121 - Sistemas Operacionais}\\
    \vspace{1.5cm}

    \textbf{\Huge Lockdep}\\
    \vspace{0.5cm}
    \textbf{\Large Detecção de Deadlocks e Violação da Ordem de Aquisição de Travas}\\
    \vspace{1.5cm}

    \begin{figure}[h]
        \centering
        % Add UFRN logo here if available
        % \includegraphics[width=0.4\textwidth]{ufrn_logo.png}
    \end{figure}

    \vspace{1.5cm}

    \begin{flushright}
        \textbf{Desenvolvido por:}\\
        Nome do Aluno\\
        Matrícula: 000000000
    \end{flushright}

    \vfill

    \textbf{\large Natal - RN}\\
    \textbf{\large \today}

\end{titlepage}


% Abstracts
\begin{abstract}
Este trabalho apresenta a implementação de um sistema de detecção de deadlocks e violação da ordem de aquisição de travas, inspirado no lockdep do kernel Linux. O sistema monitora operações de mutex em programas multithreaded, construindo um grafo de dependências entre travas e detectando potenciais situações de deadlock antes que ocorram. Utilizamos duas abordagens complementares: (1) detecção de ciclos no grafo de dependências, que indica um deadlock potencial, e (2) verificação da consistência na ordem de aquisição de travas entre diferentes threads. O sistema foi implementado através de duas técnicas distintas: uma biblioteca compartilhada que utiliza interposição de funções para interceptar chamadas à API pthread, e uma ferramenta baseada em ptrace capaz de analisar processos em execução, incluindo aqueles já em estado de deadlock. Ambas as implementações compartilham uma biblioteca modular de análise de grafos. Os testes realizados demonstram a eficácia do sistema na detecção de deadlocks clássicos como o problema AB-BA, onde duas threads tentam adquirir os mesmos mutexes em ordens diferentes, com cada abordagem apresentando vantagens específicas para diferentes cenários. Este trabalho contribui para o desenvolvimento de ferramentas que aumentam a robustez de sistemas concorrentes, ajudando programadores a identificar problemas de sincronização difíceis de detectar.

\vspace{0.5cm}

\noindent\textbf{Palavras-chave:} deadlock, mutex, concorrência, detecção de erros, grafo de dependência, sistemas operacionais, ptrace, interposição de funções.
\end{abstract}

\begin{abstract}
\selectlanguage{american}
This work presents the implementation of a system for detecting deadlocks and lock acquisition order violations, inspired by the Linux kernel's lockdep. The system monitors mutex operations in multithreaded programs, building a dependency graph between locks and detecting potential deadlock situations before they occur. We use two complementary approaches: (1) cycle detection in the dependency graph, which indicates a potential deadlock, and (2) verification of consistency in the order of lock acquisition across different threads. The system was implemented using two distinct techniques: a shared library that uses function interposition to intercept calls to the pthread API, and a ptrace-based tool capable of analyzing running processes, including those already in a deadlock state. Both implementations share a modular graph analysis library. Tests performed demonstrate the effectiveness of the system in detecting classic deadlocks such as the AB-BA problem, where two threads try to acquire the same mutexes in different orders, with each approach offering specific advantages for different scenarios. This work contributes to the development of tools that increase the robustness of concurrent systems, helping programmers identify synchronization problems that are difficult to detect.

\vspace{0.5cm}

\noindent\textbf{Keywords:} deadlock, mutex, concurrency, error detection, dependency graph, operating systems, ptrace, function interposition.
\selectlanguage{brazil}
\end{abstract}


% Custom TOC as defined in a04_toc.tex
\pagenumbering{roman}

\tableofcontents
\clearpage

\listoffigures
\clearpage

\listoftables
\clearpage

\pagenumbering{arabic}


% Introduction
\chapter{Introdução}\label{chap:intro}
\section{Arquitetura IA-32/x86-64}
\label{sec:b01_computador}

A arquitetura Intel x86 representa um dos conjuntos de instruções mais importantes e duradouros na história da computação moderna. Desde sua introdução com o processador 8086 em 1978, a arquitetura evoluiu substancialmente, mantendo compatibilidade retroativa enquanto incorporava novas capacidades e extensões. Esta seção explora os fundamentos da arquitetura x86, com foco nas suas características relevantes para o gerenciamento de memória.

\subsection{Evolução da Arquitetura x86}
\label{subsec:evolucao}

A arquitetura x86 passou por transformações significativas ao longo das décadas:

\begin{itemize}
    \item \textbf{8086/8088 (1978)}: Processador de 16 bits com barramento de endereço de 20 bits, permitindo acessar 1MB de memória através de um sistema de segmentação.

    \item \textbf{80286 (1982)}: Introduziu o modo protegido, expandindo o espaço de endereçamento para 16MB e adicionando proteção de memória mais robusta.

    \item \textbf{80386 (1985)}: Primeiro processador x86 de 32 bits (IA-32), com barramento de endereço de 32 bits permitindo acessar 4GB de memória física. Introduziu a paginação, fundamental para sistemas operacionais modernos.

    \item \textbf{80486 e Pentium (1989-1993)}: Refinamentos da arquitetura IA-32 com melhorias de desempenho, incluindo cache L1 integrado.

    \item \textbf{AMD64/Intel 64 (2003)}: Extensão de 64 bits da arquitetura x86, permitindo endereçar teoricamente até $2^{64}$ bytes de memória (limitado na prática a 48 ou 57 bits de endereçamento físico nos processadores atuais).
\end{itemize}

\subsection{Modos de Operação}
\label{subsec:modos}

A arquitetura x86 moderna suporta múltiplos modos de operação:

\begin{itemize}
    \item \textbf{Modo Real}: Emula o ambiente do 8086 original, com endereçamento segmentado de 16 bits. Limitado a 1MB de memória sem proteção entre programas.

    \item \textbf{Modo Protegido}: Introduzido no 80286, oferece proteção de memória através de segmentação e, a partir do 80386, paginação. Este é o modo utilizado pelos sistemas operacionais de 32 bits.

    \item \textbf{Modo Virtual 8086}: Permite executar programas de modo real dentro do ambiente protegido de multitarefa.

    \item \textbf{Modo Longo (64 bits)}: Disponível em processadores AMD64/Intel 64, oferece espaço de endereçamento de 64 bits, registradores estendidos e outros recursos avançados.

    \item \textbf{SMM (System Management Mode)}: Modo especial usado para funções de gerenciamento do sistema, como controle de energia.
\end{itemize}

\subsection{Registradores do Processador}
\label{subsec:registradores}

Os registradores são componentes fundamentais para o funcionamento da CPU. Na arquitetura x86, dividem-se em várias categorias:

\begin{itemize}
    \item \textbf{Registradores de Propósito Geral}: Em modo de 32 bits, incluem EAX, EBX, ECX, EDX, ESI, EDI, EBP, ESP. Em modo de 64 bits, são estendidos para RAX, RBX, etc., e novos registradores (R8-R15) são adicionados.

    \item \textbf{Registradores de Segmento}: CS (Code Segment), DS (Data Segment), SS (Stack Segment), ES, FS, GS. Fundamentais para o modelo de segmentação.

    \item \textbf{Registradores de Controle}: CR0-CR4 (em IA-32) e CR0-CR8 (em x86-64), controlam aspectos do processador como paginação, cache e modos de operação.

    \item \textbf{Registradores de Sistema}: GDTR, LDTR, IDTR e TR, apontam para estruturas do sistema como GDT (Global Descriptor Table), LDT (Local Descriptor Table), IDT (Interrupt Descriptor Table) e TSS (Task State Segment).

    \item \textbf{Registradores de Depuração}: DR0-DR7, utilizados para depuração de hardware.

    \item \textbf{Registradores de MMX, SSE, AVX}: Suportam instruções SIMD (Single Instruction Multiple Data) para processamento vetorial.
\end{itemize}

\subsection{Proteção e Privilégios}
\label{subsec:protecao}

A arquitetura x86 implementa um sistema de proteção baseado em anéis de privilégios, numerados de 0 (mais privilegiado) a 3 (menos privilegiado):

\begin{itemize}
    \item \textbf{Anel 0}: Reservado para o núcleo do sistema operacional (kernel). Tem acesso irrestrito ao hardware e pode executar todas as instruções.

    \item \textbf{Anel 1 e 2}: Raramente utilizados nos sistemas operacionais modernos, foram projetados para drivers de dispositivos e serviços do sistema.

    \item \textbf{Anel 3}: Onde executam as aplicações do usuário, com acesso restrito ao hardware e impossibilitadas de executar instruções privilegiadas.
\end{itemize}

Esta hierarquia de privilégios é fundamental para a segurança do sistema, impedindo que programas de usuário acessem diretamente o hardware ou interfiram com outros programas.

\subsection{Gerenciamento de Interrupções e Exceções}
\label{subsec:interrupcoes}

O processador x86 possui um sistema sofisticado para lidar com interrupções e exceções:

\begin{itemize}
    \item \textbf{Interrupções}: Podem ser geradas externamente (hardware) ou internamente (software via instrução INT).

    \item \textbf{Exceções}: Ocorrem quando o processador detecta uma condição anormal durante a execução de uma instrução, como divisão por zero ou falha de página.

    \item \textbf{IDT (Interrupt Descriptor Table)}: Tabela que associa vetores de interrupção/exceção a rotinas de tratamento.
\end{itemize}

Entre as exceções mais relevantes para o gerenciamento de memória estão:
\begin{itemize}
    \item \textbf{Page Fault (14)}: Ocorre quando um programa tenta acessar uma página não presente na memória física ou sem permissões adequadas.

    \item \textbf{General Protection Fault (13)}: Ocorre por violações de proteção, como tentar executar uma instrução privilegiada em modo usuário ou acessar memória além dos limites do segmento.
\end{itemize}

\subsection{Conceitos de Memória Baremetal}
\label{subsec:baremetal}

Programação baremetal refere-se ao desenvolvimento de software que executa diretamente no hardware sem um sistema operacional subjacente. No contexto da arquitetura x86, isso envolve:

\begin{itemize}
    \item \textbf{Inicialização}: O processador inicia em Modo Real e deve ser configurado para Modo Protegido ou Longo pelo código de inicialização.

    \item \textbf{GDT e IDT}: O código de inicialização deve configurar estas estruturas fundamentais.

    \item \textbf{Paginação}: Se desejada, deve ser habilitada explicitamente.

    \item \textbf{Interação direta com hardware}: O código tem acesso direto a todas as instruções e portas de E/S.
\end{itemize}

A programação baremetal é valiosa educacionalmente pois expõe os mecanismos internos do processador normalmente abstraídos pelos sistemas operacionais, permitindo uma compreensão mais profunda da arquitetura.

\section{Gerenciamento de Memória na Arquitetura x86}
\label{sec:b02_memoria}

O sistema de gerenciamento de memória na arquitetura x86 evoluiu significativamente ao longo do tempo, implementando mecanismos cada vez mais sofisticados para atender às crescentes demandas por espaço de endereçamento, proteção e eficiência. Esta seção explora os principais componentes e conceitos do sistema de memória na arquitetura Intel.

\subsection{Modos de Endereçamento}
\label{subsec:enderecamento}

A arquitetura x86 possui diferentes modos de endereçamento que evoluíram com o tempo:

\begin{itemize}
    \item \textbf{Endereçamento em Modo Real}: Usa um esquema de segmentação simples onde o endereço físico é calculado multiplicando o valor do registro de segmento por 16 (deslocamento de 4 bits) e adicionando o deslocamento (offset). Limita-se a 1MB ($2^{20}$ bytes) de memória endereçável.

    \item \textbf{Endereçamento em Modo Protegido (32 bits)}: Suporta até 4GB ($2^{32}$ bytes) de memória física e implementa mecanismos de proteção através de segmentação avançada e paginação.

    \item \textbf{Endereçamento em Modo Longo (64 bits)}: Expande o espaço de endereçamento para teoricamente $2^{64}$ bytes, embora implementações atuais suportem 48 ou 57 bits de endereçamento. A segmentação é simplificada neste modo, com foco principal na paginação.
\end{itemize}

\subsection{Segmentação}
\label{subsec:segmentacao}

A segmentação é um dos mecanismos de gerenciamento de memória fundamentais na arquitetura x86, especialmente em modos Real e Protegido.

\subsubsection{Segmentação em Modo Protegido}

No modo protegido, a segmentação torna-se mais sofisticada:

\begin{itemize}
    \item \textbf{Descritores de Segmento}: Estruturas de 8 bytes que definem as propriedades de cada segmento, incluindo:
    \begin{itemize}
        \item Base (endereço inicial do segmento na memória física)
        \item Limite (tamanho do segmento)
        \item Tipo (código, dados, sistema)
        \item Privilégios (nível de privilégio exigido para acesso)
        \item Flags diversos (presença, granularidade, etc.)
    \end{itemize}

    \item \textbf{Tabelas de Descritores}: Os descritores são organizados em tabelas:
    \begin{itemize}
        \item \textbf{GDT (Global Descriptor Table)}: Tabela global de descritores compartilhada por todos os processos
        \item \textbf{LDT (Local Descriptor Table)}: Tabela local específica para um processo ou grupo de processos
    \end{itemize}

    \item \textbf{Seletores de Segmento}: Valores de 16 bits carregados nos registradores de segmento (CS, DS, SS, ES, FS, GS), composto por:
    \begin{itemize}
        \item Índice (bits 3-15): seleciona um descritor na tabela
        \item TI (bit 2): indica se o descritor está na GDT (0) ou LDT (1)
        \item RPL (bits 0-1): Nível de privilégio solicitado
    \end{itemize}
\end{itemize}

\subsubsection{Proteção via Segmentação}

A segmentação fornece vários mecanismos de proteção:

\begin{itemize}
    \item \textbf{Verificação de Limites}: Previne acesso fora dos limites definidos para o segmento
    \item \textbf{Níveis de Privilégio}: Restringe acesso com base nos níveis de privilégio (anéis 0-3)
    \item \textbf{Controle de Tipo}: Previne execução de dados ou escrita em código
    \item \textbf{Verificação de Presença}: Garante que o segmento está presente na memória física
\end{itemize}

\subsection{Paginação}
\label{subsec:paginacao_introducao}

A paginação é o mecanismo mais importante para gerenciamento de memória em sistemas operacionais modernos baseados em x86. Ao contrário da segmentação que opera com segmentos de tamanhos variáveis, a paginação divide a memória em unidades de tamanho fixo chamadas páginas (tipicamente 4KB).

\subsubsection{Características Principais}

\begin{itemize}
    \item \textbf{Tradução de Endereços}: Converte endereços lineares (pós-segmentação) em endereços físicos
    \item \textbf{Granularidade Fina}: Opera com unidades fixas (páginas), facilitando a alocação e gerenciamento
    \item \textbf{Memória Virtual}: Permite que o espaço de endereçamento virtual exceda a memória física disponível
    \item \textbf{Compartilhamento}: Múltiplos processos podem mapear a mesma página física
    \item \textbf{Proteção por Página}: Cada página pode ter atributos de proteção independentes
\end{itemize}

\subsubsection{Evolução da Paginação no x86}

\begin{itemize}
    \item \textbf{Paginação de 32 bits (80386)}: Esquema de dois níveis com diretórios e tabelas de páginas
    \item \textbf{Páginas de 4MB (Pentium)}: Extensão PSE (Page Size Extension) para suportar páginas maiores
    \item \textbf{PAE (Physical Address Extension)}: Estende o endereçamento físico para 36 bits (64GB) mesmo em modo 32 bits
    \item \textbf{Paginação em Modo Longo}: Esquema de quatro níveis (IA-32e) ou cinco níveis (mais recente), suportando endereçamento de até 57 bits
\end{itemize}

\subsubsection{TLB (Translation Lookaside Buffer)}

O TLB é uma memória cache especializada que armazena traduções recentes de endereços virtuais para físicos:

\begin{itemize}
    \item \textbf{Objetivo}: Acelerar o processo de tradução de endereços evitando múltiplos acessos à memória para consultar tabelas de páginas
    \item \textbf{Invalidação}: Quando tabelas de página são modificadas, entradas correspondentes no TLB devem ser invalidadas através de instruções específicas (como INVLPG) ou reinicialização completa do TLB
    \item \textbf{Tipos}: Processadores modernos geralmente têm TLBs separados para instruções (ITLB) e dados (DTLB), além de múltiplos níveis de TLB
\end{itemize}

\subsection{Registradores de Controle}
\label{subsec:reg_controle}

Os registradores de controle são fundamentais para o gerenciamento de memória no x86:

\begin{itemize}
    \item \textbf{CR0}: Controla modos de operação do processador
    \begin{itemize}
        \item Bit 0 (PE): Habilita o Modo Protegido quando setado
        \item Bit 31 (PG): Habilita a paginação quando setado
        \item Bit 16 (WP): Quando setado, impede o kernel de escrever em páginas somente-leitura
    \end{itemize}

    \item \textbf{CR2}: Contém o endereço linear que causou uma falha de página

    \item \textbf{CR3}: Contém o endereço físico da tabela de diretório de páginas (PDBR - Page Directory Base Register)

    \item \textbf{CR4}: Controla extensões da arquitetura
    \begin{itemize}
        \item Bit 4 (PSE): Habilita suporte a páginas de 4MB em modo 32 bits
        \item Bit 5 (PAE): Habilita Physical Address Extension
        \item Bit 7 (PGE): Habilita suporte a páginas globais
    \end{itemize}
\end{itemize}

\subsection{Falhas de Página}
\label{subsec:page_faults}

As falhas de página são exceções (interrupção 14) geradas pelo processador quando ocorre um problema durante a tradução de endereços:

\begin{itemize}
    \item \textbf{Causas comuns}:
    \begin{itemize}
        \item Página não presente na memória (bit P = 0)
        \item Violação de privilégios (tentativa de escrita em página somente-leitura)
        \item Acesso em nível de privilégio inadequado
    \end{itemize}

    \item \textbf{Informações disponíveis}:
    \begin{itemize}
        \item CR2: Contém o endereço linear que causou a falha
        \item Código de erro na pilha: Fornece detalhes sobre o tipo de falha
        \item Bit 0: Indica se a página não estava presente (0) ou houve violação de proteção (1)
        \item Bit 1: Indica se foi tentativa de escrita (1) ou leitura (0)
        \item Bit 2: Indica se o acesso foi em modo usuário (1) ou supervisor (0)
    \end{itemize}
\end{itemize}

\subsection{Usos Avançados da Paginação}
\label{subsec:usos_avancados}

A paginação permite implementar diversas funcionalidades avançadas em sistemas operacionais:

\begin{itemize}
    \item \textbf{Memória sob demanda}: Páginas são alocadas apenas quando necessárias

    \item \textbf{Copy-on-write}: Páginas são compartilhadas entre processos até que um deles tente escrever

    \item \textbf{Swapping}: Páginas pouco usadas podem ser movidas para armazenamento secundário

    \item \textbf{Isolamento de processos}: Cada processo tem seu próprio espaço de endereçamento virtual

    \item \textbf{Compartilhamento controlado}: Bibliotecas e recursos podem ser compartilhados entre processos

    \item \textbf{ASLR (Address Space Layout Randomization)}: Randomização da localização de áreas de memória para dificultar exploits
\end{itemize}

\subsection{Considerações de Desempenho}
\label{subsec:desempenho}

O sistema de memória é frequentemente um gargalo de desempenho, e várias técnicas são usadas para otimizá-lo:

\begin{itemize}
    \item \textbf{Hierarquia de cache}: L1, L2, L3 para reduzir latência de acesso

    \item \textbf{TLB}: Reduz o custo de tradução de endereços virtuais

    \item \textbf{Páginas grandes}: Reduzem o overhead do TLB ao cobrir mais memória com menos entradas

    \item \textbf{Prefetching}: Antecipação de acessos à memória para esconder latência

    \item \textbf{Alinhamento}: Dados alinhados são acessados mais eficientemente
\end{itemize}

O gerenciamento eficaz da memória é essencial para o desempenho do sistema como um todo, e compreender os mecanismos de baixo nível da arquitetura x86 permite implementações mais eficientes tanto em nível de sistema operacional quanto em aplicações de alta performance.

\section{Sistemas Operacionais e Gerenciamento de Memória}
\label{sec:b03_sistemas_operacionais}

Os sistemas operacionais modernos têm como um dos seus papéis fundamentais o gerenciamento eficiente da memória. Esta seção explora como os sistemas operacionais utilizam os mecanismos de hardware disponíveis na arquitetura x86 para implementar seus próprios sistemas de gerenciamento de memória.

\subsection{Hierarquia de Memória}
\label{subsec:hierarquia_memoria}

Os sistemas operacionais precisam lidar com uma hierarquia complexa de memória:

\begin{itemize}
    \item \textbf{Registradores}: Gerenciados pelo compilador e não diretamente pelo SO
    \item \textbf{Cache L1, L2, L3}: Gerenciados pelo hardware, com alguma influência do SO
    \item \textbf{RAM}: Principal área de gerenciamento do SO
    \item \textbf{Memória de troca (swap)}: Extensão da RAM em armazenamento secundário
    \item \textbf{Sistemas de arquivos}: Armazenamento persistente mapeável em memória
\end{itemize}

\subsection{Abstração de Memória}
\label{subsec:abstracao_memoria}

O sistema operacional proporciona diversas abstrações de memória para os aplicativos:

\begin{itemize}
    \item \textbf{Espaço de endereçamento virtual}: Cada processo tem a ilusão de um espaço de endereçamento contíguo e privado

    \item \textbf{Isolamento entre processos}: Impede que processos acessem a memória uns dos outros sem autorização explícita

    \item \textbf{Memória compartilhada}: Mecanismos para compartilhamento controlado de memória entre processos

    \item \textbf{Alocação dinâmica}: APIs como \texttt{malloc()} e \texttt{free()} para gerenciamento de memória em tempo de execução
\end{itemize}

\subsection{Layout da Memória do Processo}
\label{subsec:layout_memoria}

Em sistemas x86, o espaço de endereçamento de um processo tipicamente inclui:

\begin{itemize}
    \item \textbf{Segmento de texto}: Código executável, geralmente somente leitura

    \item \textbf{Segmento de dados}: Variáveis globais e estáticas inicializadas

    \item \textbf{Segmento BSS}: Variáveis globais e estáticas não inicializadas

    \item \textbf{Heap}: Área para alocação dinâmica, gerenciada por \texttt{malloc()} e similares

    \item \textbf{Mapeamentos}: Bibliotecas compartilhadas, arquivos mapeados em memória

    \item \textbf{Stack}: Pilha para chamadas de função, variáveis locais e contexto
\end{itemize}

\subsection{Estratégias de Gerenciamento}
\label{subsec:estrategias}

Os sistemas operacionais implementam diversas estratégias para gerenciar a memória:

\begin{itemize}
    \item \textbf{Alocação sob demanda}: Páginas são realmente alocadas apenas quando acessadas pela primeira vez (lazy allocation)

    \item \textbf{Copy-on-write (COW)}: Otimização onde páginas são compartilhadas entre processos até que um deles tente modificá-las

    \item \textbf{Page replacement}: Algoritmos para decidir quais páginas mover para o swap quando a memória está cheia (LRU, Clock, etc.)

    \item \textbf{Compactação de memória}: Técnicas como KSM (Kernel Samepage Merging) para identificar e fundir páginas idênticas
\end{itemize}

\subsection{Memória Virtual}
\label{subsec:memoria_virtual}

A memória virtual é a abstração central nos sistemas operacionais modernos:

\begin{itemize}
    \item \textbf{Endereçamento virtual}: Processos usam endereços virtuais, traduzidos para físicos pelo MMU

    \item \textbf{Swapping}: Páginas pouco utilizadas são movidas para armazenamento secundário

    \item \textbf{Overcommit}: Alocação de mais memória virtual do que a memória física disponível

    \item \textbf{Memória sob demanda}: Páginas são carregadas do disco apenas quando necessárias
\end{itemize}

\subsection{Implementação de Paginação nos SOs}
\label{subsec:impl_paginacao}

Os sistemas operacionais x86 modernos implementam sofisticados sistemas de paginação:

\begin{itemize}
    \item \textbf{Tabelas de página multinível}: Linux, Windows e outros SOs usam a estrutura multinível suportada pelo hardware x86

    \item \textbf{Estruturas de rastreamento}: O kernel mantém estruturas auxiliares para rastrear o estado de cada página física

    \item \textbf{Page fault handler}: Rotina crítica do kernel que determina como responder a cada falha de página

    \item \textbf{Compartilhamento de páginas}: Mecanismos para mapear a mesma página física em múltiplos espaços de endereçamento
\end{itemize}

\subsection{Page Fault Handler}
\label{subsec:page_fault_handler}

O tratador de falhas de página é um componente crítico do SO, que pode responder de várias formas:

\begin{itemize}
    \item \textbf{Demand paging}: Carrega uma página do disco que foi previamente swapped out

    \item \textbf{Copy-on-write}: Cria uma cópia privada de uma página compartilhada quando ocorre tentativa de escrita

    \item \textbf{Alocação lazy}: Aloca uma página física para um endereço virtual previamente reservado mas não alocado

    \item \textbf{Stack growth}: Expande automaticamente a pilha quando necessário

    \item \textbf{Erro de segmentação}: Quando o acesso é inválido, sinaliza SIGSEGV ao processo
\end{itemize}

\subsection{Serviços de Memória para Aplicações}
\label{subsec:servicos_memoria}

Sistemas operacionais fornecem APIs para gerenciamento de memória:

\begin{itemize}
    \item \textbf{POSIX}:
    \begin{itemize}
        \item \texttt{malloc()}, \texttt{free()}: Alocação dinâmica
        \item \texttt{mmap()}, \texttt{munmap()}: Mapeamento de memória
        \item \texttt{brk()}, \texttt{sbrk()}: Modificação do limite do heap
        \item \texttt{mprotect()}: Alteração das permissões de páginas
    \end{itemize}

    \item \textbf{Windows}:
    \begin{itemize}
        \item \texttt{VirtualAlloc()}, \texttt{VirtualFree()}
        \item \texttt{HeapAlloc()}, \texttt{HeapFree()}
        \item \texttt{MapViewOfFile()}
    \end{itemize}
\end{itemize}

\subsection{Segurança de Memória}
\label{subsec:seguranca_memoria}

Os sistemas operacionais implementam diversas técnicas para melhorar a segurança da memória:

\begin{itemize}
    \item \textbf{ASLR (Address Space Layout Randomization)}: Randomiza a posição das regiões de memória para dificultar ataques

    \item \textbf{DEP/NX (Data Execution Prevention/No-eXecute)}: Impede a execução de código em páginas de dados

    \item \textbf{Stack Canaries}: Detecção de overflow de buffer na pilha

    \item \textbf{KASLR (Kernel ASLR)}: Randomiza a posição do código do kernel na memória

    \item \textbf{SMAP/SMEP}: Proteções para prevenir que o kernel acesse ou execute código de modo usuário inadvertidamente
\end{itemize}

\subsection{Considerações em Ambientes Baremetal}
\label{subsec:baremetal_so}

Em ambientes baremetal ou em kernels simples, o programador precisa implementar seu próprio sistema de gerenciamento de memória:

\begin{itemize}
    \item \textbf{Inicialização do hardware}: Configurar GDT, IDT, e habilitar paginação manualmente

    \item \textbf{Alocação básica}: Implementar alocadores simples para gerenciar memória física

    \item \textbf{Mapeamento manual}: Configurar tabelas de página diretamente, sem as abstrações de um SO completo

    \item \textbf{Tratamento limitado de falhas}: Implementar tratadores básicos para falhas de página
\end{itemize}

Esta abordagem "mínima" é valiosa para compreender os mecanismos de baixo nível que normalmente são abstraídos pelos sistemas operacionais completos, e constitui um excelente exercício educacional para entender como o hardware e software interagem no gerenciamento de memória.

\section{Paginação na Arquitetura x86}
\label{sec:b04_paginacao}

A paginação é o mecanismo fundamental de gerenciamento de memória nos processadores modernos x86, permitindo a implementação de memória virtual, proteção entre processos e diversas outras funcionalidades essenciais para sistemas operacionais. Esta seção explora em detalhes os mecanismos de paginação na arquitetura Intel IA-32/x86-64, conforme documentado no Volume 3 do Intel® 64 and IA-32 Architectures Software Developer's Manual.

\subsection{Fundamentos da Paginação}
\label{subsec:fundamentos_paginacao}

A paginação divide a memória em unidades fixas chamadas páginas e proporciona um mapeamento flexível entre endereços lineares (virtuais) e físicos:

\begin{itemize}
    \item \textbf{Páginas}: Blocos de memória de tamanho fixo, tipicamente 4 KiB na arquitetura x86
    \item \textbf{Frames}: As unidades físicas correspondentes às páginas virtuais
    \item \textbf{MMU (Memory Management Unit)}: Hardware responsável pela tradução automática de endereços
    \item \textbf{Tabelas de Página}: Estruturas de dados que definem o mapeamento entre endereços virtuais e físicos
\end{itemize}

A paginação oferece diversas vantagens:
\begin{itemize}
    \item Isolamento entre processos
    \item Compartilhamento controlado de memória
    \item Implementação eficiente de memória virtual
    \item Proteção granular (permissões por página)
    \item Uso eficiente da memória física fragmentada
\end{itemize}

\subsection{Habilitando a Paginação}
\label{subsec:habilitar_paginacao}

Para habilitar a paginação em um processador x86, é necessário:

\begin{enumerate}
    \item Preparar as estruturas de tabelas de página na memória
    \item Carregar o endereço físico da tabela de diretório de páginas no registro CR3
    \item Habilitar a paginação setando o bit PG (bit 31) no registro CR0
    \item Opcionalmente, configurar recursos avançados via CR4 (PSE, PAE, etc.)
\end{enumerate}

Esta sequência deve ser cuidadosamente implementada, pois após habilitar a paginação, o processador imediatamente começa a traduzir todos os endereços. Se o mapeamento inicial não estiver corretamente configurado, o processador não conseguirá continuar a execução.

\subsubsection{Código de Exemplo para Habilitar Paginação}

Para habilitar paginação com páginas de 4MB no modo de 32 bits:

\begin{enumerate}
    \item Declarar e inicializar a tabela de diretório de páginas:
    \begin{itemize}
        \item Alinhar em limite de 4KB: \texttt{\_\_attribute\_\_((aligned(4096)))}
        \item Inicializar pelo menos a primeira entrada para mapear os primeiros 4MB: \texttt{(1 << 7) | (1 << 1) | (1 << 0)}
        \item O bit 7 (PSE) indica página de 4MB, bit 1 (R/W) permite escrita, bit 0 (P) marca como presente
    \end{itemize}

    \item Carregar o endereço da tabela em CR3: \texttt{set\_cr3((uint32\_t)\&tabela\_paginas);}

    \item Habilitar páginas de 4MB em CR4: \texttt{set\_cr4(get\_cr4() | (1 << 4));}

    \item Habilitar paginação em CR0: \texttt{set\_cr0(get\_cr0() | (1 << 31));}
\end{enumerate}

\subsection{Estruturas de Paginação de 32 bits}
\label{subsec:estruturas_paginacao_32bits}

Na arquitetura IA-32, a paginação pode ser configurada em diferentes modos:

\subsubsection{Paginação de 32 bits Padrão (Páginas de 4KB)}

Este modo utiliza um esquema de dois níveis:

\begin{itemize}
    \item \textbf{Diretório de Páginas (PD)}: Tabela de nível superior com 1024 entradas de 4 bytes
    \item \textbf{Tabelas de Página (PT)}: Tabelas de segundo nível, cada uma com 1024 entradas de 4 bytes
\end{itemize}

Um endereço linear de 32 bits é dividido em:
\begin{itemize}
    \item Bits 31-22 (10 bits): Índice no Diretório de Páginas
    \item Bits 21-12 (10 bits): Índice na Tabela de Páginas
    \item Bits 11-0 (12 bits): Deslocamento dentro da página de 4KB
\end{itemize}

\subsubsection{PSE (Page Size Extension) - Páginas de 4MB}

Quando o bit PSE em CR4 está habilitado:
\begin{itemize}
    \item O processador suporta páginas de 4MB
    \item O esquema é simplificado para um único nível
    \item Um endereço linear de 32 bits é dividido em:
    \begin{itemize}
        \item Bits 31-22 (10 bits): Índice no Diretório de Páginas
        \item Bits 21-0 (22 bits): Deslocamento dentro da página de 4MB
    \end{itemize}
\end{itemize}

\subsubsection{PAE (Physical Address Extension)}

O PAE estende o endereçamento físico para 36 bits (permitindo acessar até 64GB de RAM):
\begin{itemize}
    \item Esquema de três níveis
    \item Diretório de Diretórios de Página (PDPT) com 4 entradas de 8 bytes
    \item Diretórios de Páginas, cada um com 512 entradas de 8 bytes
    \item Tabelas de Página, cada uma com 512 entradas de 8 bytes
    \item Suporta páginas de 4KB e 2MB (com PSE)
\end{itemize}

\subsection{Formato das Entradas de Tabelas de Página}
\label{subsec:formato_entradas}

\subsubsection{Entrada de Diretório de Página (PDE) em Modo 32 bits}

Uma entrada de 32 bits no diretório de páginas possui os seguintes campos:

\begin{itemize}
    \item Bits 31-12: Endereço físico base da tabela de páginas (para páginas de 4KB) ou da página (para 4MB)
    \item Bit 7 (PS): Page Size - 0 para 4KB, 1 para 4MB
    \item Bit 6 (D): Dirty - Indica se a página foi escrita
    \item Bit 5 (A): Accessed - Indica se a página foi acessada
    \item Bit 4 (PCD): Page Cache Disable - Desabilita cache para esta página
    \item Bit 3 (PWT): Page Write Through - Configura cache para write-through
    \item Bit 2 (U/S): User/Supervisor - 0 para página de supervisor, 1 para página de usuário
    \item Bit 1 (R/W): Read/Write - 0 para somente leitura, 1 para leitura/escrita
    \item Bit 0 (P): Present - 1 indica página presente na memória física
\end{itemize}

\subsubsection{Entrada de Tabela de Página (PTE) em Modo 32 bits}

Similar à PDE, mas sempre referencia uma página de 4KB:

\begin{itemize}
    \item Bits 31-12: Endereço físico base da página de 4KB
    \item Os bits de controle (A, D, PCD, PWT, U/S, R/W, P) têm o mesmo significado que na PDE
\end{itemize}

\subsection{Tradução de Endereços}
\label{subsec:traducao_enderecos}

O processo de tradução de endereços usando paginação de 32 bits acontece da seguinte forma:

\begin{enumerate}
    \item O processador extrai os bits 31-22 do endereço linear para obter o índice no diretório de páginas
    \item Multiplica este índice por 4 (tamanho de cada entrada) e soma ao endereço base do diretório (armazenado em CR3)
    \item Lê a entrada do diretório (PDE) neste endereço
    \item Verifica se a PDE está presente (bit P) e se tem as permissões apropriadas
    \item Se a PDE indica uma página de 4MB (bit PS=1):
        \begin{itemize}
            \item O endereço físico é formado combinando os bits 31-22 da PDE com os bits 21-0 do endereço linear
        \end{itemize}
    \item Se a PDE indica uma tabela de páginas (bit PS=0):
        \begin{itemize}
            \item Extrai os bits 21-12 do endereço linear para obter o índice na tabela de páginas
            \item Multiplica este índice por 4 e soma ao endereço base da tabela (obtido da PDE)
            \item Lê a entrada da tabela (PTE) neste endereço
            \item Verifica se a PTE está presente e tem as permissões apropriadas
            \item O endereço físico é formado combinando os bits 31-12 da PTE com os bits 11-0 do endereço linear
        \end{itemize}
\end{enumerate}

\subsection{Falhas de Página}
\label{subsec:falhas_pagina}

Uma falha de página (Page Fault) ocorre quando há um problema durante a tradução de endereço:

\begin{itemize}
    \item O processador gera uma exceção de falha de página (interrupção 14)
    \item O código de erro empilhado fornece informações sobre a causa:
        \begin{itemize}
            \item Bit 0: 0 se a página não estava presente, 1 se foi violação de proteção
            \item Bit 1: 0 se foi acesso de leitura, 1 se foi acesso de escrita
            \item Bit 2: 0 se foi acesso em modo kernel, 1 se foi acesso em modo usuário
            \item Bit 3: 1 se a falha envolveu bits reservados
            \item Bit 4: 1 se a falha ocorreu durante uma busca de instrução
        \end{itemize}
    \item O endereço que causou a falha é armazenado no registro CR2
\end{itemize}

O tratador de falhas de página pode tomar diferentes ações:
\begin{itemize}
    \item Carregar a página da memória de troca (implementando memória virtual)
    \item Alocar uma nova página para alocação sob demanda
    \item Criar uma cópia da página para implementar copy-on-write
    \item Terminar o processo se o acesso for inválido (segmentation fault)
\end{itemize}

\subsection{Translation Lookaside Buffer (TLB)}
\label{subsec:tlb}

O TLB é uma cache de traduções de endereços que acelera o processo de paginação:

\begin{itemize}
    \item Armazena as traduções mais recentes de endereços lineares para físicos
    \item É consultado antes das tabelas de página na memória
    \item Se uma tradução está presente no TLB (hit), a consulta às tabelas é evitada
    \item É gerenciado automaticamente pelo hardware, mas requer intervenção do software em alguns casos
\end{itemize}

Quando as tabelas de página são modificadas, o TLB deve ser invalidado:
\begin{itemize}
    \item \textbf{Invalidação específica}: A instrução INVLPG invalida uma entrada específica do TLB
    \item \textbf{Invalidação global}: Recarregar CR3 invalida todo o TLB (exceto entradas marcadas como globais)
\end{itemize}

\subsection{Flags e Bits de Controle}
\label{subsec:flags_controle}

Além dos bits nas entradas das tabelas, vários bits nos registradores de controle afetam o comportamento da paginação:

\begin{itemize}
    \item \textbf{CR0.WP} (Write Protect, bit 16):
        \begin{itemize}
            \item Quando 0, o kernel pode escrever em páginas somente-leitura
            \item Quando 1, as proteções de escrita são aplicadas mesmo em modo kernel
        \end{itemize}

    \item \textbf{CR4.PSE} (Page Size Extensions, bit 4):
        \begin{itemize}
            \item Quando 1, habilita suporte a páginas de 4MB em modo 32 bits
        \end{itemize}

    \item \textbf{CR4.PGE} (Page Global Enable, bit 7):
        \begin{itemize}
            \item Quando 1, permite marcar páginas como globais (não são invalidadas no TLB durante trocas de CR3)
        \end{itemize}

    \item \textbf{CR4.PAE} (Physical Address Extension, bit 5):
        \begin{itemize}
            \item Quando 1, habilita PAE para suportar endereçamento físico estendido
        \end{itemize}
\end{itemize}

\subsection{Técnicas Avançadas}
\label{subsec:tecnicas_avancadas}

\subsubsection{Mapeamentos Múltiplos}

É possível mapear o mesmo frame físico em múltiplos endereços virtuais:
\begin{itemize}
    \item Útil para compartilhamento de memória entre processos
    \item Permite ter diferentes permissões para o mesmo conteúdo físico
    \item Facilita operações copy-on-write
\end{itemize}

\subsubsection{Páginas de Kernel Mapeadas em Todo Espaço de Endereçamento}

Sistemas operacionais modernos normalmente:
\begin{itemize}
    \item Dividem o espaço de endereço virtual em porções de usuário e kernel
    \item Mapeiam o código e dados do kernel em todos os espaços de endereço
    \item Marcam essas páginas como globais para evitar invalidações do TLB durante trocas de contexto
\end{itemize}

\subsubsection{Paginação Aninhada e Virtualização}

Em ambientes virtualizados:
\begin{itemize}
    \item O hardware moderno suporta paginação aninhada (NPT/EPT)
    \item Permite que máquinas virtuais usem paginação sem overhead excessivo
    \item Adiciona uma camada extra de tradução: GVA → GPA → HPA (Guest Virtual → Guest Physical → Host Physical)
\end{itemize}

\subsection{Considerações de Implementação Prática}
\label{subsec:implementacao_pratica}

Ao implementar paginação em um ambiente baremetal ou kernel simples:

\begin{itemize}
    \item \textbf{Identity mapping inicial}: Mapeie os primeiros MB da memória física para endereços virtuais idênticos
    \item \textbf{Mapeamento do código}: Garanta que o código do kernel permaneça acessível após habilitar paginação
    \item \textbf{Alinhamento}: Todas as estruturas de paginação devem estar alinhadas em limites de página (4KB)
    \item \textbf{Tratamento de interrupções}: Configure o tratador de falhas de página antes de experimentar com mapeamentos
    \item \textbf{Inicialização cuidadosa}: A sequência de inicialização deve garantir que a próxima instrução após habilitar paginação seja acessível
\end{itemize}

A paginação é uma das funcionalidades mais poderosas e complexas da arquitetura x86. Sua compreensão detalhada é fundamental para o desenvolvimento de sistemas operacionais e para a implementação de funcionalidades avançadas de gerenciamento de memória.


% References
\chapter{Referências}\label{chap:refs}
\include{x_referencias/x01_referencias}

% Appendices if needed
\appendix
% Uncomment and add appendix includes as needed
% \chapter{Apêndice A}\label{chap:appendixA}
% \chapter{Código Fonte}\label{chap:appendixA}

\section{Arquivo lockdep.h}

Este arquivo define as estruturas de dados e a API pública do sistema de detecção de deadlocks.

\begin{lstlisting}[language=C, caption={lockdep.h - API de detecção de deadlocks}]
// ARCHITECTURE OVERVIEW:
//
// 1. LOCK GRAPH: All locks should be tracked as nodes in a directed graph where
//    edges represent ordering dependencies (A → B means A acquired before B).
//
// 2. THREAD TRACKING: Each thread should maintain a stack of currently held
//    locks to detect nested locking patterns and build dependencies.
//
// 3. DEADLOCK DETECTION: The system checks for cycles in the lock graph
//    to detect potential deadlocks. If a cycle is found, the system should
//    identify the lock and prevent the acquisition that would lead to a
//    deadlock.

#ifndef LOCKDEP_H
#define LOCKDEP_H

#include <pthread.h>
#include <stdbool.h>
#include <stddef.h>
#include <stdint.h>

typedef struct lock_node lock_node_t;
typedef struct dependency_edge dependency_edge_t;
typedef struct thread_context thread_context_t;

// Representa todos os locks conhecidos pelo sistema lockdep como um nó de lista encadeada.
// Cada nó identifica unicamente um lock (ex: por seu endereço) e permite
// percorrer todos os locks registrados.
typedef struct lock_node {
    // Identifica unicamente o lock (ex: endereço do mutex)
    void* lock_addr;
    // Para percorrer a lista
    struct lock_node* next;
} lock_node_t;

// Representa uma aresta direcionada de dependência no grafo de locks.
// Cada aresta codifica a ordenação "lock A adquirido antes do lock B" e
// forma uma lista encadeada para algoritmos de detecção de ciclos.
typedef struct dependency_edge {
    // Nó de lock de origem (lock "de")
    lock_node_t* from;
    // Nó de lock de destino (lock "para")
    lock_node_t* to;
    // Lista encadeada de todas as arestas de dependência
    struct dependency_edge* next;
} dependency_edge_t;

// Nó de pilha representando um lock atualmente mantido por uma thread.
// Permite rastrear aquisições de locks aninhados por thread.
typedef struct held_lock {
    // O lock que está sendo mantido
    lock_node_t* lock;
    // Próximo lock na pilha (lock mais recente no topo)
    struct held_lock* next;
} held_lock_t;

// Contexto por thread para rastrear locks mantidos por cada thread.
// Mantém uma pilha de locks mantidos, a profundidade atual da pilha e vincula
// todos os contextos de thread para percorrer.
typedef struct thread_context {
    pthread_t thread_id;
    // Pilha de locks atualmente mantidos por esta thread
    held_lock_t* held_locks;
    // O tamanho da pilha de locks mantidos
    int lock_depth;
    // Vincula todos os contextos de thread para fácil percurso
    struct thread_context* next;
} thread_context_t;

void lockdep_init(void);
void lockdep_cleanup(void);

// Registra a aquisição de um lock pela thread atual. `lock_addr` é o
// endereço do lock sendo adquirido. Retorna true se a aquisição é permitida,
// false se causaria um deadlock.
bool lockdep_acquire_lock(void* lock_addr);

// Registra a liberação de um lock pela thread atual. `lock_addr` é o
// lock sendo liberado.
void lockdep_release_lock(void* lock_addr);

// Verifica o grafo de locks por ciclos (deadlocks potenciais).
// Retorna true se um deadlock é detectado, false caso contrário.
bool lockdep_check_deadlock(void);

// Mostra todas as relações A → B no grafo de locks.
void lockdep_print_dependencies(void);

// Para desabilitar o lockdep sem recompilação.
extern bool lockdep_enabled;

#endif  // LOCKDEP_H!
\end{lstlisting}

\section{Arquivo lockdep\_core.c}

Este arquivo implementa a lógica principal de detecção de deadlocks, incluindo o rastreamento de dependências de locks e a verificação de ciclos no grafo.

\begin{lstlisting}[language=C, caption={lockdep\_core.c - Implementação do sistema de detecção de deadlocks}]
#include <execinfo.h>
#include <pthread.h>
#include <stdbool.h>
#include <stdio.h>
#include <stdlib.h>
#include <string.h>
#include <unistd.h>

#include "lockdep.h"

bool lockdep_enabled = true;

// Estado global do grafo de locks
static lock_node_t* lock_graph = NULL;
static dependency_edge_t* dependencies = NULL;
static thread_context_t* thread_contexts = NULL;

// Mutex para proteger o estado interno do grafo de locks
static pthread_mutex_t lockdep_mutex = PTHREAD_MUTEX_INITIALIZER;

// Declarações avançadas para funções internas
static thread_context_t* get_thread_context(void);
static lock_node_t* find_or_create_lock_node(void* lock_addr);
static bool check_cycle_from(lock_node_t* start, lock_node_t* target, bool* visited);
static bool add_dependency(lock_node_t* from, lock_node_t* to);
static void print_backtrace(void);

void lockdep_init(void) {
    const char* env = getenv("LOCKDEP_DISABLE");
    if (env && strcmp(env, "1") == 0) {
        lockdep_enabled = false;
        return;
    }

    fprintf(stderr, "[LOCKDEP] Lockdep initialized\n");
}

void lockdep_cleanup(void) {
    pthread_mutex_lock(&lockdep_mutex);

    // Libera os nós de lock
    lock_node_t* node = lock_graph;
    while (node) {
        lock_node_t* next = node->next;
        free(node);
        node = next;
    }

    // Libera as dependências
    dependency_edge_t* edge = dependencies;
    while (edge) {
        dependency_edge_t* next = edge->next;
        free(edge);
        edge = next;
    }

    // Libera os contextos de thread e seus locks mantidos
    thread_context_t* ctx = thread_contexts;
    while (ctx) {
        thread_context_t* next_ctx = ctx->next;

        // Libera a pilha de locks mantidos
        held_lock_t* held = ctx->held_locks;
        while (held) {
            held_lock_t* next_held = held->next;
            free(held);
            held = next_held;
        }

        free(ctx);
        ctx = next_ctx;
    }

    lock_graph = NULL;
    dependencies = NULL;
    thread_contexts = NULL;

    pthread_mutex_unlock(&lockdep_mutex);
}

bool lockdep_acquire_lock(void* lock_addr) {
    if (!lockdep_enabled) {
        return true;
    }

    pthread_mutex_lock(&lockdep_mutex);

    printf("[LOCKDEP] Thread %lu acquiring lock at %p\n",
           (unsigned long)pthread_self(), lock_addr);

    // Obtém ou cria o nó de lock para este endereço de lock
    lock_node_t* lock_node = find_or_create_lock_node(lock_addr);

    // Obtém o contexto da thread
    thread_context_t* thread_ctx = get_thread_context();

    // Verifica se já temos locks mantidos e precisamos adicionar dependências
    if (thread_ctx->held_locks != NULL) {
        // O lock adquirido mais recentemente deve ter uma dependência neste novo lock
        lock_node_t* prev_lock = thread_ctx->held_locks->lock;

        // Adiciona dependência: prev_lock -> lock_node
        if (!add_dependency(prev_lock, lock_node)) {
            // A dependência criaria um ciclo - deadlock potencial!
            fprintf(stderr, "[LOCKDEP] AVISO: Violação de ordem de lock detectada!\n");
            fprintf(stderr, "[LOCKDEP] Thread %lu tentando adquirir %p enquanto mantém %p\n",
                    (unsigned long)pthread_self(), lock_addr, prev_lock->lock_addr);
            fprintf(stderr, "[LOCKDEP] Isso viola a ordem de lock observada anteriormente e pode levar a deadlocks.\n");
            print_backtrace();

            // Verifica se temos um ciclo real no grafo de dependência
            bool result = lockdep_check_deadlock();
            if (result) {
                fprintf(stderr, "[LOCKDEP] POTENCIAL DE DEADLOCK: Dependência circular de lock detectada!\n");
                pthread_mutex_unlock(&lockdep_mutex);
                return false;
            } else {
                fprintf(stderr, "[LOCKDEP] Apenas aviso: Sem dependência circular ainda, mas ordem de lock inconsistente\n");
            }
        }
    }

    // Empurra este lock para a pilha de locks mantidos pela thread
    held_lock_t* new_held = malloc(sizeof(held_lock_t));
    if (!new_held) {
        perror("[LOCKDEP] Falha ao alocar memória para lock mantido");
        pthread_mutex_unlock(&lockdep_mutex);
        return true; // Continua sem rastreamento em caso de falha na alocação
    }

    new_held->lock = lock_node;
    new_held->next = thread_ctx->held_locks;
    thread_ctx->held_locks = new_held;
    thread_ctx->lock_depth++;

    pthread_mutex_unlock(&lockdep_mutex);
    return true;
}

void lockdep_release_lock(void* lock_addr) {
    if (!lockdep_enabled) {
        return;
    }

    pthread_mutex_lock(&lockdep_mutex);

    printf("[LOCKDEP] Thread %lu releasing lock at %p\n",
           (unsigned long)pthread_self(), lock_addr);

    // Obtém o contexto da thread
    thread_context_t* thread_ctx = get_thread_context();

    // Encontra e remove o lock da pilha de locks mantidos pela thread
    held_lock_t** curr = &thread_ctx->held_locks;
    while (*curr) {
        if ((*curr)->lock->lock_addr == lock_addr) {
            held_lock_t* to_free = *curr;
            *curr = (*curr)->next;
            free(to_free);
            thread_ctx->lock_depth--;
            break;
        }
        curr = &(*curr)->next;
    }

    pthread_mutex_unlock(&lockdep_mutex);
}

bool lockdep_check_deadlock(void) {
    if (!lockdep_enabled) {
        return false;
    }

    pthread_mutex_lock(&lockdep_mutex);

    bool deadlock_detected = false;

    // Aloca array de visitados para cada nó de lock
    int node_count = 0;
    lock_node_t* node;
    for (node = lock_graph; node != NULL; node = node->next) {
        node_count++;
    }

    // Para cada lock, verifica se há um caminho de volta para si mesmo
    for (node = lock_graph; node != NULL; node = node->next) {
        bool* visited = calloc(node_count, sizeof(bool));
        if (!visited) {
            perror("[LOCKDEP] Falha ao alocar memória para detecção de deadlock");
            continue;
        }

        if (check_cycle_from(node, node, visited)) {
            fprintf(stderr, "[LOCKDEP] Potencial de deadlock: Encontrado ciclo começando no lock %p\n",
                    node->lock_addr);
            deadlock_detected = true;
            free(visited);
            break;
        }

        free(visited);
    }

    pthread_mutex_unlock(&lockdep_mutex);
    return deadlock_detected;
}

void lockdep_print_dependencies(void) {
    if (!lockdep_enabled) {
        return;
    }

    pthread_mutex_lock(&lockdep_mutex);

    printf("\n[LOCKDEP] === Grafo de Dependência de Locks ===\n");

    // Imprime todas as arestas no grafo de dependência
    dependency_edge_t* edge = dependencies;
    while (edge) {
        printf("[LOCKDEP] %p -> %p\n", edge->from->lock_addr, edge->to->lock_addr);
        edge = edge->next;
    }

    // Imprime todos os contextos de thread e seus locks mantidos
    printf("\n[LOCKDEP] === Estados de Lock das Threads ===\n");
    thread_context_t* ctx = thread_contexts;
    while (ctx) {
        printf("[LOCKDEP] Thread %lu mantém %d locks: ",
               (unsigned long)ctx->thread_id, ctx->lock_depth);

        held_lock_t* held = ctx->held_locks;
        while (held) {
            printf("%p ", held->lock->lock_addr);
            held = held->next;
        }
        printf("\n");

        ctx = ctx->next;
    }

    printf("[LOCKDEP] ===========================\n\n");

    pthread_mutex_unlock(&lockdep_mutex);
}

// Função auxiliar para obter o contexto de thread para a thread atual
static thread_context_t* get_thread_context(void) {
    pthread_t self = pthread_self();

    // Verifica se já temos um contexto para esta thread
    thread_context_t* ctx = thread_contexts;
    while (ctx) {
        if (pthread_equal(ctx->thread_id, self)) {
            return ctx;
        }
        ctx = ctx->next;
    }

    // Cria um novo contexto de thread se não for encontrado
    ctx = malloc(sizeof(thread_context_t));
    if (!ctx) {
        perror("[LOCKDEP] Falha ao alocar memória para contexto de thread");
        return NULL;
    }

    ctx->thread_id = self;
    ctx->held_locks = NULL;
    ctx->lock_depth = 0;
    ctx->next = thread_contexts;
    thread_contexts = ctx;

    return ctx;
}

// Função auxiliar para encontrar ou criar um nó de lock
static lock_node_t* find_or_create_lock_node(void* lock_addr) {
    // Verifica se o lock já existe
    lock_node_t* node = lock_graph;
    while (node) {
        if (node->lock_addr == lock_addr) {
            return node;
        }
        node = node->next;
    }

    // Cria um novo nó de lock
    node = malloc(sizeof(lock_node_t));
    if (!node) {
        perror("[LOCKDEP] Falha ao alocar memória para nó de lock");
        return NULL;
    }

    node->lock_addr = lock_addr;
    node->next = lock_graph;
    lock_graph = node;

    return node;
}

// Função auxiliar para verificar ciclos no grafo de dependência usando DFS
static bool check_cycle_from(lock_node_t* current, lock_node_t* target, bool* visited) {
    // Encontra o índice do nó atual
    int current_idx = 0;
    lock_node_t* node = lock_graph;
    while (node != current) {
        current_idx++;
        node = node->next;
    }

    // Se já visitamos este nó nesta travessia DFS, pulamos
    if (visited[current_idx]) {
        return false;
    }

    // Marca o nó atual como visitado
    visited[current_idx] = true;

    // Verifica todas as arestas de saída do nó atual
    dependency_edge_t* edge = dependencies;
    while (edge) {
        if (edge->from == current) {
            // Se encontramos nosso alvo, temos um ciclo
            if (edge->to == target) {
                return true;
            }

            // Continua DFS a partir do nó de destino
            if (check_cycle_from(edge->to, target, visited)) {
                return true;
            }
        }
        edge = edge->next;
    }

    return false;
}

// Função auxiliar para adicionar uma dependência entre locks
static bool add_dependency(lock_node_t* from, lock_node_t* to) {
    // Primeiro verifica se esta dependência já existe
    dependency_edge_t* edge = dependencies;
    while (edge) {
        if (edge->from == from && edge->to == to) {
            return true; // Dependência já existe
        }
        edge = edge->next;
    }

    // Adiciona a nova dependência
    edge = malloc(sizeof(dependency_edge_t));
    if (!edge) {
        perror("[LOCKDEP] Falha ao alocar memória para aresta de dependência");
        return true; // Continua sem adicionar em caso de falha na alocação
    }

    edge->from = from;
    edge->to = to;
    edge->next = dependencies;
    dependencies = edge;

    // Verifica se esta nova dependência cria um ciclo
    bool* visited = calloc(1000, sizeof(bool)); // Assumindo máximo de 1000 locks por simplicidade
    if (!visited) {
        perror("[LOCKDEP] Falha ao alocar memória para detecção de ciclo");
        return true; // Continua sem verificar em caso de falha na alocação
    }

    // Verifica se há um caminho de 'to' de volta para 'from', o que criaria um ciclo
    bool has_cycle = check_cycle_from(to, from, visited);

    free(visited);
    return !has_cycle; // Retorna falso se o ciclo existir
}

// Função auxiliar para imprimir um backtrace quando violações de ordem de lock são detectadas
static void print_backtrace(void) {
    void* callstack[128];
    int frames = backtrace(callstack, 128);
    char** symbols = backtrace_symbols(callstack, frames);

    fprintf(stderr, "[LOCKDEP] Backtrace de violação de ordem de lock:\n");
    for (int i = 0; i < frames; i++) {
        fprintf(stderr, "  %s\n", symbols[i]);
    }

    free(symbols);
}
\end{lstlisting}

\section{Arquivo pthread\_interpose.c}

Este arquivo implementa a camada de interposição que intercepta as chamadas de mutex do pthread.

\begin{lstlisting}[language=C, caption={pthread\_interpose.c - Interposição de funções pthread}]
#include <dlfcn.h>
#include <errno.h>
#include <pthread.h>
#include <semaphore.h>
#include <stdio.h>

#include "lockdep.h"

static int (*real_pthread_mutex_lock)(pthread_mutex_t*) = NULL;
static int (*real_pthread_mutex_unlock)(pthread_mutex_t*) = NULL;
static int (*real_pthread_mutex_trylock)(pthread_mutex_t*) = NULL;

/// Esta função interpõe as funções reais do mutex pthread para adicionar validação lockdep
static void init_real_functions(void) {
    if (!real_pthread_mutex_lock) {
        real_pthread_mutex_lock = dlsym(RTLD_NEXT, "pthread_mutex_lock");
    }
    if (!real_pthread_mutex_unlock) {
        real_pthread_mutex_unlock = dlsym(RTLD_NEXT, "pthread_mutex_unlock");
    }
    if (!real_pthread_mutex_trylock) {
        real_pthread_mutex_trylock = dlsym(RTLD_NEXT, "pthread_mutex_trylock");
    }
}

__attribute__((constructor)) static void lockdep_constructor(void) {
    lockdep_init();
    init_real_functions();
}

__attribute__((destructor)) static void lockdep_destructor(void) {
    lockdep_cleanup();
}

/// O lockdep usa um mutex para proteger seu estado interno, então usamos isto para
/// evitar recursão na validação do lockdep através dele mesmo
static __thread bool in_interpose = false;

int pthread_mutex_lock(pthread_mutex_t* mutex) {
    init_real_functions();

    if (lockdep_enabled && !in_interpose) {
        in_interpose = true;
        if (!lockdep_acquire_lock(mutex)) {
            fprintf(
                stderr,
                "[LOCKDEP] DEADLOCK PREVENIDO - recusando adquirir lock\n");
            in_interpose = false;
            return EDEADLK;
        }
        in_interpose = false;
    }

    int result = real_pthread_mutex_lock(mutex);

    return result;
}

int pthread_mutex_unlock(pthread_mutex_t* mutex) {
    init_real_functions();

    int result = real_pthread_mutex_unlock(mutex);

    if (lockdep_enabled && !in_interpose) {
        in_interpose = true;
        lockdep_release_lock(mutex);
        in_interpose = false;
    }

    return result;
}

int pthread_mutex_trylock(pthread_mutex_t* mutex) {
    init_real_functions();

    int result = real_pthread_mutex_trylock(mutex);

    if (result == 0 && lockdep_enabled && !in_interpose) {
        in_interpose = true;
        if (!lockdep_acquire_lock(mutex)) {
            fprintf(stderr,
                    "[LOCKDEP] DEADLOCK DETECTADO em trylock - desbloqueando e "
                    "falhando\n");
            real_pthread_mutex_unlock(mutex);
            in_interpose = false;
            return EBUSY;
        }
        in_interpose = false;
    }

    return result;
}
\end{lstlisting}


\end{document}

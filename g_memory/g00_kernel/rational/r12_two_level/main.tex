\documentclass[a4paper,12pt]{report}

% Basic packages for LuaLaTeX with Unicode support
\usepackage{fontspec}
\usepackage[brazil]{babel}
\usepackage{graphicx}
\usepackage{hyperref}
\usepackage{amsmath}
\usepackage{amssymb}
\usepackage{listings}
\usepackage{xcolor}
\usepackage{geometry}
\usepackage{indentfirst}
\usepackage{setspace}

% Hyperref configuration
\hypersetup{
    colorlinks=true,
    linkcolor=blue,
    filecolor=magenta,
    urlcolor=cyan,
    pdftitle={Task 12: Two-Level Paging},
    pdfauthor={Your Name},
    pdfsubject={Memory Management in x86},
    pdfkeywords={paging, two-level paging, memory management, x86}
}

% Code listing styling
\definecolor{codegreen}{rgb}{0,0.6,0}
\definecolor{codegray}{rgb}{0.5,0.5,0.5}
\definecolor{codepurple}{rgb}{0.58,0,0.82}
\definecolor{backcolour}{rgb}{0.95,0.95,0.92}

\lstdefinestyle{mystyle}{
    backgroundcolor=\color{backcolour},
    commentstyle=\color{codegreen},
    keywordstyle=\color{magenta},
    numberstyle=\tiny\color{codegray},
    stringstyle=\color{codepurple},
    basicstyle=\ttfamily\footnotesize,
    breakatwhitespace=false,
    breaklines=true,
    captionpos=b,
    keepspaces=true,
    numbers=left,
    numbersep=5pt,
    showspaces=false,
    showstringspaces=false,
    showtabs=false,
    tabsize=2
}

\lstset{style=mystyle}

\geometry{margin=2.5cm}
\onehalfspacing

\begin{document}

\chapter{Task 12: Two-Level Paging}

\section{Task Description}
Implement two-level paging using 4KB pages. This involves setting up a hierarchical page table structure where the root table points to a child node, enabling finer-grained memory management.

\section{Planned Implementation}
\begin{enumerate}
    \item \textbf{Initialize the Root Page Table:}
    Declare the root page table and align it to 4096 bytes using the \texttt{\_\_attribute\_\_((aligned(4096)))} directive.
    \item \textbf{Create a Child Node:}
    Allocate a new page-aligned table to serve as the child node. Initialize the child node with all entries set to zero.
    \item \textbf{Update the Root Table:}
    Modify the root table to point to the child node for a specific entry (e.g., \texttt{raiz[2]}).
    \item \textbf{Create a Pointer:}
    Define a pointer (\texttt{ptr3}) that points to a virtual address 4MB beyond the range of the first-level table.
    \item \textbf{Test Access:}
    Attempt to read the content of \texttt{ptr3} and observe the behavior.
    \item \textbf{Document Results:}
    Record the observed behavior and compare it with the expected outcome.
\end{enumerate}

\section{Expected Outcome}
\begin{itemize}
    \item The system should attempt to access the second-level page table for the address pointed to by \texttt{ptr3}.
    \item Since the child node is initialized to zero, the access should result in a page fault.
\end{itemize}

\section{Implementation Details}
(To be filled after implementation.)

\section{Challenges}
(To be filled after implementation.)

\section{Final Outcome}
(To be filled after implementation.)

\end{document}

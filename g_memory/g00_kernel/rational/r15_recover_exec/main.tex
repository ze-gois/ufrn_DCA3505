\documentclass[a4paper,12pt]{report}

% Basic packages for LuaLaTeX with Unicode support
\usepackage{fontspec}
\usepackage[brazil]{babel}
\usepackage{graphicx}
\usepackage{hyperref}
\usepackage{amsmath}
\usepackage{amssymb}
\usepackage{listings}
\usepackage{xcolor}
\usepackage{geometry}
\usepackage{indentfirst}
\usepackage{setspace}

% Hyperref configuration
\hypersetup{
    colorlinks=true,
    linkcolor=blue,
    filecolor=magenta,
    urlcolor=cyan,
    pdftitle={Task 15: Recover Execution Without Modifying Root},
    pdfauthor={Your Name},
    pdfsubject={Memory Management in x86},
    pdfkeywords={paging, memory management, x86, recovery}
}

% Code listing styling
\definecolor{codegreen}{rgb}{0,0.6,0}
\definecolor{codegray}{rgb}{0.5,0.5,0.5}
\definecolor{codepurple}{rgb}{0.58,0,0.82}
\definecolor{backcolour}{rgb}{0.95,0.95,0.92}

\lstdefinestyle{mystyle}{
    backgroundcolor=\color{backcolour},
    commentstyle=\color{codegreen},
    keywordstyle=\color{magenta},
    numberstyle=\tiny\color{codegray},
    stringstyle=\color{codepurple},
    basicstyle=\ttfamily\footnotesize,
    breakatwhitespace=false,
    breaklines=true,
    captionpos=b,
    keepspaces=true,
    numbers=left,
    numbersep=5pt,
    showspaces=false,
    showstringspaces=false,
    showtabs=false,
    tabsize=2
}

\lstset{style=mystyle}

\geometry{margin=2.5cm}
\onehalfspacing

\begin{document}

\chapter{Task 15: Recover Execution Without Modifying Root}

\section{Task Description}
The goal of this task is to recover program execution without modifying the root page table. This involves implementing a solution specific to the program that allows execution to continue despite constraints on modifying the root.

\section{Planned Implementation}
\begin{enumerate}
    \item \textbf{Analyze the Problem:}
    \begin{itemize}
        \item Understand the limitations imposed by the root page table.
        \item Identify the specific scenarios where execution fails due to these limitations.
    \end{itemize}
    \item \textbf{Design a Solution:}
    \begin{itemize}
        \item Develop a mechanism to handle the failure scenarios without altering the root page table.
        \item This could involve:
        \begin{itemize}
            \item Using temporary page tables to handle specific memory regions.
            \item Dynamically mapping memory as needed during runtime.
            \item Leveraging exception handling to intercept and resolve access violations.
        \end{itemize}
    \end{itemize}
    \item \textbf{Implement the Solution:}
    \begin{itemize}
        \item Write the necessary code to implement the designed solution.
        \item Use the \texttt{invlpg()} function to invalidate specific TLB entries dynamically.
        \item Ensure the solution integrates seamlessly with the existing system.
    \end{itemize}
    \item \textbf{Test the Solution:}
    \begin{itemize}
        \item Create test cases to validate the solution under various scenarios.
        \item Simulate edge cases, such as accessing unmapped memory or overlapping regions.
        \item Ensure execution is successfully recovered in all cases.
    \end{itemize}
\end{enumerate}

\section{Expected Outcome}
\begin{itemize}
    \item The program should recover execution without requiring modifications to the root page table.
    \item The solution should be robust and handle all failure scenarios gracefully.
    \item The system should demonstrate stability and correctness under stress tests.
\end{itemize}

\section{Implementation Details}
(To be filled after implementation)

\section{Challenges}
(To be filled after implementation)

\section{Final Outcome}
(To be filled after implementation)

\end{document}

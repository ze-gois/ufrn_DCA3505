\documentclass[a4paper,12pt]{report}

% Basic packages for LuaLaTeX with Unicode support
\usepackage{fontspec}
\usepackage[brazil]{babel}
\usepackage{graphicx}
\usepackage{hyperref}
\usepackage{amsmath}
\usepackage{amssymb}
\usepackage{listings}
\usepackage{xcolor}
\usepackage{geometry}
\usepackage{indentfirst}
\usepackage{setspace}

% Hyperref configuration
\hypersetup{
    colorlinks=true,
    linkcolor=blue,
    filecolor=magenta,
    urlcolor=cyan,
    pdftitle={Task 13: Update Child Node for Access},
    pdfauthor={Your Name},
    pdfsubject={Memory Management in x86},
    pdfkeywords={paging, memory management, x86, exception handling}
}

% Code listing styling
\definecolor{codegreen}{rgb}{0,0.6,0}
\definecolor{codegray}{rgb}{0.5,0.5,0.5}
\definecolor{codepurple}{rgb}{0.58,0,0.82}
\definecolor{backcolour}{rgb}{0.95,0.95,0.92}

\lstdefinestyle{mystyle}{
    backgroundcolor=\color{backcolour},
    commentstyle=\color{codegreen},
    keywordstyle=\color{magenta},
    numberstyle=\tiny\color{codegray},
    stringstyle=\color{codepurple},
    basicstyle=\ttfamily\footnotesize,
    breakatwhitespace=false,
    breaklines=true,
    captionpos=b,
    keepspaces=true,
    numbers=left,
    numbersep=5pt,
    showspaces=false,
    showstringspaces=false,
    showtabs=false,
    tabsize=2
}

\lstset{style=mystyle}

\geometry{margin=2.5cm}
\onehalfspacing

\begin{document}

\chapter{Task 13: Update Child Node for Access}

\section{Task Description}
The goal of this task is to modify the exception handler to dynamically map addresses during a page fault. Specifically, when a page fault occurs for a pointer (\texttt{ptr3}), the handler should map the address to the same physical page as \texttt{ptr1} and \texttt{ptr2}. Other entries in the page table should remain absent.

\section{Planned Implementation}
\begin{enumerate}
    \item \textbf{Modify the Exception Handler:}
    \begin{itemize}
        \item Update the \texttt{page\_fault\_handler} function to handle page faults for \texttt{ptr3}.
        \item Dynamically map the faulting address to the same physical page as \texttt{ptr1} and \texttt{ptr2}.
    \end{itemize}
    \item \textbf{Set Up the Page Table:}
    \begin{itemize}
        \item Ensure the root page table (\texttt{raiz}) has an entry pointing to a child node for the second-level page table.
        \item Initialize the child node with all entries set to absent.
    \end{itemize}
    \item \textbf{Create Pointer \texttt{ptr3}:}
    \begin{itemize}
        \item Define a pointer \texttt{ptr3} that points to a virtual address 4MB ahead of \texttt{ptr2}.
    \end{itemize}
    \item \textbf{Trigger a Page Fault:}
    \begin{itemize}
        \item Attempt to read from \texttt{ptr3} to trigger a page fault.
    \end{itemize}
    \item \textbf{Map the Address:}
    \begin{itemize}
        \item In the exception handler, map the faulting address to the same physical page as \texttt{ptr1} and \texttt{ptr2}.
    \end{itemize}
    \item \textbf{Verify the Mapping:}
    \begin{itemize}
        \item Print the contents of \texttt{ptr1}, \texttt{ptr2}, and \texttt{ptr3} to confirm they all point to the same physical memory.
    \end{itemize}
\end{enumerate}

\section{Expected Outcome}
\begin{itemize}
    \item The exception handler dynamically maps the faulting address for \texttt{ptr3}.
    \item The contents of \texttt{ptr1}, \texttt{ptr2}, and \texttt{ptr3} are identical, confirming they point to the same physical memory.
    \item Other entries in the page table remain absent.
\end{itemize}

\section{Implementation Details}
(To be filled after implementation)

\section{Challenges}
(To be filled after implementation)

\section{Final Outcome}
(To be filled after implementation)

\end{document}

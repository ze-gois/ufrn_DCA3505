\documentclass[a4paper,12pt]{report}

% Basic packages for LuaLaTeX with Unicode support
\usepackage{fontspec}
\usepackage[brazil]{babel}
\usepackage{graphicx}
\usepackage{hyperref}
\usepackage{amsmath}
\usepackage{amssymb}
\usepackage{listings}
\usepackage{xcolor}
\usepackage{geometry}
\usepackage{indentfirst}
\usepackage{setspace}

% Hyperref configuration
\hypersetup{
    colorlinks=true,
    linkcolor=blue,
    filecolor=magenta,
    urlcolor=cyan,
    pdftitle={Task 06: Remove Mapping},
    pdfauthor={Your Name},
    pdfsubject={Memory Management in x86},
    pdfkeywords={paging, memory management, x86 architecture}
}

% Code listing styling
\definecolor{codegreen}{rgb}{0,0.6,0}
\definecolor{codegray}{rgb}{0.5,0.5,0.5}
\definecolor{codepurple}{rgb}{0.58,0,0.82}
\definecolor{backcolour}{rgb}{0.95,0.95,0.92}

\lstdefinestyle{mystyle}{
    backgroundcolor=\color{backcolour},
    commentstyle=\color{codegreen},
    keywordstyle=\color{magenta},
    numberstyle=\tiny\color{codegray},
    stringstyle=\color{codepurple},
    basicstyle=\ttfamily\footnotesize,
    breakatwhitespace=false,
    breaklines=true,
    captionpos=b,
    keepspaces=true,
    numbers=left,
    numbersep=5pt,
    showspaces=false,
    showstringspaces=false,
    showtabs=false,
    tabsize=2
}

\lstset{style=mystyle}

\geometry{margin=2.5cm}
\onehalfspacing

\begin{document}

\chapter{Task 06: Remove Mapping}

\section{Objective}
The goal of this task is to remove a memory mapping and observe the behavior when accessing unmapped memory. This involves:
\begin{itemize}
    \item Reading the content of a pointer (\texttt{ptr2}).
    \item Removing the mapping associated with \texttt{ptr2}.
    \item Attempting to read the content of \texttt{ptr2} again.
    \item Documenting the expected and actual results.
\end{itemize}

\section{Implementation Plan}
\begin{enumerate}
    \item \textbf{Setup:} Use the existing setup from the previous task where \texttt{ptr2} is mapped. Ensure \texttt{ptr2} points to a valid memory location initially.
    \item \textbf{Remove Mapping:} Modify the page table to remove the mapping for \texttt{ptr2}. Use the \texttt{invlpg()} function to invalidate the TLB for the unmapped address.
    \item \textbf{Access Unmapped Memory:} Attempt to read the content of \texttt{ptr2} after the mapping is removed. Observe and document the behavior.
    \item \textbf{Expected Results:} Accessing \texttt{ptr2} after the mapping is removed should trigger a page fault. The system should handle the page fault gracefully if an exception handler is installed.
\end{enumerate}

\section{Expected Outcome}
\begin{itemize}
    \item The program should demonstrate the effect of removing a mapping on memory access.
    \item A page fault should occur when accessing \texttt{ptr2} after the mapping is removed.
\end{itemize}

\section{Code Example}
Below is a simplified example of how the mapping removal might be implemented:
\begin{lstlisting}[language=C, caption=Removing a Mapping]
void remove_mapping(void *ptr) {
    // Modify the page table to remove the mapping
    page_table[INDEX(ptr)] = 0;

    // Invalidate the TLB for the specific address
    invlpg(ptr);
}
\end{lstlisting}

\section{Results}
(To be filled after implementation and testing.)

\section{Challenges}
(To be filled after implementation and testing.)

\section{Conclusion}
(To be filled after implementation and testing.)

\end{document}
